%%
%% Copyright (c) 2018-2019 Weitian LI <liweitianux@sjtu.edu.cn>
%% Creative Commons BY 4.0
%%

\chapter{再电离时期的探测}
\label{chap:detection}


\emph{\acf{eor}}是早期宇宙的一段缺乏了解的时期,目前的理论研究以及有限的观测证据表明
该时期从宇宙大爆炸之后约 \SI{300}{\Myr} 持续到约 \SI{1}{\Gyr},对应红移范围
约为 \numrange{6}{15} (参见 \citeay{koopmans2015} 及其所引文献).
充分探明并理解该时期是为进一步揭示更早的\ac{cd}和\ac{da}($z > 15$)、
建立完整的宇宙演化图景的关键环节.
在低频射电波段(约 \SIrange{50}{200}{\MHz})
探测源自再电离时期的 21\,cm 信号(即\emph{EoR 信号})
是目前研究该时期的最直接而有效的办法 \cite{madau1997,tozzi2000,furlanetto2006}.


%=====================================================================
\section{EoR 信号}
\label{sec:eor-signal}

%---------------------------------------------------------------------
\subsection{21\texorpdfstring{\,}{ }cm~谱线}
\label{sec:21cm-line}

中性氢原子的原子核(即质子)和电子均有 1/2 \ac{spin},
因此均具有相应的内禀磁矩 (intrinsic magnetic moment):
\begin{align}
  \B{\mu}_p & = |g_p| \frac{\acs{magneton-n}}{\acs{h-bar}} \B{S}_p , \\
  \B{\mu}_e & = - |g_e| \frac{\acs{magneton-b}}{\acs{h-bar}} \B{S}_e ,
\end{align}
其中
\ac{h-bar} 为\acl{h-bar},
$\ac{magneton-n} = e \,\ac{h-bar} / (2 \ac{mass-p})$ 为\acl{magneton-n},
$\ac{magneton-b} = e \,\ac{h-bar} / (2 \ac{mass-e})$ 为 \acl{magneton-b},
$g_p$ 和 $g_e$ 分别为质子和电子的 $g$ 因子,
$\B{S}_p$ 和 $\B{S}_e$ 分别为质子和电子的自旋.
电子带负电,所以磁矩方向与其自旋方向相反.

\begin{figure}[tbp]
  \centering
  \includegraphics[width=0.5\textwidth]{hydrogen-spinflip}
  \bicaption[氢原子的自旋翻转跃迁]{%
    氢原子在基态的两个超精细分裂能级之间发生自旋翻转跃迁,产生\acs*{21cmline}.
  }{%
    An hydrogen atom makes a spin-flip transition between the two
    hyperfine levels of the ground state, emitting the \acl*{21cmline}.
    \\\textcopyright{}
    Tiltec,
    \url{https://en.wikipedia.org/wiki/File:Hydrogen-SpinFlip.svg},
    (2019-03-31), 公有领域.
  }
  \label{fig:hi-spinflip}
\end{figure}

由于质子和电子的自旋相互作用,氢原子的\ac{ground-state}
发生\ac{hyperfine-splitting}成为两个态:
(1) 质子的自旋 $\B{S}_p$ 和电子的自旋 $\B{S}_e$ 平行,总角动量 $F = 1$;
(2) $\B{S}_p$ 和 $\B{S}_e$ 反平行,总角动量 $F = 0$.
想像两个共中心的载流线圈,系统的稳定状态(即能量最低)为两个线圈的电流方向相同,
即两个载流线圈的磁矩平行.
将此应用于上述两个态,可知 $F = 1$ 态(即质子和电子的自旋反平行、磁矩平行)
的能量比 $F = 0$ 态更高 \cite{griffiths1982}.
当氢原子从 $F = 1$ 态跃迁到 $F = 0$ 态时,
将产生频率约为 $\nu = \SI{1420}{\MHz}$ 的辐射,
对应的波长约为 $\lambda = \SI{21}{\cm}$,因此称为\emph{\acf{21cmline}}.
此跃迁过程亦称为\emph{\acf{sft}},如\autoref{fig:hi-spinflip} 所示.
对于 $F = 1$ 态(即上能级),磁量子数 $m$ 可取 $\{ -1, 0, 1 \}$,
因此这个态的\ac{dod}为 3,称为\emph{\acf{triplet}};
对于 $F = 0$ 态(即下能级),磁量子数 $m$ 只能取 0,
因此这个态是\emph{\acf{singlet}}.

中性氢的 \ac{21cmline}%
首次由 H.~C. van~de~Hulst 在 1945 年提出 \cite{vanDeHulst1945},
并由 H.~I. Ewen 和 E.~M. Purcell 在 1951 年观测到 \cite{ewen1951}.
该谱线的频率是目前测量最精确的几个物理量之一 \cite{hellwig1970,essen1971}:
\begin{equation}
  \label{eq:hi-line-frequency}
  \nu_0 = \SI{1420405751.7667 +- 0.0009}{\Hz} ,
\end{equation}
对应真空中的波长为:
\begin{equation}
  \label{eq:hi-line-wavelength}
  \lambda_0 = \SI{21.1061140542}{\cm} .
\end{equation}

%---------------------------------------------------------------------
\subsection{谱线亮温度}

中性氢的 \ac{21cmline}的跃迁概率非常低,其\ac{em-spontaneous}系数 $A_{21}$ 为:
\begin{equation}
  A_{21} = \SI[separate-uncertainty=false]{2.86888(7)e-15}{\per\second} .
\end{equation}
对应的\ac{em-spontaneous}的半衰期为:
\begin{equation}
  t_{1/2} \approx 1 / A_{21}
    = \SI{3.49e14}{\second} \approx \SI{11.1}{\Myr} .
\end{equation}
该\ac{timescale}远大于典型中性氢云中的氢原子因碰撞而改变自旋的\ac{timescale},
因此中性氢的超精细结构的能级布居将由碰撞决定.

由\autoref{eq:t-excitation-def}可知,\acl{T-excitation}描述了能级的相对布居.
鉴于\ac{21cmline}源自中性氢的\ac{sft},在此情况下,
\acl{T-excitation}亦被称为\emph{\acf{T-spin}},
描述了氢原子自旋态(即\ac{singlet}和\ac{triplet})的相对布居 \cite{field1958}:
\begin{equation}
  \frac{N_2}{N_1} = \frac{g_2}{g_1}
    \exp\left( -\frac{\acs{hp} \nu_0}{\acs{kb} \acs{T-spin}} \right)
    = \frac{g_2}{g_1} \exp\left( -\frac{T_*}{\acs{T-spin}} \right),
\end{equation}
其中
$g_1 = 1$ 和 $g_2 = 3$ 分别为两个能级的\ac{dod},
$T_* \equiv \acs{hp} \nu_0 / \acs{kb} \approx \SI{68.2}{\mK}$.
在实际的天体物理应用中,均有 $\acs{T-spin} \gg T_*$,因此:
\begin{equation}
  \frac{N_2}{N_1} = \frac{g_2}{g_1} = 3 .
\end{equation}
将此代入\autoref{eq:coef-absorption-einstein3},
近似后可得中性氢云的\acl{coef-absorption}:
\begin{equation}
  \acs{coef-absorption}
    = \frac{3 \acs{hp} \acs{speed-light}^2}{32\Cpi \nu_0}
      \frac{A_{21} n_{\R{HI}}}{\acs{kb} \acs{T-spin}} \phi(\nu) ,
\end{equation}
其中 $n_{\R{HI}} = N_1 + N_2 = 4 N_1$ 为中性氢的数密度.

设一团均匀的中性氢云位于红移 $z$ 处,沿观测的视线方向的线性尺度为 $s$,
中性氢的数密度为 $n_{\R{HI}}$,
则其\acl{optical-depth} [\autoref{eq:optical-depth}] 为:
\begin{align}
  \acs{optical-depth}_{\nu0}
    & = \int_{\R{cloud}} \acs{coef-absorption}(s') \,\D{s'}  \\
    & = \frac{3 \acs{hp} \acs{speed-light}^2}{32\Cpi \nu_0}
      \frac{A_{21}}{\acs{kb} \acs{T-spin}} \phi(\nu)
      \int_{\R{cloud}} n_{\R{HI}}(s') \,\D{s'}  \\
    & = \frac{3 \acs{hp} \acs{speed-light}^2}{32\Cpi \nu_0}
      \frac{A_{21}}{\acs{kb} \acs{T-spin}} N_{\R{HI}} \phi(\nu) ,
  \label{eq:21cm-optical-depth1}
\end{align}
其中 $N_{\R{HI}}$ 为中性氢的\ac{column-density},可进一步写成:
\begin{equation}
  \label{eq:column-density}
  N_{\R{HI}} = \acs{hi-fraction} n_{\R{HI}} s ,
\end{equation}
其中 \acs{hi-fraction} 为\acl{hi-fraction} (neutral fraction of hydrogen).

一般说来,\ac{line-profile}包含自然展宽 (natural broadening)、
热展宽 (thermal broadening)、压力展宽 (pressue broadening)、
体运动展宽 (bulk motion broadening) 等因素,
但对于 \ac{21cmline}来说,最重要的因素是宇宙膨胀引起的 Doppler 展宽.
因此,线性尺度为 $s$ 的中性氢云的速度弥散约为 $\Delta v \sim s \acs{Hz}$,
于是\ac{line-profile}可近似为:
\begin{equation}
  \phi(\nu) \sim \frac{\acs{speed-light}}{\nu \Delta v}
    \sim \frac{\acs{speed-light}}{\nu s \acs{Hz}} .
\end{equation}
将上式和\autoref{eq:column-density} 代入\autoref{eq:21cm-optical-depth1},
可得\acl{optical-depth}:
\begin{equation}
  \label{eq:21cm-optical-depth2}
  \acs{optical-depth}_{\nu0}
    \approx \frac{3 \acs{hp} \acs{speed-light}^3}{32\Cpi \nu_0^2}
      \frac{A_{21}}{\acs{kb} \acs{T-spin}}
      \frac{\acs{hi-fraction} n_{\R{HI}}}{\acs{Hz}} .
\end{equation}
如果需要考虑中性氢云的\ac{v-peculiar},则其\acl{optical-depth}应为
\cite{furlanetto2006,pritchard2012}:
\begin{equation}
  \label{eq:21cm-optical-depth3}
  \acs{optical-depth}_{\nu0}
    \approx \frac{3 \acs{hp} \acs{speed-light}^3}{32\Cpi \nu_0^2}
      \frac{A_{21}}{\acs{kb} \acs{T-spin}}
      \frac{\acs{hi-fraction} n_{\R{HI}}}{
        (1+z) (\partial v_{\parallel} / \partial r_{\parallel})} ,
\end{equation}
其中 $\partial v_{\parallel} / \partial r_{\parallel}$ 是\ac{v-proper}
沿视线方向的梯度.
此外,通常均有\acl{optical-depth} $\acs{optical-depth}_{\nu0} \ll 1$
\cite{madau1997,furlanetto2006,pritchard2010mn}.

\begin{figure}[tbp]
  \centering
  \includegraphics[width=0.9\textwidth]{21cm-radiative-transfer}
  \bicaption[CMB 辐射穿过中性氢云的辐射转移示意图]{%
    CMB 辐射穿过中性氢云的辐射转移示意图:
    从\acl*{T-spin}为 \acs*{T-spin} 的中性氢云出射的
    CMB 辐射的温度为 \acs*{T-b}.
  }{%
    An illustration of the radiative transfer process:
    the CMB radiation goes through a cloud of hydrogen with
    spin temperature \acs*{T-spin} and emerges with a temperature
    \acs*{T-b} measured by a telescope.
    \\\textcopyright{}
    \citeay{zaroubi2013}, \S\,4.1.
    [经过了左右翻转]
  }
  \label{fig:21cm-cmb-rt}
\end{figure}

在实际观测中,需要考虑 \ac{cmb} 辐射穿过中性氢云的转移过程,
如\autoref{fig:21cm-cmb-rt} 所示.
考虑一团处于热平衡状态、\acl{T-spin}为 \acs{T-spin}、位于红移 $z$ 处的中性氢云,
结合 Kirchhoff 定律 [\autoref{eq:kirchhoff-law}] 可得\ac{rt}方程为:
\begin{equation}
  \diff{\acs{I-nu}}{s}
    = -\acs{coef-absorption}\,\acs{I-nu} + \acs{coef-emission}
    = -\acs{coef-absorption}\,[\acs{I-nu} - B_{\nu}(\acs{T-spin})] ,
\end{equation}
其中
$B_{\nu}(\acs{T-spin})$ 是温度为 \acs{T-spin} 的黑体辐射谱 [\autoref{eq:planck}].
利用 Rayleigh--Jeans 近似 [\autoref{eq:rj-approx}]
将 \acs{I-nu} 表示为\acl{T-b} $\acs{T-b}(\nu)$:
\begin{equation}
  \diff{\acs{T-b}}{s}
    = -\acs{coef-absorption} [\acs{T-b} - \acs{T-spin}] .
\end{equation}
根据\acl{optical-depth}的定义 [\autoref{eq:optical-depth}]
有 $\D{\tau'} = -\acs{coef-absorption}\,\D{s}$,
代入上式可得:
\begin{equation}
  \diff{\acs{T-b}}{\tau'} = \acs{T-b} - \acs{T-spin} .
\end{equation}
上式两边均乘上 $\Ce^{-\tau'}\,\D{\tau'}$ 并积分:
\begin{equation}
  \acs{T-b}\,\Ce^{-\tau'}\,\Big|_0^{\tau_{\nu0}}
      - \int_0^{\tau_{\nu0}} \acs{T-b}\,\D{(\Ce^{-\tau'})}
    = \int_0^{\tau_{\nu0}} \acs{T-b}\,\Ce^{-\tau'}\,\D{\tau'}
      + \acs{T-spin} \int_0^{\tau_{\nu0}} \Ce^{-\tau'}\,\D{\tau'} ,
\end{equation}
可得:
\begin{equation}
  \acs{T-b}(\tau'=\tau_{\nu0})\,\Ce^{-\tau_{\nu}} - \acs{T-b}(\tau'=0) =
    \acs{T-spin} (\Ce^{-\tau_{\nu}} - 1) ,
\end{equation}
其中 $\acs{T-b}(\tau'=\tau_{\nu}) = T_{\R{cmb}}(z)$
为辐射入射中性氢云时的亮温度,
于是获得从中性氢云出射 ($\tau = 0$) 的辐射亮温度 $\acs{T-b}(\nu_0)$ 为:
\begin{equation}
  \acs{T-b}(\nu_0)
    = \acs{T-spin} (1 - \Ce^{-\tau_{\nu0}})
      + T_{\R{cmb}}(z) \,\Ce^{-\tau_{\nu0}} .
\end{equation}
其中 $T_{\R{cmb}}(z)$ 为 \ac{cmb} 辐射在红移 $z$ 时的\ac{comoving-frame}
中的亮温度:
\begin{align}
  T_{\R{cmb}}(z)
    & = T_{\R{cmb}}(z=0)\,(1+z)  \\
    & = 2.73 (1+z) \quad [\si{\kelvin}] ,
\end{align}
类似地,宇宙膨胀的红移效应将中性氢的 \ac{21cmline}红移至频率 $\nu = \nu_0/(1+z)$,
同时观测到的中性氢云的亮温度将为:
\begin{equation}
  T_b^{\R{obs}}(\nu) = \frac{T_b(\nu_0)}{1 + z} .
\end{equation}
目前观测 \ac{21cmline}的策略是测量其相对于 \ac{cmb} 背景辐射的差异,
即\emph{\acf{dT-b}}.
综上,并结合\autoref{eq:21cm-optical-depth3},最终可得:
\begin{align}
  \acs{dT-b}
    & = \frac{T_b(\nu_0)}{1 + z} - T_{\R{cmb}}(z=0)  \\
    & = \frac{\acs{T-spin} - T_{\R{cmb}}(z)}{1+z}
      (1 - \Ce^{-\acs{optical-depth}_{\nu0}})  \\
    & \approx \frac{\acs{T-spin} - T_{\R{cmb}}(z)}{1+z}
      \acs{optical-depth}_{\nu0}  \\
    & \approx \frac{3 \acs{hp} \acs{speed-light}^3 A_{21}}{
      32\Cpi \nu_0^2 \acs{kb}}
      \frac{\acs{hi-fraction} n_{\R{HI}}}{
        (1+z) (\partial v_{\parallel} / \partial r_{\parallel})}
      \left[ 1 - \frac{T_{\R{cmb}}(z)}{\acs{T-spin}} \right] .
    \label{eq:eor-signal}
\end{align}

%---------------------------------------------------------------------
\subsection{亮温度的演化}

由\autoref{eq:eor-signal} 易知,\acl{T-spin} \acs{T-spin}
对 EoR 信号 $\acs{dT-b}(\nu)$ 的强度起关键作用.
当 $\acs{T-spin} > T_{\R{cmb}}(z)$ 时,$\acs{dT-b}(\nu) > 0$ 为发射信号,
并且当 $\acs{T-spin} \gg T_{\R{cmb}}(z)$ 时 $\acs{dT-b}(\nu)$ 达到饱和;
当 $\acs{T-spin} < T_{\R{cmb}}(z)$ 时,$\acs{dT-b}(\nu) < 0$ 表现为吸收信号.
\acl{T-spin} \acs{T-spin} 主要由以下三个过程决定:
\begin{itemize}
  \item CMB 辐射光子被中性氢吸收或者使中性氢产生\ac{em-stimulated},
    该过程使得 \acs{T-spin} 与 $T_{\R{cmb}}(z)$ 相互耦合;
  \item 中性氢原子与其他原子或电子碰撞,
    这个过程将 \acs{T-spin} 耦合到介质的\acl{T-kinetic} \acs{T-kinetic};
  \item 中性氢原子与 Lyα 光子发生\ac{resonant-scattering}而改变自旋状态,
    即 Wouthuysen--Field 效应 \cite{wouthuysen1952,field1958},
    该过程将 \acs{T-spin} 与 Lyα 辐射的\ac{T-color}关联起来.
\end{itemize}

\begin{figure}[tbp]
  \centering
  \includegraphics[width=\textwidth]{eor-signal-evolution}
  \bicaption[EoR 信号的平均强度的演化示意图]{%
    \uline{(上栏)}宇宙学模拟\cite{santos2008}给出的 EoR 信号的演化过程.
    \uline{(下栏)}理论模型给出的 EoR 信号的全天平均强度
    $\delta\overline{T}_b$ 的变化过程.
  }{%
    \emph{(Upper)} The time evolution of the EoR signal dervied from
    a cosmological simulation \cite{santos2008}.
    \emph{(Lower)} The expected evolution of the sky-averaged EoR signal
    $\delta\overline{T}_b$.
    \\\textcopyright{}
    \citeay{pritchard2012}, \S\,1.
  }
  \label{fig:eor-signal-evo}
\end{figure}

上述过程与宇宙的再电离过程、第一代天体的形成、星系的形成与演化等环节密切相关,
因此 EoR 信号随宇宙的演化而发生复杂的变化,并反映在其频谱上
(不同的频率对应于不同的红移,即宇宙年龄),如\autoref{fig:eor-signal-evo} 所示.
EoR 信号的完整演化过程可大致分为以下几个阶段 \cite{pritchard2012}:
\begin{enumerate}
  \item $200 \lesssim z \lesssim 1100$:
    \ac{recomb}之后残留的自由电子通过 Compton 散射使气体与 \ac{cmb} 保持热耦合,
    即有 $\acs{T-kinetic} = T_{\R{cmb}}$.
    因气体密度高,有效的碰撞使得 $\acs{T-spin} = T_{\R{cmb}}$,
    所以此阶段没有可探测的 EoR 信号 ($\delta\overline{T}_b = 0$).

  \item $40 \lesssim z \lesssim 200$:
    气体绝热冷却 $\acs{T-kinetic} \propto (1+z)^2$
    从而有 $\acs{T-kinetic} < T_{\R{cmb}}$,但气体中的碰撞仍然足够有效,
    因此 $\acs{T-spin} = \acs{T-kinetic} < T_{\R{cmb}}$,
    所以 EoR 信号在此阶段为吸收信号 ($\delta\overline{T}_b < 0$).

  \item $z_* \lesssim z \lesssim 40$:
    ($z_*$ 对应第一代天体形成的时刻.)
    气体密度随着宇宙膨胀而显著降低,导致碰撞耦合不如 \ac{cmb} 辐射的耦合有效,
    因此有 $\acs{T-spin} = T_{\R{cmb}}$,所以此阶段无可探测的 EoR 信号.

  \item $z_{\alpha} \lesssim z \lesssim z_*$:
    ($z_{\alpha}$ 对应 Lyα 光子的耦合达到饱和的时刻.)
    第一代天体开始形成,辐射 Lyα 光子和 X 射线,
    并将 \acs{T-spin} 与冷气体的温度耦合在一起,
    即 $\acs{T-spin} \sim \acs{T-kinetic} < T_{\R{cmb}}$,
    因此该阶段可探测到 EoR 吸收信号 ($\delta\overline{T}_b < 0$).

  \item $z_h \lesssim z \lesssim z_{\alpha}$:
    ($z_h$ 对应气体被加热至与 $T_{\R{cmb}}$ 相同温度的时刻.)
    随着星系和宇宙大尺度结构的形成,气体逐渐被加热,
    对 \acs{T-spin} 的影响也越来越显著.
    在气体温度 $\acs{T-kinetic} \sim \acs{T-spin} < T_{\R{cmb}}$
    的这一阶段,可继续探测到 EoR 吸收信号 ($\delta\overline{T}_b < 0$).

  \item $z_t \lesssim z \lesssim z_h$:
    随着气体被继续加热,$\acs{T-kinetic} \sim \acs{T-spin} > T_{\R{cmb}}$,
    因此 EoR 信号从上一阶段的吸收信号转变为发射信号 ($\delta\overline{T}_b > 0$).
    直到 $z_t$ 时刻,气体的温度已足够高
    $\acs{T-kinetic} \sim \acs{T-spin} \gg T_{\R{cmb}}$,
    EoR 信号的强度达到饱和.

  \item $z_r \lesssim z \lesssim z_t$:
    在此阶段,气体的温度 ($\acs{T-kinetic} \sim \acs{T-spin} \gg T_{\R{cmb}}$)
    对 EoR 信号的影响已变得不重要,
    因此 EoR 信号的强度将主要取决于\acl{hi-fraction} \acs{hi-fraction}.
    随着再电离的进行,\acs{hi-fraction} 逐渐变小,EoR 信号的强度也随之减小,
    直到 $z_r$ 时再电离结束.

  \item $z \lesssim z_r$:
    再电离结束后,残留的 21\,cm 信号主要源自一些塌缩的中性氢岛,比如\ac{dla}.
\end{enumerate}

由于缺乏足够的观测证据的约束,上述阶段的划分仍有很大的不确定性,
相邻阶段之间可能有显著交叠,甚至 $z_{\alpha}$ 和 $z_h$ 的次序可能需要修正
\cite{nusser2005,pritchard2012}.
基于目前非常有限的观测证据,理论模型显示宇宙的再电离过程约从红移 $z \sim 15$ 开始,
应在红移 $z > 6.5$ 之前完成,但也不能显著早于 $z \sim \numrange{7}{8}$
\cite{choudhury2006,pritchard2010mn}.


%=====================================================================
\section{探测方法}
\label{sec:det-methods}

目前有三种测量 EoR 信号的方法,由易到难分别为:
(1) 测量全天总功率;
(2) 测量功率谱;
(3) 直接获取再电离区域的图像.

第一种方法仅测量 EoR 信号的全天总功率随红移(即观测频率)的变化,
所得结果可以用于推断物质的电离过程,帮助检验和约束再电离模型
\cite{pritchard2012,liu2016}.
该方法相对简单易行,通常采用小型专用设备,一般包含单个或少量天线.
目前已有一批采用该方法的 EoR 探测实验,主要包括
位于澳大利亚的 \ac{edges} \cite{bowman2008} 和
\ac{bighorns} \cite{sokolowski2015}、
位于美国的 \ac{leda} \cite{greenhill2012}、
位于墨西哥的 \ac{sci-hi} \cite{voytek2014}
以及位于印度的 \ac{saras} \cite{singh2018}.
值得一提的是,\acs{edges} 在 2018 年初报导称发现全天平均射电信号在 \SI{78}{\MHz}
附近存在吸收,该吸收信号所处位置大致符合早期恒星所引发的 21\,cm 信号,
但其强度是目前理论预测值的两倍以上 \cite{bowman2018}.

后两种方法则进一步测量 EoR 信号的统计分布规律甚至三维图像,能够提供更加全面丰富
的信息用于系统性地研究\acl{eor}.
尽管这两种测量方法更加强大有效,但需要大型低频干涉阵列,
如 \autoref{sec:instruments} 所介绍的主要干涉阵列,
其中仅有 \acs{ska} 将拥有足够高的灵敏度实现对再电离区域的直接成像观测.


%=====================================================================
\section{主要困难}
\label{sec:det-difficulties}

EoR 探测实验,尤其是采用干涉阵列,面临着一系列困难.
这些困难可主要分为以下几类:
\begin{itemize}
\item
\emph{前景干扰:}
源自银河系以及河外源的前景辐射非常强烈,可达数百 \si{\kelvin},
是 EoR 信号(仅约几 \si{\mK} 至几十 \si{\mK})的 \numrange{4}{5} 个数量级.
虽然对干涉阵列而言重要的是辐射的空间涨落幅度而非其平均强度,
但是前景辐射的涨落幅度仍达数 \si{\kelvin} 到数十 \si{\kelvin},
远远压制了待测 EoR 信号 \cite{zaroubi2013}.
\autoref{fig:eor-foregrounds} 显示了主要的前景成分及其在
\SI{120}{\MHz} 处的强度.
因此,即便是轻微的前景处理不当,都会导致微弱的 EoR 信号被淹没而无法被捕捉到.
此外,部分前景成分(如银河系\ac{rad-syn})存在显著偏振,
该偏振成分可能发生泄漏而影响前景强度的测量,即\ac{pl}效应 (ref???),
导致前景的频谱结构复杂化而变得更加难以处理 (ref???).
如何处理强烈的前景干扰并成功分离 EoR 信号,
是目前 EoR 探测领域的一个关键任务,
不仅需要系统深入地理解前景的特征 \cite{offringa2016,carroll2016,procopio2017} (ref???),
还需要研发有效的前景扣除与信号分离算法 \cite{chapman2015,chapman2016} (ref???).

\begin{figure}[tbp]
  \centering
  \includegraphics[width=0.7\textwidth]{eor-foregrounds}
  \bicaption[主要前景成分及其强度示意图]{%
    主要前景成分以及强度示意图.
    图中的数值代表在 \SI{120}{\MHz} 处的\ac{rms}值.
  }{%
    A diagram showing the major foreground components contaminating
    the EoR signal.
    The numbers in the figure represent the \acs{rms} values
    at \SI{120}{\MHz}.
    \\\textcopyright{}
    \citeay{zaroubi2013}.
  }
  \label{fig:eor-foregrounds}
\end{figure}

%.......................................
\item
\emph{人工源的\acl{rfi}:}
随着科技的进步和社会的发展,人类活动产生的无线电波已在地球上无处不在,
对射电天文观测产生了严重的\acf{rfi}.
这些人工源主要有:\ac{am}和\ac{fm}广播、卫星通信、\ac{gps}信号、
对讲机、手机、移动通信基站、航空通信、雷达、等等.
虽然 EoR 探测设备通常建设在人烟稀少的射电宁静区域,但是仍不可避免受到
人工源的 \ac{rfi},甚至由月亮以及太空碎片反射回来的无线电波都可能
对 EoR 观测产生一定程度的影响 \cite{mckinley2013,tingay2013rfi}.
如\autoref{fig:rfi-mwa} 所示的是 MWA 在其各子频段的\ac{vis}数据
被标记为 \ac{rfi} 的比例,其中突显了\ac{fm}广播、卫星通信以及数字电视
等干扰源对 EoR 探测所造成的影响.
\ac{rfi} 的强度通常会高出天空信号的若干个数量级 \cite{bentum2011} (ref???),
并且实时发生变化.
目前的主要办法是识别并屏蔽存在明显 \ac{rfi} 的时间和频率片段
\cite{offringa2010,offringa2012,prasad2012},
但是残留的干扰可能会对前景处理以及 EoR 信号测量均产生严重影响 \cite{offringa2015}.

\begin{figure}[tbp]
  \centering
  \includegraphics[width=0.7\textwidth]{RFI-MWA}
  \bicaption[MWA 各子频段内的 RFI 比例]{%
    MWA 各子频段的\acs*{vis}数据被标记为 \acs*{rfi} 的比例.
  }{%
    The \acs*{rfi} occupancy, calculated as the percentage of
    visibilities that are detected as \acs*{rfi} by the flagger,
    per sub-band for the MWA.
    \\\textcopyright{}
    \citeay{offringa2015}.
  }
  \label{fig:rfi-mwa}
\end{figure}

%.......................................
\item
\emph{电离层干扰:}
\acf{ionosphere}是地球大气层上部被太阳辐射电离的部分,从约 \SI{60}{\km}
延伸至约 \SI{1000}{\km} 的高空,覆盖了大气层的\ac{thermosphere}以及
部分\ac{mesosphere}和\ac{exosphere},是地球\ac{magnetosphere}的内界
(如\autoref{fig:ionosphere} 所示).
电离层的大气已经非常稀薄,因此被太阳辐射中的紫外线和 X 射线电离的
空气分子所产生的自由电子在复合前可以短暂地自由活动,形成等离子体,能够对电磁波的
传播产生影响.
在 \SI{300}{\MHz} 的低频波段,电离层主要对电磁波产生折射、
传播延迟、Faraday 旋转等影响,导致测量数据存在相位和幅度误差
\cite{intema2009,thompson2017}.
由于主要受太阳活动的影响,电离层的状态会随时间和位置而发生剧烈变化,
因此对干涉阵列各天线产生的干扰程度也存在差异且时刻发生变化.
为了高质量的图像,必须实时(分钟量级???)校准观测数据 (ref???),
而且对每个天线施加的校准需要有针对性 (ref???),这将成为一个严重的计算负担,
还需要发展更加有效的电离层校准算法 \cite{intema2009,deGasperin2018}.

\begin{figure}[tbp]
  \centering
  \includegraphics[width=0.5\textwidth]{atmosphere-with-ionosphere}
  \bicaption[大气层和电离层的关系]{%
    地球的大气层和电离层之间的关系.
    电离层是大气层上部被太阳辐射电离的部分.
  }{%
    The relation between Earth's atmosphere and ionosphere, which
    is the ionized part of upper atmosphere.
    \\\textcopyright{}
    Bhamer,
    \url{https://en.wikipedia.org/wiki/File:Atmosphere_with_Ionosphere.svg},
    (2018-10-13), 公有领域.
  }
  \label{fig:ionosphere}
\end{figure}

%.......................................
\item
\emph{仪器效应:}
当代的干涉阵列通常由成千上万根天线组成. 由于生产和安装过程的差异以及随环境和时间的变化,
每根天线的性能都不可能完全相同,导致所形成的\ac{stb}存在很多不确定因素,
而且各个\ac{station}的波束也互不相同.
对于采用数字\ac{bf}技术的\ac{pa}而言,波束的形态更会随着所指方向而发生大幅变化
\cite{smirnov2011iii,vanWeeren2016,jagannathan2017}.
因此,如果未能全面地校准\ac{stb},那么后续对其他仪器效应的校准、亮点源剥离、
前景去除等任务都会受到严重影响 \cite{noordam2004,neben2016}.
此外,还有一系列已知和未知的复杂仪器效应,比如:
显著的旁瓣 \cite{thyagarajan2015,mort2017}、
波束的频率依赖效应 \cite{liu2009ps,datta2010,morales2012}
(另见 \autoref{sec:eor-window} 和 \autoref{sec:fdeffect})、
\ac{pl} \cite{asad2016,asad2018,lenc2017}、
天线响应随频率的变化 \cite{bernardi2015,trott2017}、
信号传输过程中在电缆内的反射 \cite{beardsley2016}.
如何准确有效地校准仪器,发挥出仪器的设计性能,是目前最迫切的任务之一
\cite{wijnholds2010} (ref???).

%.......................................
\item
\emph{海量数据:}
大型的干涉阵列将产生海量数据,如 SKA1-Low 的数据流量预计高达 TB/s,
由此引发出一系列难题 \cite{norris2011} (ref???),比如:
如何对原始数据进行实时相关处理?
如何传输和存储如此海量的数据?
如何实现有效的数字\ac{bf}和多波束技术?
如何进行海量数据的校准处理?
如何处理海量数据实现大视场高动态范围成像?
缓解或解决这些问题,不仅依赖于更快更高效的计算资源 \cite{magro2014,vermij2017},
建设新型的数据中心 \cite{chrysostomou2018},
还需要研发新算法以及编写新软件,优化数据处理流程,充分利用大规模并行计算资源
\cite{morales2009,gunst2018} (ref???).

\end{itemize}


%=====================================================================
\section{主要前景成分}
\label{sec:fg-intro}

%---------------------------------------------------------------------
\subsection{银河系同步辐射}

\ac{rad-syn} ...
polarization leakage ...
spectral index variation ...

%---------------------------------------------------------------------
\subsection{银河系轫致辐射}

\ac{rad-brem} ...
TODO

%---------------------------------------------------------------------
\subsection{河外点源}

\ac{src-point} ...
TODO

clustering effect ...

%---------------------------------------------------------------------
\subsection{星系团}

\ac{gc} ...
radio halos, relics, mini-halos ...

\subsubsection{射电晕}

\ac{rh} ...
radio halos ...

\subsubsection{射电遗迹}

\ac{rr} ...
radio relics ...

\subsubsection{迷你射电晕}

\ac{rmh} ...
radio mini-halos ...

%---------------------------------------------------------------------
\subsection{超星系团和大尺度纤维状结构}

\ac{sc} ...
\ac{lsf} ...
intergalactic medium (virial shocks) ...
superclusters, large-scale filaments ...


%=====================================================================
\section{前景处理方法}
\label{sec:fg-methods}

key characteristic: frequency structure difference between
the 21~cm signal and foreground emission.

Exploit a single, common spectral property that distinguishes the
foregrounds from the 21\,cm signal: each foreground components has
a slowly varying power-law-like spectrum.
This results in a large spectral coherence scale and is in contrast
to the redshifted 21\,cm signal from the reionization epoch, which
fluctuates relatively rapidly in all three spatial dimensions, and
thus has a short coherence scale, both in frequency and angle.
about 10 Mpc, sub-MHz, sub-degree fluctuations ...
\cite{morales2004,bowman2009}

导致问题更加困难的是,我们并不知道 EoR 信号的确切性质,只清楚...

(参见 \citeay{chapman2016} 及其所引文献)

%---------------------------------------------------------------------
\subsection{前景扣除法}
\label{sec:fgrm}

\ac{fgrm} ...
parametric approaches, non-parametric approaches ...

%---------------------------------------------------------------------
\subsection{前景回避法}
\label{sec:fgavd}

\ac{fgavd} ...
2D power spectrum, EoR window


%=====================================================================
\section{再电离窗口}
\label{sec:eor-window}

EoR window, foreground wedge, explanation ...

The EoR signal is the redshifted 21\,cm line emission from \ac{hi},
which is observed by a radio interferometer and form an image cube
$I(\B{\theta}, \nu)$, where the two angular dimensions $\B{\theta}$
translate into the transverse distances $\B{r}_{\bot}$ on the sky plane,
and the spectral dimension $\nu$ represents the line-of-sight distance
$r_{\parallel}$.

First, we take the 3D Fourier transform on the image cube:
\begin{equation}
  \label{eq:ps-3d-ft}
  V(\B{u}, \eta) = \B{\R{F}}(\{\B{u}, \eta\}, \{\B{\theta}, \nu\})
    I(\B{\theta}, \nu),
\end{equation}
where $\B{\theta}$ is the angular sizes on the sky plane, $\nu$ is the
frequency, and $(\B{u}, \eta)$ are the Fourier duals to
$(\B{\theta}, \nu)$, respectively.
Then we transform the coordinate from $(\B{u}, \eta)$ into the
cosmological coordinate $\B{k} = (k_x, k_y, k_z)$
(in units of \si{\per\cMpc}):
\begin{equation}
  \label{eq:coordinate-transform}
  V(\B{k}) = \M{J}(\B{k}, \{\B{u}, \eta\}) V(\B{u}, \eta).
\end{equation}
(coordinate transform ...)
We therefore obtain the 3D power spectrum by:
\begin{equation}
  \label{eq:3d-ps}
  P(\B{k}) = |V(\B{k})|^2.
\end{equation}

Considering that the signal is isotropic among the sky plane, we can
squeeze these two dimensions into $\kperp \equiv \sqrt{k_x^2 + k_y^2}$ by
cylindrically averaging the 3D power spectrum, and obtained the 2D
cylindrical power spectrum $P(\kperp, \klos)$ with $\klos \equiv k_z$.
The 2D cylindrical power spectrum has the advantage to better separate
the EoR signal from the foreground contamination, which is supposed to
be reside in the lower-right wedge-shape region, therefore defining the
EoR window.


%=====================================================================
\section{小结}

TODO


%% EOF
