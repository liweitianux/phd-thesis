%%
%% Copyright (c) 2018-2019 Weitian LI <liweitianux@sjtu.edu.cn>
%% Creative Commons BY 4.0
%%

\chapter{射电干涉技术基础}
\label{chap:interferometry}

主要参考资料:
\citeay{condon2016}, \citeay{clark1999}, \citeay{thompson1999},
\citeay{thompson2017}, \citeay{wilson2014} ...

%=====================================================================
\section{射电天文学简介}
\label{sec:radio-astronomy}

我们对宇宙的几乎所有认识都来自于观测并研究

%---------------------------------------------------------------------
\subsection{射电天文学是什么?}

TODO

%---------------------------------------------------------------------
\subsection{射电窗口}

TODO

\begin{figure}[!htp]
  \centering
  \includegraphics[width=\textwidth]{atmospheric-em-transmittance}
  \bicaption[大气层的电磁辐射透射率]{%
    大气层的电磁辐射透射率随波长(即频率)的变化。
    除了光学窗口,大气层还有一个更加宽广的射电窗口,
    从 $\lambda \sim \SI{1}{\mm} (\nu \sim \SI{300}{\GHz})$
    延伸到 $\lambda \sim \SI{20}{\meter} (\nu \sim \SI{15}{\MHz})$。
  }{%
    Electromagnetic transmittance of the Earth's atmosphere.
    In addition to the visible optical window, there is another
    much wider radio window, which spans from
    $\lambda \sim \SI{1}{\mm} (\nu \sim \SI{300}{\GHz})$
    to $\lambda \sim \SI{20}{\meter} (\nu \sim \SI{15}{\MHz})$.
    \\\textcopyright{}
    \citeay{condon2016}, \S\,1.1.2.
  }
  \label{fig:atmospheric-emt}
\end{figure}

%---------------------------------------------------------------------
\subsection{机遇和挑战}

TODO


%=====================================================================
\section{辐射理论}
\label{sec:radiation}

TODO ...
spectral brightness $I_{\nu}$

%---------------------------------------------------------------------
\subsection{亮度和流量密度}

In the radio frequency regime, the Rayleigh-Jeans approximation
always holds, therefore the spectral brightness $I_{\nu}$ can be
equivalently expressed by \emph{brightness temperature} $T_b(\nu)$
through the relation \cite{condon2016}:
\begin{equation}
  \label{eq:Tb}
  T_b(\nu) \equiv \frac{I_{\nu} c^2}{2 k_B \nu^2}.
\end{equation}

%---------------------------------------------------------------------
\subsection{黑体辐射和亮温度}

TODO


%=====================================================================
\section{天线原理}
\label{sec:antenna}

TODO

%---------------------------------------------------------------------
\subsection{辐射方向图}

TODO

%---------------------------------------------------------------------
\subsection{增益和阻抗}

TODO

%---------------------------------------------------------------------
\subsection{主瓣和旁瓣}

TODO

ERA: eq:(3.96,3.118)

%---------------------------------------------------------------------
\subsection{有效面积}

TODO

%---------------------------------------------------------------------
\subsection{互易定理}

(???) TODO

%---------------------------------------------------------------------
\subsection{天线温度}

TODO


%=====================================================================
\section{干涉测量原理}
\label{sec:interferometry}

根据衍射原理,望远镜的角分辨率为 $\theta \sim \lambda / D$,
其中 $\lambda$ 为辐射信号的波长(对应于观测频率),$D$ 为望远镜的直径。
相比光学波段,射电信号的波长要长得多,为了实现足够好的角分辨率,
必须建造巨型的望远镜,如 \SI{100}{\meter} 口径的 \ac{gbt}、
\SI{305}{\meter} 口径的 Arecibo、
\SI{500}{\meter} 口径的 \ac{fast}。
然而,在数百 \si{\MHz} 的低频射电波段,望远镜的口径需达到惊人的 \SI{10}{\km}
才能在 \SI{100}{\MHz} 实现约 \SI{1}{\arcminute} 的角分辨率,
这对于单口径望远镜而言显然是不现实的。
因此,低频射电观测通常使用干涉测量技术,通过联合一系列小望远镜开展相干测量并综合,
获得高分辨率图像。

%---------------------------------------------------------------------
\subsection{二元准单色干涉仪}

\begin{figure}
  \centering
  \includegraphics[width=0.7\textwidth]{interferometer}
  \bicaption[二元准单色干涉仪]{%
    二元准单色干涉仪的构成示意图。
    两个相同的天线相距 $\B{b}$ 放置并指向位于 $\hat{\B{s}}$ 方向的辐射源,
    接收的信号被放大后,再经过\acl*{ctor}的相乘($\times$)
    和时间平均($\langle\;\rangle$),得到输出响应 $R$。
  }{%
    The components of a two-element quasi-monochromatic interferometer
    observing in a very narrow radio frequency band centered at
    $\nu = \omega / (2\Cpi)$.
    The two identical antennas are separated by the baseline vector
    $\B{b}$ and point to the source in direction $\hat{\B{s}}$.
    The signals received by the antennas are amplified,
    multiplied ($\times$), and time averaged ($\langle\;\rangle$)
    by the correlator to yield the output response $R$.
    \\\textcopyright{}
    \citeay{condon2016}, \S\,3.7.1.
  }
  \label{fig:interferometer}
\end{figure}

考虑一个最简单的二元准单色干涉仪(如\autoref{fig:interferometer} 所示),
由两个相同的天线和相关器构成,只测量频率为 $\nu$ 的准单色信号。
由于辐射源非常遥远,其信号到达干涉仪时为平面波(忽略电离层扰动等影响)。
同一个波面被两个天线接收之间存在一个时间差,即\ac{tau-g}:
\begin{equation}
  \label{eq:tau-g}
  \acs{tau-g} = \frac{\B{b} \cdot \hat{\B{s}}}{\acs{c}},
\end{equation}
其中 \acs{c} 是\acl{c},
$\B{b}$ 为两天线之间的基线矢量(由天线 1 指向天线 2),
$\hat{\B{s}}$ 为指向辐射源的单位矢量。
两个天线接收信号后分别输出电压响应:
\begin{align}
  V_1(t) &= V \cos [\omega (t - \acs{tau-g})], \\
  V_2(t) &= V \cos (\omega t),
\end{align}
其中 $\omega = 2\Cpi\nu$ 为角频率。
然后,\ac{ctor}将两个天线的响应相乘:
\begin{align}
  V_1(t) V_2(t) &= V^2 \cos [\omega (t - \acs{tau-g})] \cos (\omega t) \\
    &= \frac{1}{2} V^2 \left[ \cos (2\omega t - \omega \acs{tau-g})
      + \cos (\omega \acs{tau-g}) \right].
\end{align}
接着,\acl{ctor}再对相乘后的响应进行时间平均 [参见\autoref{eq:ctor-avgtime}]:
\begin{equation}
  \label{eq:resp-corr}
  R = \langle V_1(t) V_2(t) \rangle
    = \frac{1}{2} V^2 \cos (\omega \acs{tau-g}).
\end{equation}
由于天线响应 $V_1$ 和 $V_2$ 正比于辐射源的电场强度以及天线\acl{gain},
因此\acl{ctor}响应 $R$ 正比于辐射源的流量密度 $S$
以及 $\sqrt{A_1 A_2}$,其中 $A_1$ 和 $A_2$ 为两个天线的有效接收面积。

由于地球的自转,辐射源的方向 $\hat{\B{s}}$ 发生变化,\acl{tau-g} \acs{tau-g}
也随之变化,于是\acl{ctor}的输出 $R$ 出现正弦形式的振荡,即\ac{fringe},其相位为:
\begin{equation}
  \phi = \omega\tau_g = 2\Cpi b \cos\theta / \lambda,
\end{equation}
其中 $\theta$ 为方向 $\hat{\B{s}}$ 和基线矢量 $\B{b}$ 之间的夹角。
于是有
\begin{equation}
  \diff{\phi}{\theta} = 2\Cpi \left( \frac{b \sin\theta}{\lambda} \right),
\end{equation}
可得单个\acl{fringe}的宽度,即干涉仪的\ac{sb-width}:
\begin{equation}
  \label{eq:sb-width}
  \acs{sb-width} = \frac{\lambda}{b \sin\theta},
\end{equation}
这也代表着干涉仪的角分辨能力。

天线的\ac{pp}描述了其响应随方向的变化情况,也被称为干涉仪的\ac{pb},
将对输出 $R$ 产生调制,如\autoref{fig:interferometer} 所示,
其中\acl{ctor}的输出\acl{fringe}的包络示意了\acl{pb}的衰减情况。

对于一个表面亮度分布为 $\acs{Ifreq}(\hat{\B{s}})$ 的\acl{extsrc},
由于不同位置产生的辐射互不相干,因此可以被当作一系列独立的点源处理,
于是上述二元准单色干涉仪的输出响应为:
\begin{equation}
  \label{eq:resp-cos}
  R_c = \int \acs{Ifreq}(\hat{\B{s}}) \cos \left(
      \omega \B{b}\cdot\hat{\B{s}} / c \right) \D{\Omega}
    = \int \acs{Ifreq}(\hat{\B{s}}) \cos \left(
      2\Cpi\, \B{b}\cdot\hat{\B{s}} / \lambda \right) \D{\Omega},
\end{equation}
其中下标 \enquote{$c$} 表示 \enquote{cosine} \acl{ctor}的输出,
以区分下文将要介绍的 \enquote{sine} \acl{ctor}。

一个任意的亮度分布 $I$ 可分解为奇对称成分 $I_O$ 与偶对称成分 $I_E$ 之和,
然而\autoref{eq:resp-cos} 描述的 cosine \acl{ctor}只能测量其中的
偶对称成分 $I_E$,即:
\begin{equation}
  R_c = \int \left[ I_O(\hat{\B{s}}) + I_E(\hat{\B{s}}) \right]
      \cos \left( 2\Cpi\, \B{b}\cdot\hat{\B{s}} / \lambda \right)
      \D{\Omega}
    = \int I_E(\hat{\B{s}}) \cos \left(
      2\Cpi\, \B{b}\cdot\hat{\B{s}} / \lambda \right) \D{\Omega}.
\end{equation}
为了能够测量另一个奇对称成分 $I_O$,则需要一个 \enquote{sine} \acl{ctor},
可通过对其中一个天线的输出增加 $\Cpi/2$ 的相位延迟来实现,于是有:
\begin{equation}
  R_s = \int \left[ I_O(\hat{\B{s}}) + I_E(\hat{\B{s}}) \right]
      \sin \left( 2\Cpi\, \B{b}\cdot\hat{\B{s}} / \lambda \right)
      \D{\Omega}
    = \int I_O(\hat{\B{s}}) \sin \left(
      2\Cpi\, \B{b}\cdot\hat{\B{s}} / \lambda \right) \D{\Omega}.
\end{equation}
\ac{cctor} 即为 cosine 和 sine \acl{ctor}的组合。
相应地,\ac{Vis},简称\emph{\acl{vis}},可定义为:
\begin{equation}
  \label{eq:vis}
  \acs{Vis} \equiv R_c - \Ci R_s
    = \int I_{\nu}(\hat{\B{s}}) \exp
      (-2\Cpi\Ci\, \B{b}\cdot\hat{\B{s}} / \lambda) \D{\Omega}.
\end{equation}

%---------------------------------------------------------------------
\subsection{有限带宽}

现在,将上述二元干涉仪推广至测量有限带宽的信号。
考虑一段中心频率为 $\nu_c$ 且宽度为 $\Delta\nu$ 的窄频带,
若辐射源的亮度以及天线的响应在此频带内基本不变,
则可知测量的\acl{vis} [\autoref{eq:vis}] 为:
\begin{align}
  \acs{Vis} &= \int \left[ \frac{1}{\Delta\nu}
        \int_{\nu_c-\Delta\nu/2}^{\nu_c+\Delta\nu/2}
        I(\hat{\B{s}}, \nu) \exp (-2\Cpi\Ci\, \nu\tau_g)
      \right] \D{\Omega} \\
    &\approx \int I_{\nu}(\hat{\B{s}}) \sinc (\Delta\nu\,\tau_g)
      \exp (-2\Cpi\Ci\, \nu_c\tau_g) \D{\Omega},
  \label{eq:vis-bw}
\end{align}
其中 $\sinc(\cdot)$ 为归一化 sinc 函数,定义如下:
\begin{equation}
  \label{eq:sinc}
  \sinc(x) =
    \begin{cases}
      \sin(\Cpi x) / (\Cpi x), & \quad \text{if } x \neq 0, \\
      1, & \quad \text{if } x = 0.
    \end{cases}
\end{equation}
因此,干涉仪测得的\acl{fringe}幅度减弱至原来的 $\sinc(\Delta\nu\,\tau_g)$ 倍。
为了最小化该损失,可以给前导天线的输出信号增加\ac{tau-c},
如\autoref{fig:interferometer-tau0} 所示,并随地球自转而调节 \acs{tau-c}
使其满足 $|\acs{tau-c} - \acs{tau-g}| \ll (\Delta\nu)^{-1}$。
在此情况下,到达\acl{ctor}时两个天线的信号的相位是同步的。
同时,满足 $\acs{tau-c} = \acs{tau-g}$ 的方向 $\hat{\B{s}}_0$
称为\ac{delay-center}或\ac{phase-refpos}。

\begin{figure}
  \centering
  \includegraphics[width=0.6\textwidth]{interferometer-tau0}
  \bicaption[\acl*{tau-c} \acs*{tau-c}]{%
    通过给前导天线(即天线 2)增加\acl*{tau-c}
    $\acs*{tau-c} \approx \acs*{tau-g}$,
    使得两天线的信号在相关运算时相位尽量同步,从而最小化带宽效应对测量条纹的衰减影响。
  }{%
    By introducing the compensating delay
    $\acs*{tau-c} \approx \acs*{tau-g}$ in the signal path of the leading
    antenna (i.e., antenna 2), the phases of the two signals are almost
    in sync when they reach the correlator, hence minimizing the
    attenuation to the measured fringes caused by the finite bandwidth
    effect.
    \\\textcopyright{}
    \citeay{condon2016}, \S\,3.7.3.
  }
  \label{fig:interferometer-tau0}
\end{figure}

\acl{tau-g} \acs{tau-g} 会随方向而变化,因此\acl{tau-c} \acs{tau-c}
只对特定方向 $\hat{\B{s}}_0$ (即\acl{delay-center})是正好准确的。
偏离\acl{delay-center}的角度 $\Delta\theta$ 越大,\acl{tau-c} \acs{tau-c}
的修正效果越差,带宽带来的损失越大,即\ac{bw-smear}。
该涂污效应限制了有效的视场大小,为满足:
\begin{equation}
  \Delta\nu \Delta\tau_g
    \approx \Delta\nu \diff{\tau_g}{\theta} \Delta\theta
    = \frac{b \sin\theta}{c} \Delta\nu \Delta\theta
    \ll 1,
\end{equation}
则要求:
\begin{equation}
  \Delta\theta \ll \frac{\nu \acs{sb-width}}{\Delta\nu}.
\end{equation}
另一个解决\acl{bw-smear}的方法是将宽频带划分为一系列足够窄的频率通道
(如每个通道仅宽几十 \si{\kHz}),每个通道的信号都被独立地相关运算得到\acl{vis}。

类似地,如果\acl{ctor}的积分时间 $\Delta t$ 过长,则辐射源的位置 $\hat{\B{s}}$
会因地球自转而发生显著改变(可与 \acs{sb-width} 相比拟),
导致\acl{tau-c} \acs{tau-c} 的修正效果变差,即\ac{t-smear}。
一个距离\acl{delay-center} $\Delta\theta$ 的目标的移动速度为
$v = \omega_e \Delta\theta$,其中 $\omega_e$ 为地球自转的角速度。
为了最小化\acl{t-smear}的影响,\acl{ctor}的积分时间 $\Delta t$ 需满足:
\begin{equation}
  v \Delta t = \omega_e \Delta\theta \Delta t \ll \theta_s,
\end{equation}
即:
\begin{equation}
  \label{eq:ctor-avgtime}
  \Delta t \ll \frac{\theta_s}{\omega_e \Delta\theta}.
\end{equation}

%---------------------------------------------------------------------
\subsection{成像原理}

\begin{figure}
  \centering
  \includegraphics[width=0.5\textwidth]{interferometer-coordsys}
  \bicaption[$(u,v,w)$ 坐标系]{%
    干涉仪的 $(u,v,w)$ 直角坐标系,其中 $w$ 轴指向参考方向 $\hat{\B{s}}_0$,
    通常为目标的中心,$u$ 轴向东,$v$ 轴向北。
    基线矢量可表示为 $\B{b} = (u,v,w) \lambda$,
    目标的亮度分布则用\acl*{dc}描述,即 $I(\hat{\B{s}}) = I(l,m)$,
    其中 $l, m$ 为方向矢量 $\hat{\B{s}}$ 分别对 $u, v$ 轴的投影长度。
  }{%
    The $(u,v,w)$ Cartesian coordinate system for interferometers,
    in which the $w$-axis points in the reference direction
    $\hat{\B{s}}_0$ (usually toward the source center), and
    the $u$- and $v$-axes point east and north, respectively.
    A baseline vector is represented as $\B{b} = (u,v,w) \lambda$,
    and the source brightness distribution is described as
    $I(\hat{\B{s}}) = I(l,m)$, where $l, m$ are direction cosines,
    i.e., the projected lengths of the direction vector $\hat{\B{s}}$
    against the $u$- and $v$-axes, respectively.
    \\\textcopyright{}
    \citeay{thompson2017}, \S\,3.1.
  }
  \label{fig:interferometer-coordsys}
\end{figure}

为了能够实际运用\autoref{eq:vis} 或\autoref{eq:vis-bw} 获得图像,
需要定义一个坐标系,如\autoref{fig:interferometer-coordsys}
所示的 $(u,v,w)$ 直角坐标系是最常用的,其中 $w$ 轴指向参考方向 $\hat{\B{s}}_0$,
通常为目标的中心,$u$ 轴向东,$v$ 轴向北。
于是,基线矢量 $\B{b} = (u,v,w) \lambda$,
方向矢量 $\hat{\B{s}} = (l,m,\sqrt{1-l^2-m^2})$,
其中 $l, m$ 为 $\hat{\B{s}}$ 分别对 $u, v$ 轴的投影长度,即\ac{dc}。
再利用 $\D{\Omega} = \D{l}\D{m} \,/ \sqrt{1-l^2-m^2}$,
\autoref{eq:vis} 可表示为:
\begin{equation}
  \label{eq:vis-uvw}
  \acs{Vis}(u,v,w) = \iint \frac{I_{\nu}(l,m)}{\sqrt{1-l^2-m^2}}
    \exp \left[ -2\Cpi\Ci \left( ul+vm+w\sqrt{1-l^2-m^2} \right) \right]
    \D{l}\D{m}.
\end{equation}
需注意,这\emph{不是}二维 Fourier 变换。

然而,在下述两种常见的特殊情况下,上式可变成二维 Fourier 变换,
从而通过逆运算获得目标的亮度分布。
\begin{itemize}
\item
\emph{所有基线矢量共面:}
这可进一步分为两种情形:
(1) 干涉阵列的天线只沿东西方向分布,如 \ac{wsrt},尽管地球自转,
所有基线矢量均落在同一垂直于地球自转轴的平面内;
(2) 虽然干涉阵列的天线分布在一个二维平面,如 \autoref{ssec:miteor}
将介绍的 \ac*{miteor},但只考虑瞬时观测。
此时,可以选取合适的坐标系使得 $w = 0$,于是\autoref{eq:vis-uvw}
变成二维 Fourier 变换,利用其逆变换可得目标的亮度分布为:
\begin{equation}
  \label{eq:vis-inv1}
  \frac{I_{\nu}(l,m)}{\sqrt{1-l^2-m^2}} = \iint \acs{Vis}(u,v, w \equiv 0)
    \exp [2\Cpi\Ci\, (ul+vm)] \D{l}\D{m}.
\end{equation}

%.......................................
\item
\emph{小视场成像:}
对于任何干涉阵列,如果只考虑以参考方向 $\hat{\B{s}}_0$ 为中心的足够小的区域,
于是有:
\begin{equation}
  w\sqrt{1-l^2-m^2} \approx w \left[ 1 - \frac{1}{2} (l^2+m^2) \right].
\end{equation}
则\autoref{eq:vis-uvw} 成为:
\begin{equation}
  \acs{Vis}(u,v,w) \approx \exp (-2\Cpi\Ci\,w) \iint
    \frac{I_{\nu}(l,m)}{\sqrt{1-l^2-m^2}}
    \exp [-2\Cpi\Ci\, (ul+vm) -\Ci\Cpi w(l^2+m^2)] \D{l}\D{m}.
\end{equation}
如果 $|\Cpi w(l^2+m^2)| \ll 1$,即 $\exp [-\Ci\Cpi w(l^2+m^2)] \sim 1$,
则上式成为二维 Fourier 变换,即:
\begin{equation}
  \acs{Vis}(u,v) \equiv \acs{Vis}(u,v,w) \exp (2\Cpi\Ci\,w)
    = \iint \frac{I_{\nu}(l,m)}{\sqrt{1-l^2-m^2}}
    \exp [-2\Cpi\Ci\, (ul+vm)] \D{l}\D{m},
\end{equation}
对其进行逆变换即可得到目标的亮度分布 $I_{\nu}(l,m)$:
\begin{equation}
  \label{eq:vis-inv2}
  \frac{I_{\nu}(l,m)}{\sqrt{1-l^2-m^2}} = \iint \acs{Vis}(u,v)
    \exp [2\Cpi\Ci\, (ul+vm)] \D{l}\D{m}.
\end{equation}

\end{itemize}

根据\autoref{eq:vis} 中基线矢量 $\B{b}$ 和方向矢量 $\hat{\B{s}}$ 之间的对称性,
上述第一种情况要求 $\B{b}$ 全部在同一平面内,
第二种情况要求 $\hat{\B{s}}$ 在天球上的对应点全部在同一平面内,即视场足够小,
因此,这两种情况可理解为相同近似条件的不同表现形式 \cite{clark1999}。
如果无法满足以上两种情况,比如大视场成像,则需要采用专门的成像方法
\cite{cornwell1992,sault2007},
比如 \ac{w-proj}法 \cite{cornwell2008}、\ac{w-stack}法 \cite{humphreys2011}
(\autoref{ssec:imaging} 将使用的 WSClean 成像软件实现了该方法
\cite{offringa2014,offringa2017})。

%---------------------------------------------------------------------
\subsection{\texorpdfstring{$uv$}{uv} 覆盖}

由于天空的亮度分布 $I_{\nu}(l,m)$ 为实数,因此测量的\acl{vis}满足
$\acs{Vis}(-u,-v) = \acs{Vis}^*(u,v)$,
于是基线矢量为 $\B{b} = (u,v,w)\lambda$ 的二元干涉仪在每个时刻测量
$uv$ 平面上两个相互对称的点的\acl{vis}。
随着地球自转,$\B{b}$ 的各分量逐渐变化,经过 \SI{24}{\hour},
该基线所测量的\acl{vis}数据对应 $uv$ 平面上两个相互对称的椭圆。
一个干涉阵列通常由大量天线组成,因此 $N_A$ 个天线之间可形成 $N_A(N_A-1)/2$ 条基线,
每条基线都将测量 $uv$ 平面上相应位置的\acl{vis},
可以显著增加 $uv$ 覆盖度,即 $uv$ 平面上被测量的点。
\autoref{fig:uv-coverages} 展示了不同天线数目、不同观测时长的 $uv$ 覆盖样例。

\begin{figure}
  \centering
  \includegraphics[width=0.8\textwidth]{uv-coverages}
  \bicaption[$uv$ 覆盖样例]{%
    $uv$ 覆盖样例。
    \emph{从上往下:} 干涉阵列分别包括 2、5、10 和 50 个呈对数螺旋状分布的天线;
    \emph{从左到右:} 观测时间分别为 \SI{10}{\second}、\SI{2}{\hour}、
    \SI{4}{\hour} 和 \SI{6}{\hour}。
  }{%
    Examples of $uv$ coverages.
    \emph{Top to bottom:} the interferometer includes 2, 5, 10, and 50
    antennas in a logarithmic spiral pattern, respectively;
    \emph{Left to right:} the observing time is \SI{10}{\second},
    \SI{2}{\hour}, \SI{4}{\hour}, and \SI{6}{\hour}, respectively.
    \\\textcopyright{}
    \citeay{avison2013}.
  }
  \label{fig:uv-coverages}
\end{figure}

干涉阵列的天线数目总是有限的,在实际观测中 $uv$ 平面不可能被完全覆盖,
具体覆盖情况可由\ac{sf}描述:
\begin{equation}
  \label{eq:sf}
  \acs{S-uv} = \sum_{k,t} \delta(u - u_{k,t}, v - v_{k,t}),
\end{equation}
其中 $u_{k,t}, v_{k,t}$ 为基线 $\B{b}_k$ 在 $t$ 时刻在 $uv$ 平面内的分量。
于是,干涉阵列实际测量的\acl{vis}数据为 $\acs{Vis}(u,v) \acs{S-uv}$,
由于无法获得目标亮度分布的全部信息,
根据\autoref{eq:vis-inv2} 对此进行逆 Fourier 变换仅能得到目标的\ac{dirty-map}:
\begin{equation}
  \label{eq:dirty-map}
  \frac{I_{\nu}^D(l,m)}{\sqrt{1-l^2-m^2}} = \iint
    \acs{Vis}(u,v) \acs{S-uv} \exp [2\Cpi\Ci\, (ul+vm)] \D{l}\D{m}.
\end{equation}
利用\ac{conv-theorem},上式可表示为:
\begin{equation}
  I_{\nu}^D(l,m) = I_{\nu}(l,m) * B(l,m),
\end{equation}
其中
\begin{equation}
  \label{eq:syn-beam}
  B(l,m) = \iint \acs{S-uv} \exp [2\Cpi\Ci\, (ul+vm)] \D{l}\D{m}
\end{equation}
是\acl{sf} \acs{S-uv} 的 Fourier 变换,称为\ac{sb}或\ac{psf}。
\autoref{fig:imaging-relations} 展示了成像过程中各种变换关系。
为了从\acl{dirty-map} $I_{\nu}^D$ 尽可能地恢复目标的真实图像,
则需要使用复杂的非线性\ac{deconv}方法,
比如 CLEAN 算法 \cite{hogbom1974,cornwell1999}、
\ac{mem} \cite{narayan1986}。

\begin{figure}
  \centering
  \includegraphics[width=\textwidth]{imaging-relations}
  \bicaption[成像过程中的变换关系]{%
    成像过程中的各种变换关系。
    \emph{(a)} 天空的真实图像;
    \emph{(b)} 干涉阵列的\acl*{sb},对应 (e) 的 Fourier 变换;
    \emph{(c)} 脏图,对应 (f) 的 Fourier 变换;
    \emph{(d)} \acl*{vis}的真实数据,对应 (a) 的 Fourier 变换;
    \emph{(e)} 干涉阵列的\acl*{sf};
    \emph{(f)} 实际测量到的\acl*{vis}数据,为 (d) 和 (e) 乘积。
  }{%
    The transform relations among the imaging process.
    \emph{(a)} The true sky map;
    \emph{(b)} The synthesized beam of the interferometer, which is the
    Fourier Transform of (e);
    \emph{(c)} The dirty map, which is the Fourier Transform of (f);
    \emph{(d)} The true visibility data, which are the Fourier Transform
    of (a);
    \emph{(e)} The sampling function of the interferometer;
    \emph{(f)} The actually measured visibility data, which are the
    product of (d) and (e).
    \\\textcopyright{}
    Dale E. Gary, Radio Astronomy, Lecture 6,
    \url{https://web.njit.edu/~gary/728/Lecture6.html}, (2018-11-21).
    [反转了颜色]
  }
  \label{fig:imaging-relations}
\end{figure}

%---------------------------------------------------------------------
\subsection{灵敏度}

point-source sensitivity, brightness sensitivity

ERA: \S 3.6.3.2:confusion; \S 3.7.6


%=====================================================================
\section{主要低频干涉阵列}
\label{sec:instruments}

大型低频干涉阵列是目前测量 EoR 信号的主要设备。
近十几年以来,国内外已建成一批各具特色的低频干涉阵列,
还有若干新型干涉阵列正在兴建或准备建设。
以下对其中主要的干涉阵列作简要介绍。

%---------------------------------------------------------------------
\subsection{21CMA}

\acf{21cma} 是我国开展“宇宙第一缕曙光”探测的低频射电干涉阵列,
位于中国西部天山深处的乌拉斯台,环绕在四周的高山能提供宁静的射电环境。
\acs{21cma} 的 81 个站点呈 T 形分布在东西约 \SI{6}{\km}、
南北约 \SI{4}{\km} 的两条直线上
(\autoref{fig:21cma} 展示了沿东西方向的部分站点)。
每个站点包含 127 根对数周期天线,工作频率为 \SIrange{50}{200}{\MHz},
频率分辨率为 \SI{24.4}{\kHz},
角分辨率达 \SI{1}{\arcminute}(在 \SI{200}{\MHz} 处),
采用模拟波束合成固定观测以北天极为中心、半径约 \SI{5}{\degree} 的天区
\cite{wang2013,zheng2016}。
\acs{21cma} 已于 2006 年建设完成,并于 2009 年升级了新型低噪声放大器和
基于 \acs{gpu} 的数据采集系统,目前已积累多年的观测数据。
\acs{21cma} 作为中国主要的 \acs{ska} 探路者项目,
项目成员开发了完整的数据处理流程及软件、
提出了射频干涉探测及抑制新方法 \cite{huang2016}、
探测并编录了北天极视场内的 624 个射电源 \cite{zheng2016}。
目前,\acs{21cma} 正在改造升级数字多波束合成系统,
以实现多目标跟踪观测,掌握低频脉冲星的搜寻技术。

\begin{figure}
  \centering
  \includegraphics[width=0.8\textwidth]{21CMA}
  \bicaption[21CMA 东西方向的部分站点]{%
    \acs{21cma} 东西方向的部分站点,每个站点包含 127 根对数周期天线。
  }{%
    Part of the \acs{21cma} stations along the east-west direction,
    with each station including 127 log-periodic antennas.
    \\\textcopyright{}
    \acs{21cma}, \acl{nao}.
  }
  \label{fig:21cma}
\end{figure}

%---------------------------------------------------------------------
\subsection{LOFAR}

\acf{lofar} 是由\ac{astron}建造的新型低频干涉阵列 \cite{vanHaarlem2013},
由工作在 \SIrange{10}{90}{\MHz} 波段的低频段天线(LBA)和
工作在 \SIrange{110}{250}{\MHz} 波段的高频段天线(HBA)两部分组成。
\acs{lofar} 共有 51 个站点,
其中 24 个站点分布在半径 \SI{2}{\km} 的核心区域
(\autoref{fig:lofar} 显示了最中心的部分),
14 个站点呈螺旋状分布在外围区域,
还有 13 个国际站点分布在德国、法国、瑞士、英国、波兰和爱尔兰,
基线长达 \SI{1500}{\km}。
荷兰境内的 38 个站点各包含 96 个 LBA 和 48 个 HBA,
13 个国际站点每个包含 96 个 LBA 和 96 个 HBA。
\acs{lofar} 采用了数字多波束合成技术,能实现多目标跟踪观测,
并且显著提高巡天效率,为 \acs{ska1low} 提供强有力的技术支持
\cite{deVos2009,vanHaarlem2013,pizzo2018}。
\acs{lofar} 于 2012 年建设完成并开始观测,
已经完成北天 \SIrange{120}{168}{\MHz} 的深度巡天
\ac{lotss} \cite{shimwell2017,shimwell2019}。
目前,\acs{lofar} 正在提议 2.0 升级计划。

\begin{figure}
  \centering
  \includegraphics[width=0.8\textwidth]{LOFAR-superterp}
  \bicaption[LOFAR 核心区域的中心]{%
    \acs{lofar} 核心区域的中心。
    小块深色区域为 LBA,大块深色区域为 HBA。
  }{%
    The heart of the \acs{lofar} core.
    The small dark regions are installed with LBA,
    while the big dark regions are installed with HBA.
    \\\textcopyright{}
    \citeay{vanHaarlem2013}.
  }
  \label{fig:lofar}
\end{figure}

%---------------------------------------------------------------------
\subsection{MWA}

\acf{mwa} 位于澳大利亚西部的 Murchison 射电天文台,是 \acs{ska1low} 的先驱
\cite{lonsdale2009,bowman2013,tingay2013,wayth2018}。
该阵列的主要科学目标包括宇宙再电离信号探测、河内及河外射电源、暂现源和空间气候。
\acs{mwa} 工作在 \SIrange{80}{300}{\MHz} 频段,使用一种双极化偶极子天线,
每个站点包含 16 个天线(按 4 行 4 列规则排列)。
所有天线均固定指向天顶,工作时通过调控各天线的时延来控制波束的合成与指向。
\acs{mwa} 自 2007 年开始建设,于 2012 年完成了一期 128 个站点的建设,
于 2017 年底完成了二期 128 个新站点的扩建工作\cite{wayth2018},目前已投入使用。
\autoref{fig:mwa} 显示了 \acs{mwa} 东侧六边形区域内的站点。
\acs{mwa} 的站点分为两部分:
一部分紧凑地排列在六边形区域内,主要用于探测再电离信号以及研究银河系大尺度结构;
另一部分散布于四周较大区域,实现较高的空间分辨率,便于开展河外射电源等研究。
\acs{mwa} 一期已完成了 \ac{gleam} 巡天项目 \cite{wayth2015},
并已发布一批成果,比如点源目录 \cite{hurleyWalker2017}、
银河系内 \Hii/ 区目录 \cite{su2018}、
高分辨率 EoR 前景模型 \cite{procopio2017}、等等。
使用 \acs{mwa} 二期开展的 \ac{gleam-x} 巡天也正在积极进行 \cite{hurleyWalker2017prop}。
作为少有的覆盖南天的低频射电巡天,\acs{gleam} 和 \acs{gleam-x}
将为 \acs{ska1low} 的巡天工作提供校准指导和星表的交叉认证。
同时 \acs{mwa} 也将会为 \acs{ska1low} 的宇宙再电离探测任务提供
更精准的天空模型和天区指导。

\begin{figure}
  \centering
  \includegraphics[width=\textwidth]{MWA}
  \bicaption[MWA 东部站点]{%
    \acs{mwa} 东侧六边形区域内的站点,每个站点包含 16 个天线。
  }{%
    The stations inside the \acs{mwa}'s east hexagonal region,
    with each station consisting of 16 antennas.
    \\\textcopyright{}
    \acuse{icrar}\ac{icrar}/\acs{mwa},
    \url{https://www.icrar.org/multimedia/images/}, (2018-10-04).
  }
  \label{fig:mwa}
\end{figure}

%---------------------------------------------------------------------
\subsection{LWA}

\acf{lwa} 是一个正在建设于美国新墨西哥州中部的大型低频干涉阵列,
计划由 53 个分布远达 \SI{400}{\km} 的站点组成,
每个站点的大小约 \SI{100x100}{\meter} 并且包含 256 个双极化天线,
总接收面积达 \SI{1}{\km\squared}(在 \SI{10}{\MHz} 处),
工作在非常低频的 \SIrange{10}{88}{\MHz} 波段,
这是我们目前了解最少的射电波段 \cite{ellingson2009}。
借助其高灵敏度和高角分辨率,\acs{lwa} 将打开这一个新射电窗口,
研究宇宙高能粒子加速机制、早期宇宙及其演化、暂现源、银河系星际介质、
太阳活动及电离层性质等。
\acs{lwa} 所采用的大站点设计使其更适合研究银河系的大尺度结构。
\acs{lwa} 的首个站点(LWA1;\autoref{fig:lwa})位于\ac{vla} 附近,
已于 2009 年建设完成,并于 2011 年开始正式观测 \cite{taylor2012,ellingson2013};
其他站点正在积极建设之中。
\acs{lwa} 亦采用数字波束合成技术,但其创新之处在于每个站点均可独立使用并成像。
目前已使用 \acs{lwa}1 开展巡天并获得了 \SIrange{35}{80}{\MHz}
北天图像 \cite{dowell2017}。

\begin{figure}
  \centering
  \includegraphics[width=0.8\textwidth]{LWA1}
  \bicaption[LWA 的首个站点(LWA1)]{%
    位于 \acs{vla} 附近的 \acs{lwa} 的首个站点(LWA1),包含 256 个天线。
  }{%
    The first station of \acs{lwa}, i.e., LWA1,
    which locates near the \acs{vla} and contains 256 antennas.
    \\\textcopyright{}
    \citeay{taylor2012}.
  }
  \label{fig:lwa}
\end{figure}

%---------------------------------------------------------------------
\subsection{MITEoR}
\label{ssec:miteor}

\acf{miteor} 是一个使用\ac{fftt} 新技术\cite{tegmark2009,tegmark2010}的先导阵列,
由 64 个按 8 行 8 列规则分布的全同双极化天线构成(\autoref{fig:miteor}),
在 \SIrange{100}{200}{\MHz} 范围内覆盖两个宽度为 \SI{25}{\MHz} 的频段
\cite{zheng2014}。
利用这种天线布局方式,可以直接对天线采集信号运用\ac{fft}进行成像,
避免了传统干涉阵列耗时的天线/站点间两两相关运算,
将计算复杂度由 $O(N_{\!A}^2)$ 显著降为 $O(\acs{N-ant} \log\acs{N-ant})$,
其中 \acs{N-ant} 为\acl{N-ant};
同时还将数据存储压力从 $O(N_{\!A}^2)$ 大幅减轻至 $O(\acs{N-ant})$。
如此可以极大地降低建设和运行成本,非常有利于建设超大规模的干涉阵列,
实现极高的灵敏度。
此外,阵列中的大量冗余基线能为系统自校准提供有效帮助 \cite{dillon2016}。
目前,\acs{miteor} 已开展观测并公布了 \SIrange{128}{175}{\MHz} 的北天图像
\cite{zheng2017},充分验证了 \acs{fftt} 技术的可行性。
该技术的创新性和巨大潜力能在未来 EoR 实验中发挥重要作用。

\begin{figure}
  \centering
  \includegraphics[width=\textwidth]{MITEoR}
  \bicaption[MITEoR 干涉阵列]{%
    在 2013 年夏天部署完成的 \acs{miteor} 干涉阵列,
    64 个双极化天线规则地分布在 \SI{21x21}{\meter} 的矩形区域,
    相互之间分隔 \SI{3}{\meter}。
  }{%
    The \acs{miteor} array deployed in the summer of 2013.
    The 64 dual-polarization antennas were laid on a \SI{21x21}{\meter}
    regular grid with a separation of \SI{3}{\meter}.
    \\\textcopyright{}
    \citeay{zheng2014}.
  }
  \label{fig:miteor}
\end{figure}

%---------------------------------------------------------------------
\subsection{HERA}

\acf{hera} 是正在南非 Karoo 射电天文保护区建造的、
继 \ac{paper} 等探路者阵列之后的第二代宇宙再电离时期探测阵列 \cite{deboer2017}。
其首要科学目标是精确测量源自宇宙再电离时期甚至\acl{cd}时期的 \hisignal/
及其功率谱的演化,从而描绘宇宙再电离时期以及之前的宇宙大尺度结构。
该阵列将由 350 面直径为 \SI{14}{\meter} 的固定式抛物面碟形天线构成,
观测频率为 \SIrange{50}{250}{\MHz}。
\acs{hera} 的阵型和天线设计保证了它能够为 EoR 信号的观测提供高灵敏度、
波束形状和其他仪器效应相对简单可控、
易于借助大量冗余基线对系统进行高精度校准 \cite{dillon2016}。
该设计还保证 \acs{hera} 能够有效利用\ac{ds} 技术\cite{parsons2012}
和\acl{fgavd}方法(详见 \autoref{sec:fgavd})来处理观测数据。
目前 \acs{hera} 的第一期 37 面天线已经安装完毕并开始试观测
(\autoref{fig:hera} 显示了 \acs{hera} 已于 2016 年建成的 19 面天线),
第二期的 128 面天线也已开始建设,是 \acs{ska1low} 的有力竞争者。

\begin{figure}
  \centering
  \includegraphics[width=\textwidth]{HERA19}
  \bicaption[HERA 已建成的 19 面天线]{%
    \acs{hera} 在 2016 年建成的 19 面碟形天线。
    后方的小型天线属于 \acs{paper} 项目。
  }{%
    The \acs{hera}'s 19 dish antennas deployed in South Africa in 2016.
    The small antennas in the background belong to the \acs{paper}
    experiment.
    \\\textcopyright{}
    \acs{hera}/SKA Africa, \url{http://reionization.org/}, (2018-10-04).
  }
  \label{fig:hera}
\end{figure}

%---------------------------------------------------------------------
\subsection{SKA}

\acf{ska} 是由澳大利亚、加拿大、中国、印度、意大利、新西兰、南非、瑞典、荷兰
以及英国共同参与建设的下一代巨型射电望远镜阵列,
由位于澳大利亚西部 Muchison 的低频阵列 (SKA-Low)
和位于南非 Karoo 的中频阵列 (SKA-Mid) 组成(\autoref{fig:ska}),
计划最终包含上百万个低频天线和上千面中频碟形天线,
达到约 \SI{1}{\km\squared} 的接收面积,
实现极高的灵敏度、分辨率和巡天速度。
SKA 的建设分别两期,第一期 (SKA1) 将建设约 \SI{10}{\percent} 的天线。
SKA1-Low 将由分布在 512 个站点的约 13 万根天线组成,最长基线约 \SI{65}{\km},
观测频率为 \SIrange{50}{350}{\MHz},
SKA1-Mid 将由 197 面(包含 MeerKAT 的 64 面)碟形天线组成,
最长基线约 \SI{150}{\km},观测频率为 \SI{350}{\MHz} 至 \SI{15.3}{\GHz}。
SKA 的关键科学目标有\cite{braun2015}:
宇宙再电离时期和黎明时期(功率谱测量甚至直接成像观测)\cite{mellema2013,mellema2015,koopmans2015}、
宇宙学和暗能量\cite{maartens2015,santos2015}、
脉冲星和黑洞\cite{kramer2015}、
星系形成与演化(连续谱巡天和中性氢巡天)\cite{prandoni2015,staveley2015}、
暂现源\cite{fender2015}、
宇宙磁场的起源与演化\cite{johnston2015}、
以及地外生命\cite{hoare2015}。
目前 SKA1 已在积极建设之中,预计 2025 年左右能够使用部分天线开展科学观测。

\begin{figure}
  \centering
  \includegraphics[width=0.5\textwidth]{SKA1-low-closeup}%
  \includegraphics[width=0.5\textwidth]{SKA1-mid-closeup}
  \bicaption[SKA 低频和中频天线示意图]{%
    SKA-Low 偶极天线(左栏)和 SKA-Mid 碟形天线(右栏)的想像图。
  }{%
    The artist rendition of the SKA-Low dipole antennas (left)
    and the SKA-Mid dishes (right).
    \\\textcopyright{}
    SKA Organisation, \url{https://www.skatelescope.org/multimedia/image/}, (2019-03-17).
  }
  \label{fig:ska}
\end{figure}


%=====================================================================
\section{小结}

TODO


%% EOF
