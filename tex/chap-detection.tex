%%
%% Copyright (c) 2018 Weitian LI <liweitianux@sjtu.edu.cn>
%% Creative Commons BY 4.0
%%

\chapter{再电离时期的探测}
\label{chap:detection}


\acf{eor}是早期宇宙的一段缺乏了解的时期,目前的理论研究以及有限的观测证据表明
该时期从宇宙大爆炸之后约 \SI{300}{\Myr} 持续到约 \SI{1}{\Gyr},对应红移范围
约为 \numrange{6}{15} (参见 \citeay{koopmans2015} 及其所引文献)。
充分探明并理解该时期是为进一步揭示更早期的\acl{cd}和\acl{da}($z > 15$)、
建立完整的宇宙演化图景的关键环节 (ref???)。
在低频射电波段(约 \SIrange{50}{200}{\MHz})
探测源自\acl{eor}的\acl{hi} \hisignal/是目前研究该时期
的最直接而有效的办法 (ref???)。


%=====================================================================
\section{中性氢~21\texorpdfstring{\,}{ }cm~信号}
\label{sec:21cm-signal}

physical theory ...
EoR observation principle ...
delta-Tb (spin temperature) ...

Furlanetto 2016!

21 cm 信号平均强度随红移和频率(亦即宇宙年龄)的变化规律。
在最初的宇宙黑暗时期,重子物质与 CMB 光子脱耦并随宇宙膨胀而冷却,结构也开始形成,
其中的冷气体可通过 21 cm 吸收信号被观测;
然后第一代恒星及星系产生并辐射大量 Ly-alpha 光子,
使得中性氢的自旋态与气体温度紧密耦合,导致强烈的 21 cm 吸收信号;
但随着星系的大量形成而加热其中的气体,中性氢的 21 cm 辐射信号逐渐变强;
最后中性氢被逐步电离完,21 cm 辐射信号也衰减至消失
\cite{pritchard2012}。


%=====================================================================
\section{探测方法和主要困难}
\label{sec:det-methods}

目前有三种测量 EoR 信号的方法,由易到难分别为:
(1)测量全天总功率;(2)测量功率谱;(3)直接获取再电离区域的图像。
第一种方法仅测量 EoR 信号的全天总功率随红移(即观测频率)的变化,
所得结果可以用于推断物质的电离过程,帮助检验和约束再电离模型
\cite{pritchard2012,liu2016}。
该方法相对简单易行,通常采用小型专用设备,一般包含单个或少量天线。
目前已有一批采用该方法的 EoR 探测实验,主要包括
位于澳大利亚的 \ac{edges} \cite{bowman2008} 和
\ac{bighorns} \cite{sokolowski2015}、
位于美国的 \ac{leda} \cite{greenhill2012}、
位于墨西哥的 \ac{sci-hi} \cite{voytek2014}
以及位于印度的 \ac{saras} \cite{singh2018}。
值得一提的是,\acs{edges} 在 2018 年初报导称发现全天平均射电信号在 \SI{78}{\MHz}
附近存在吸收,该吸收信号所处位置大致符合早期恒星所引发的 \hisignal/,
但其强度是目前理论预测值的两倍以上 \cite{bowman2018}。

后两种方法则进一步测量 EoR 信号的统计分布规律甚至三维图像,能够提供更加全面丰富
的信息用于系统性地研究\acl{eor}。
尽管这两种测量方法更加强大有效,但需要大型低频干涉阵列,
如 \autoref{sec:instruments} 所介绍的主要干涉阵列,
其中仅有 \acs{ska} 将拥有足够高的灵敏度实现对再电离区域的直接成像观测。

然而,EoR 探测实验,尤其是采用干涉阵列,面临着一系列困难。
这些困难可主要分为以下几类:
\begin{itemize}
\item
\emph{前景干扰:}
源自银河系以及河外源的前景辐射非常强烈,可达数百 \si{\kelvin},
是 EoR 信号(仅约几 \si{\mK} 至几十 \si{\mK})的 \numrange{4}{5} 个数量级。
虽然对干涉阵列而言重要的是辐射的空间涨落幅度而非其平均强度,
但是前景辐射的涨落幅度仍达数 \si{\kelvin} 到数十 \si{\kelvin},
远远压制了待测 EoR 信号 \cite{zaroubi2013}。
\autoref{fig:eor-foregrounds} 显示了主要的前景成分及其在
\SI{120}{\MHz} 处的强度。
因此,即便是轻微的前景处理不当,都会导致微弱的 EoR 信号被淹没而无法被捕捉到。
此外,部分前景成分(如银河系\acl{synrad})存在显著偏振,
该偏振成分可能发生泄漏而影响前景强度的测量,即\ac{pl}效应 (ref???),
导致前景的频谱结构复杂化而变得更加难以处理 (ref???)。
如何处理强烈的前景干扰并成功分离 EoR 信号,
是目前 EoR 探测领域的一个关键任务,
不仅需要系统深入地理解前景的特征 \cite{offringa2016,carroll2016,procopio2017} (ref???),
还需要研发有效的前景扣除与信号分离算法 \cite{chapman2015,chapman2016} (ref???)。

\begin{figure}[tbp]
  \centering
  \includegraphics[width=0.7\textwidth]{eor-foregrounds}
  \bicaption[主要前景成分及其强度示意图]{%
    主要前景成分以及强度示意图。
    图中的数值代表在 \SI{120}{\MHz} 处的\acuse{rms}\acl{rms}值。
  }{%
    A diagram showing the major foreground components contaminating
    the EoR signal.
    The numbers in the figure represent the \acs{rms} values
    at \SI{120}{\MHz}.
    \\\textcopyright{}
    \citeay{zaroubi2013}.
  }
  \label{fig:eor-foregrounds}
\end{figure}

%.......................................
\item
\emph{人工源的\acl{rfi}:}
随着科技的进步和社会的发展,人类活动产生的无线电波已在地球上无处不在。
这些人工源主要有:\ac{am}和\ac{fm}广播、卫星通信、\ac{gps}信号、
对讲机、手机、移动通信基站、航空通信、雷达、等等。
虽然 EoR 探测设备通常建设在人烟稀少的射电宁静区域,但是仍不可避免受到
人工源的\acl{rfi},甚至由月亮以及太空碎片反射回来的无线电波都可能
对 EoR 观测产生一定程度的影响 \cite{mckinley2013,tingay2013rfi}。
如\autoref{fig:rfi-mwa} 所示的是 MWA 在其各子频段的\acl{vis}数据
被标记为\acl{rfi}的比例,其中突显了\ac{fm}广播、卫星通信以及数字电视
等干扰源对 EoR 探测所造成的影响。
\acl{rfi}的强度通常会高出天空信号的若干个数量级 \cite{bentum2011} (ref???),
而且会随时发生变化。
目前的主要办法是识别并屏蔽存在明显\acl{rfi}的时间和频率片段
\cite{offringa2010,offringa2012,prasad2012},
但是残留的干扰可能会对前景处理以及 EoR 信号测量均产生严重影响 \cite{offringa2015}。

\begin{figure}[tbp]
  \centering
  \includegraphics[width=0.7\textwidth]{RFI-MWA}
  \bicaption[MWA 各子频段内的\acl*{rfi}比例]{%
    MWA 各子频段的\acl{vis}数据被标记为\acl{rfi}的比例。
  }{%
    The \acs{rfi} occupancy, calculated as the percentage of
    visibilities that are detected as \acs{rfi} by the flagger,
    per sub-band for the MWA.
    \\\textcopyright{}
    \citeay{offringa2015}.
  }
  \label{fig:rfi-mwa}
\end{figure}

%.......................................
\item
\emph{电离层干扰:}
\ac{ionosphere}是地球大气层上部被太阳辐射电离的部分,从约 \SI{60}{\km}
延伸至约 \SI{1000}{\km} 的高空,覆盖了大气层的\ac{thermosphere}以及
部分\ac{mesosphere}和\ac{exosphere},是地球\ac{magnetosphere}的内界
(如\autoref{fig:ionosphere} 所示)。
\acl{ionosphere}的大气已经非常稀薄,因此被太阳辐射中的紫外线和 X 射线电离的
空气分子所产生的自由电子在复合前可以短暂地自由活动,形成等离子体,能够对电磁波的
传播产生影响。
在 \SI{300}{\MHz} 的低频波段,\acl{ionosphere}主要对电磁波产生折射、
传播延迟、Faraday 旋转等影响,导致测量数据存在相位和幅度误差
\cite{intema2009,thompson2017}。
由于主要受太阳活动的影响,\acl{ionosphere}的状态会随时间和位置而发生剧烈变化,
因此对干涉阵列各天线产生的干扰程度也存在差异且时刻发生变化。
为了高质量的图像,必须实时(分钟量级???)校准观测数据 (ref???),
而且对每个天线施加的校准需要有针对性 (ref???),这将成为一个严重的计算负担,
还需要发展更加有效的\acl{ionosphere}校准算法 \cite{intema2009,deGasperin2018}。

\begin{figure}[tbp]
  \centering
  \includegraphics[width=0.5\textwidth]{atmosphere-with-ionosphere}
  \bicaption[大气层和电离层的关系]{%
    地球的大气层和电离层之间的关系。
    \acl{ionosphere}是大气层上部被太阳辐射电离的部分。
  }{%
    The relation between Earth's atmosphere and ionosphere, which
    is the ionized part of upper atmosphere.
    \\\textcopyright{}
    Bhamer,
    \url{https://en.wikipedia.org/wiki/File:Atmosphere_with_Ionosphere.svg},
    (2018-10-13), public domain.
  }
  \label{fig:ionosphere}
\end{figure}

%.......................................
\item
\emph{仪器效应:}
当代的干涉阵列通常由成千上万根天线组成。由于生产和安装过程的差异以及随环境和时间的变化,
每根天线的性能都不可能完全相同,导致所形成的\acl{stb}存在很多不确定因素,
而且各个\acl{station}的波束也互不相同。
对于采用数字\acl{bf}技术的\acl{pa}而言,波束的形态更会随着所指方向而发生大幅变化
\cite{smirnov2011iii,vanWeeren2016,jagannathan2017}。
因此,如果未能全面地校准\acl{stb},那么后续对其他仪器效应的校准、亮点源剥离、
前景去除等任务都会受到严重影响 \cite{noordam2004,neben2016}。
此外,还有一系列已知和未知的复杂仪器效应,比如:
显著的旁瓣 \cite{thyagarajan2015,mort2017}、
波束的频率依赖效应 \cite{liu2009ps,datta2010,morales2012}
(另见 \autoref{sec:eor-window} 和 \autoref{sec:fdeffect})、
\acl{pl} \cite{asad2016,asad2018,lenc2017}、
天线响应随频率的变化 \cite{bernardi2015,trott2017}、
信号传输过程中在电缆内的反射 \cite{beardsley2016}。
如何准确有效地校准仪器,发挥出仪器的设计性能,是目前最迫切的任务之一
\cite{wijnholds2010} (ref???)。

%.......................................
\item
\emph{海量数据:}
大型的干涉阵列将产生海量数据,如 \acs{ska1low} 的数据流量预计高达 TB/s,
由此引发出一系列难题 \cite{norris2011} (ref???),比如:
如何对原始数据进行实时相关处理?
如何传输和存储如此海量的数据?
如何实现有效的数字\acl{bf}和多波束技术?
如何进行海量数据的校准处理?
如何处理海量数据实现大视场高动态范围成像?
缓解或解决这些问题,不仅依赖于更快更高效的计算资源 \cite{magro2014,vermij2017},
建设新型的数据中心 \cite{chrysostomou2018},
还需要研发新算法以及编写新软件,优化数据处理流程,充分利用大规模并行计算资源
\cite{morales2009,} (ref???)。

\end{itemize}


%=====================================================================
\section{主要前景成分}
\label{sec:fg-intro}

%---------------------------------------------------------------------
\subsection{银河系\acl*{synrad}}  % 同步辐射

polarization leakage ...
spectral index variation ...

%---------------------------------------------------------------------
\subsection{银河系\acl*{brad}}  % 轫致辐射

TODO

%---------------------------------------------------------------------
\subsection{河外\acl*{pntsrc}}  % 河外点源

TODO

clustering effect ...

%---------------------------------------------------------------------
\subsection{\acl*{gc}}  % 星系团

radio halos, relics, mini-halos ...

\subsubsection{\acl*{rh}}  % 射电晕

radio halos ...

\subsubsection{\acl*{rr}}  % 射电遗迹

radio relics ...

\subsubsection{\acl*{rmh}}  % 迷你射电晕

radio mini-halos ...

%---------------------------------------------------------------------
\subsection{\acl*{sc}和\acl*{lsf}}  % 超星系团+大尺度纤维状结构

intergalactic medium (virial shocks) ...
superclusters, large-scale filaments ...


%=====================================================================
\section{前景处理方法}
\label{sec:fg-methods}

key characteristic: frequency structure difference between
the 21~cm signal and foreground emission.

导致问题更加困难的是,我们并不知道 EoR 信号的确切性质,只清楚。。。

(参见 \citeay{chapman2016} 及其所引文献)

%---------------------------------------------------------------------
\subsection{\acl*{fgrm}}  % 前景扣除法

parametric approaches, non-parametric approaches ...

%---------------------------------------------------------------------
\subsection{\acl*{fgav}}  % 前景回避法

2D power spectrum, EoR window


%=====================================================================
\section{\acl*{eor-window}}  % 再电离窗口
\label{sec:eor-window}

EoR window, foreground wedge, explanation ...

The EoR signal is the redshifted 21\,cm line emission from \acs{hi},
which is observed by a radio interferometer and form an image cube
$I(\B{\theta}, \nu)$, where the two angular dimensions $\B{\theta}$
translate into the transverse distances $\B{r}_{\bot}$ on the sky plane,
and the spectral dimension $\nu$ represents the line-of-sight distance
$r_{\parallel}$.

First, we take the 3D Fourier transform on the image cube:
\begin{equation}
  \label{eq:ps-3d-ft}
  V(\B{u}, \eta) = \B{\R{F}}(\{\B{u}, \eta\}, \{\B{\theta}, \nu\})
    I(\B{\theta}, \nu),
\end{equation}
where $\B{\theta}$ is the angular sizes on the sky plane, $\nu$ is the
frequency, and $(\B{u}, \eta)$ are the Fourier duals to
$(\B{\theta}, \nu)$, respectively.
Then we transform the coordinate from $(\B{u}, \eta)$ into the
cosmological coordinate $\B{k} = (k_x, k_y, k_z)$
(in units of \si{\per\cMpc}):
\begin{equation}
  \label{eq:coordinate-transform}
  V(\B{k}) = \M{J}(\B{k}, \{\B{u}, \eta\}) V(\B{u}, \eta).
\end{equation}
(coordinate transform ...)
We therefore obtain the 3D power spectrum by:
\begin{equation}
  \label{eq:3d-ps}
  P(\B{k}) = |V(\B{k})|^2.
\end{equation}

Considering that the signal is isotropic among the sky plane, we can
squeeze these two dimensions into $\kperp \equiv \sqrt{k_x^2 + k_y^2}$ by
cylindrically averaging the 3D power spectrum, and obtained the 2D
cylindrical power spectrum $P(\kperp, \klos)$ with $\klos \equiv k_z$.
The 2D cylindrical power spectrum has the advantage to better separate
the EoR signal from the foreground contamination, which is supposed to
be reside in the lower-right wedge-shape region, therefore defining the
EoR window.


%=====================================================================
\section{小结}

TODO


%% EOF
