%%
%% Copyright (c) 2018 Weitian LI <liweitianux@sjtu.edu.cn>
%% Creative Commons BY 4.0
%%

\chapter{补充公式}
\label{chap:formulas}


在本文所采用的平直 \lcdm/ 宇宙中,
\ac{delta-crit}随红移的变化关系可表示为 \cite{kitayama1996,randall2002}:
\begin{equation}
  \label{eq:delta-crit}
  \acs{delta-crit} = \frac{D(z=0)}{\acs{Dz}}
    \left[ \frac{3 (12\pi)^{2/3}}{20} \right]
    \left[1 + 0.0123 \log_{10} \acs{Ofz} \right] ,
\end{equation}
其中 \acs{Ofz} 是\acl{Ofz}:
\begin{equation}
  \label{eq:omega-fz}
  \acs{Ofz} = \frac{\acs{Om0} (1+z)^3}{\acs{Om0} (1+z)^3 + \acs{Ol0}} ,
\end{equation}
\acs{Dz} 是\acl{Dz},可由下述公式计算:
\begin{equation}
  \label{eq:growth-factor}
  D(x) = \frac{(x^3 + 2)^{1/2}}{x^{3/2}}
    \mathlarger{\int_0^x} y^{3/2} (y^3 + 2)^{-3/2} \,\D{y} ,
\end{equation}
并且 $x_0 \equiv (2 \acs{Ol0}/\acs{Om0})^{1/3}$、$x = x_0 / (1+z)$
[\textcite{peebles1980}, 式~(13.6)]。

在红移 $z$ 时的宇宙年龄具有如下解析计算形式:
\begin{align}
  \label{eq:universe-age}
  t(z; \acs{Om0})
    & = \frac{1}{\acs{H0}} \mathlarger{\int_z^{\infty}}
      \!\frac{\D{z'}}{(1+z')\sqrt{1 + z' (3+3z'+z'^2) \acs{Om0}}}
      \nonumber \\
    & = \frac{2}{3 \acs{H0} \sqrt{1-\acs{Om0}}} \sinh^{-1}
      \!\left( \sqrt{\frac{\acs{Om0}^{-1} - 1}{(1+z)^3}} \right),
\end{align}
[参见 \textcite{thomas2000}, 式~(18)]。

\acl{Hz}为:
\begin{equation}
  \label{eq:hubble-z}
  \acs{Hz} = \acs{H0} \, \acs{Ez}
    = \acs{H0} \sqrt{\acs{Om0} (1+z)^3 + \acs{Ol0}} ,
\end{equation}
其中 \acs{Ez} 是\acl{Ez} \cite{hogg1999}。
此时的宇宙临界密度为:
\begin{equation}
  \label{eq:rho-crit}
  \acs{rho-crit} = \frac{3 H^2(z)}{8 \pi \acs{G}} ,
\end{equation}
其中 \acs{G} 是\acl{G}。

星系团的\ac{r-vir}由下式给出:
\begin{equation}
  \label{eq:radius-virial}
  \acs{r-vir} = \left[
    \frac{3 \acs{M-vir}}{4\pi \acs{Delta-vir} \acs{rho-crit}}
  \right]^{1/3},
\end{equation}
其中 \acs{M-vir} 是星系团的\acl{M-vir}(亦可当作其总质量)、
\acs{Delta-vir} 是\acl{Delta-vir},由下式给出
\cite{kitayama1996,cassano2005}:
\begin{equation}
  \label{eq:delta-vir}
  \acs{Delta-vir} = 18\pi^2 \left[ 1 + 0.4093 \, w(z)^{0.9052} \right],
\end{equation}
并且 $w(z) \equiv \Omega_f^{-1}(z) - 1$。


The \emph{angular diameter distance} $D_A$ is defined as the ratio of
an object's physical transverse size to its (observed) angular size
(in radians).  Note that it does \emph{not} increases indefinitely
as $z \to \infty$, therefore more distant (e.g., $z > 1$)
objects of same physical size may actually appear larger!
The angular diameter distance is used to convert the observed angular
separations between sources into their proper separations, and it is
related to the \emph{transverse comoving distance} $D_M$ by
\cite{weinberg1972,peebles1993,hogg1999}:
\begin{equation}
  \label{eq:da-dm}
  D_A(z) = \frac{D_M(z)}{1 + z}.
\end{equation}

The \emph{luminosity distance} $D_L$ is defined by the relationship
between the measured bolometric (i.e., integrated over all frequencies)
flux $S_{\R{bolo}}$ and the object's intrinsic bolometric luminosity
$L_{\R{bolo}}$:
\begin{equation}
  \label{eq:dl-def}
  D_L \equiv \sqrt{\frac{L_{\R{bolo}}}{4\pi S_{\R{bolo}}}}.
\end{equation}
And it is related to the transverse comoving distance and angular
diameter distance by \cite{weinberg1972,hogg1999,ellis2007}:
\begin{equation}
  \label{eq:dl-dm-da}
  D_L(z) = (1+z) D_M(z) = (1+z)^2 D_A(z).
\end{equation}

In the radio frequency regime, the Rayleigh-Jeans approximation
always holds, therefore the spectral brightness $I_{\nu}$ can be
equivalently expressed by \emph{brightness temperature} $T_b(\nu)$
through the relation \cite{condon2016}:
\begin{equation}
  \label{eq:brightness-temp}
  T_b(\nu) \equiv \frac{I_{\nu} c^2}{2 k_B \nu^2}.
\end{equation}


%% EOF
