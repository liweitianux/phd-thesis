%%
%% Copyright (c) 2019 Weitian LI <liweitianux@sjtu.edu.cn>
%% Creative Commons BY 4.0
%%

\documentclass{beamer}

\usetheme{metropolis}
\metroset{
  progressbar=foot,
}

\setsansfont{Fira Sans Light}[
  BoldFont={Fira Sans Medium}
]
\setmonofont{Fira Code Light}[
  BoldFont={Fira Code Medium}
]

\usepackage{newtxsf}

\usepackage{xeCJK}
\setCJKsansfont{Source Han Sans SC Light}[
  BoldFont={Source Han Sans SC Medium}
]
\xeCJKsetup{PunctStyle=kaiming}

\usepackage{hyperref}
\hypersetup{
  pdfstartview={Fit}
}

% Suppress the navigation bar
\beamertemplatenavigationsymbolsempty

\AtBeginSection[]{
  \begin{frame}{目~~录}
    \tableofcontents[currentsection]
  \end{frame}
}

\newcommand{\email}[1]{\href{mailto:#1}{\texttt{#1}}}


%=====================================================================

\title[探测宇宙再电离时期]{%
  射电晕对宇宙再电离探测的影响和\texorpdfstring{\\}{}%
  基于深度学习的再电离信号分离新算法%
}
\author[李维天]{李维天 <\email{liweitianux@sjtu.edu.cn}>}
\institute{%
  物理与天文学院\\%
  上海交通大学%
}
\date{2019 年 ?? 月 ?? 日}
\subject{博士学位论文答辩}


%=====================================================================

\begin{document}

\maketitle

\begin{frame}{目~~录}
  \tableofcontents
\end{frame}

%=====================================================================
\section{绪~~论}

\begin{frame}{Hi}
Hello world! \textbf{Hello world!}

测试 \textbf{测试}

\texttt{code} \textbf{\texttt{code}}

Math:
$E = m c^2$.

X$X$X|x$x$x|M$M$M.

\end{frame}

%=====================================================================
\section{总~~结}

\begin{frame}{总~~结}
  全文总结...
\end{frame}

\begin{frame}[standout]
  Thank you!
\end{frame}

\end{document}
