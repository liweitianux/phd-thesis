%%
%% Copyright (c) 2019 Weitian LI <liweitianux@sjtu.edu.cn>
%% Creative Commons BY 4.0
%%

\usetheme{metropolis}
\metroset{progressbar=foot}

\setbeamertemplate{section in toc}[sections numbered]
% Numbered sections in section page
% Credit: https://github.com/matze/mtheme/issues/271#issuecomment-292799444
\makeatletter
\setbeamertemplate{section page}{
  \centering
  \begin{minipage}{25em}
    \raggedright
    \usebeamercolor[fg]{section title}
    \usebeamerfont{section title}
    \thesection.~\insertsectionhead\\[-1ex]
    \usebeamertemplate*{progress bar in section page}
    \par
    \ifx\insertsubsectionhead\@empty\else%
      \usebeamercolor[fg]{subsection title}%
      \usebeamerfont{subsection title}%
      \thesection.\thesubsection.~\insertsubsectionhead
    \fi
  \end{minipage}
  \par
  \vspace{\baselineskip}
}
\makeatother

% Change frametitle style for subsection pages
\setbeamerfont{subsec}{size=\large,series=\normalfont}
\setbeamercolor{subsec rule}{
  use=normal text,
  fg=normal text.fg!50!normal text.bg,
}
\makeatletter
\providebool{my@subsec}
\define@key{beamerframe}{subsec}[true]{%
  \booltrue{my@subsec}
  \begingroup
    \setbeamertemplate{frametitle}{%
      \vspace*{0.25\baselineskip}
      \usebeamerfont{subsec}%
      \usebeamercolor[fg]{normal text}%
      \insertframetitle\strut%
      \par\rule[0.5\baselineskip]{\textwidth}{0.2ex}%
      \par\vspace*{-0.5\baselineskip}
    }
}
\apptocmd{\beamer@reseteecodes}{%
  \ifbool{my@subsec}{
    \endgroup
    \boolfalse{my@subsec}
  }{}
}{}{}
\makeatother

\setbeamertemplate{frametitle continuation}{(\insertcontinuationcount)}
\setbeamertemplate{bibliography item}{\insertbiblabel}

\setbeamertemplate{itemize items}[circle]
\setbeamertemplate{itemize subitem}[triangle]
% Suppress itemize indentation
% Credit: https://tex.stackexchange.com/a/450317
\settowidth{\leftmargini}{\usebeamertemplate{itemize item}}
\addtolength{\leftmargini}{\labelsep}
\settowidth{\leftmarginii}{\usebeamertemplate{itemize subitem}}
\addtolength{\leftmarginii}{\labelsep}

%\usepackage{pgfpages}
%\setbeameroption{hide notes}  % only slides
%\setbeameroption{show only notes}  % only notes
%\setbeameroption{show notes on second screen=right}  % both

% Solarized theme
% Credit: https://ethanschoonover.com/solarized/
\definecolor{SolBase02}{HTML}{073642}
\definecolor{SolBase3}{HTML}{fdf6e3}
\definecolor{SolOrange}{HTML}{cb4b16}
\definecolor{SolGreen}{HTML}{859900}
%
\setbeamercolor{normal text}{fg=SolBase02, bg=SolBase3}
%\setbeamercolor{normal text}{bg=white}  % for print
\setbeamercolor{alerted text}{fg=SolOrange}
\setbeamercolor{example text}{fg=SolGreen}

\setsansfont{Fira Sans Light}[
  BoldFont={Fira Sans Medium}
]
\setmonofont{Fira Code Light}[
  BoldFont={Fira Code Medium}
]

\usepackage{bm}
\usepackage{newtxsf}
\newcommand{\R}[1]{\text{#1}}  % text math alphabets
\newcommand{\Ce}{\R{e}}  % constant e
\newcommand{\Ci}{\R{i}}  % constant i
\newcommand{\Cpi}{\piup}  % upright 'pi', provided by 'newtxsf' package
\newcommand{\B}[1]{\bm{\mathsf{#1}}}  % single-letter bold math
\newcommand{\D}[1]{\R{d}#1}
\newcommand{\diff}[2]{\frac{\D{#1}}{\D{#2}}}
\newcommand{\pdiff}[2]{\frac{\partial #1}{\partial #2}}

\usepackage{xeCJK}
\setCJKsansfont{Source Han Sans SC Light}[
  BoldFont={Source Han Sans SC Medium}
]

\usepackage{hyperref}
\hypersetup{
  pdfstartview={Fit}
}

\usepackage{appendixnumberbeamer}
\usepackage{booktabs}
\usepackage{csquotes}

\usepackage{caption}
\captionsetup{%
  figurename={图},
  format=plain,
  labelformat=simple,
  labelsep=period,
  justification=centering,
  textfont={footnotesize},
  labelfont={footnotesize},
}

\usepackage{keyval}
\usepackage{etoolbox}

\makeatletter
% Set the keys (arguments) for including a figure
\newlength{\myfigure@width}
\newlength{\myfigure@height}
\newtoggle{myfigure@vertcap}
\define@key{myfigure}{width}{\setlength\myfigure@width{#1}}
\define@key{myfigure}{height}{\setlength\myfigure@height{#1}}
\define@key{myfigure}{vertcap}[true]{\toggletrue{myfigure@vertcap}}
%
% Custom command to include a figure with a horizontal/vertical caption
\NewDocumentCommand{\myfigure}{m m m}{
  \setkeys{myfigure}{width=0pt, height=0pt, #1}
  %
  % Get the size that a figure is being rendered
  % Credit: https://tex.stackexchange.com/a/3664
  \newsavebox{\myfigure@box}
  \ifdimequal{\myfigure@height}{0pt}{
    \savebox{\myfigure@box}{%
      \includegraphics[width=\myfigure@width]{#2}}
  }{%
    \savebox{\myfigure@box}{%
      \includegraphics[height=\myfigure@height]{#2}}
  }
  \settowidth{\myfigure@width}{\usebox{\myfigure@box}}
  \settoheight{\myfigure@height}{\usebox{\myfigure@box}}
  %
  \begin{figure}[!h]
    \centering
    \usebox{\myfigure@box}
    \iftoggle{myfigure@vertcap}{%
      \hspace{-0.5em}%
      % Credit: https://tex.stackexchange.com/a/44433
      \rotatebox{90}{%
        \begin{minipage}{\myfigure@height}
          \caption{#3}
        \end{minipage}
      }
    }{%
      \caption{#3}
    }
  \end{figure}
  % Credit: https://tex.stackexchange.com/a/18174
  \global\let\myfigure@box\relax
}
%
\makeatother

\usepackage{xparse}
% Credit: https://tex.stackexchange.com/a/376366
\ExplSyntaxOn
\NewDocumentCommand{\cspace}{O{1em}}{%
  \tl_map_inline:nn {空} { \makebox[#1]{\phantom{##1}} }
}
\ExplSyntaxOff

\usepackage{siunitx}
% siunitx settings and new units
\sisetup{
  range-phrase=\text{--},
  range-units=single,
  product-units=repeat,
  list-separator={, },
  list-final-separator={, and },
  separate-uncertainty=true,
  detect-all,  % detecting fonts
}
%
\DeclareSIUnit\arcsec{arcsec}
\DeclareSIUnit\arcmin{arcmin}
\DeclareSIUnit\cMpc{cMpc}  % comoving Mpc
\DeclareSIUnit\cGpc{cGpc}  % comoving Gpc
\DeclareSIUnit\deg{deg}
\DeclareSIUnit\dyne{dyn}
\DeclareSIUnit\erg{erg}
\DeclareSIUnit\esu{esu}
\DeclareSIUnit\franklin{Fr}
\DeclareSIUnit\gauss{G}
\DeclareSIUnit\hubble{\ensuremath{\mathit{h}}}
\DeclareSIUnit\jansky{Jy}
\DeclareSIUnit\lightyear{ly}
\DeclareSIUnit\parsec{pc}
\DeclareSIUnit\rayleigh{Rayleigh}
\DeclareSIUnit\solarmass{\ensuremath{\text{M}_{\odot}}}
\DeclareSIUnit\statcoulomb{statC}
\DeclareSIUnit\year{yr}
%
\DeclareSIUnit\kpc{\kilo\parsec}
\DeclareSIUnit\mJy{\milli\jansky}
\DeclareSIUnit\mK{\milli\kelvin}
\DeclareSIUnit\Gpc{\giga\parsec}
\DeclareSIUnit\Gyr{\giga\year}
\DeclareSIUnit\Mpc{\mega\parsec}
\DeclareSIUnit\Myr{\mega\year}
\DeclareSIUnit\uG{\micro\gauss}

\usepackage{journalabbrv}

\usepackage[%
  backend=biber,
  style=authoryear-comp,
]{biblatex}

\AtBeginBibliography{
  \linespread{1.0}
  \footnotesize
}
\addbibresource{../references.bib}

\graphicspath{
  {./}
  {figures/}
  {../figures/}
  {../figures/self/}
  {../sjtuthesis/}
}

% Change 'emph' style to bold face
\let\emph\relax  % there's no \RedeclareTextFontCommand
\DeclareTextFontCommand{\emph}{\boldmath\bfseries}

\newcommand{\email}[1]{\href{mailto:#1}{\texttt{#1}}}
\newcommand{\doi}[1]{\href{https://doi.org/#1}{\textsc{doi}:#1}}
\newcommand{\ads}[1]{\href{http://adsabs.harvard.edu/abs/#1}{\textsc{ads}:#1}}
\newcommand{\arxiv}[1]{\href{https://arixv.org/abs/#1}{\textsc{arXiv}:#1}}

\endinput
