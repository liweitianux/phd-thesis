%%
%% Copyright (c) 2018 Weitian LI <liweitianux@sjtu.edu.cn>
%% Creative Commons BY 4.0
%%

% Class options:
%   bachelor|master|doctor	% 必选项
%   fontset=adobe|windows	% 只测试了adobe
%   oneside|twoside		% 单面/双面打印
%   openany|openright		% 可以在奇数(默认)或者偶数页开新章
%   zihao=-4|5			% 正文字号: 小四、五号(默认)
%   review			% 盲审论文,隐去作者姓名、学号、
%				% 导师姓名、致谢、发表论文和参与的项目
%   submit			% 定稿提交的论文,插入签名扫描版的
%				% 原创性声明、授权声明
\documentclass[doctor, openright, twoside]{sjtuthesis}


%% Fonts
\defaultfontfeatures{Mapping=tex-text}
\setmainfont{TeX Gyre Pagella}
\setsansfont{TeX Gyre Heros}
\setmonofont{M+ 1mn}
\setmathfont{TeX Gyre Pagella Math}

%% Chinese
\setCJKmainfont[ItalicFont={Noto Sans CJK SC}]{Noto Serif CJK SC}
\setCJKsansfont{AR PL KaitiM GB}
\setCJKmonofont{Noto Sans Mono CJK SC}
\xeCJKsetup{PunctStyle=kaiming}

% References database
\addbibresource{bib/thesis.bib}


%%
%% Title pages
%%

\title{射电晕对探测宇宙再电离信号的影响}
\author{李维天}
\studentnumber{0130729026}
\advisor{徐海光 教授}
%\coadvisor{某某 教授}
\school{上海交通大学}
\institute{物理与天文学院}
\major{物理学}
\defenddate{2018 年 mm 月 dd 日}

\englishtitle{\textsc{%
  The Impacts of Radio Halos on Detecting the
  Epoch of Reionization Signal
}}
\englishauthor{\textsc{Weitian Li}}
\englishadvisor{Prof. \textsc{Haiguang Xu}}
%\englishcoadvisor{Prof. \textsc{Uom Uom}}
\englishinstitute{School of Physics and Astronomy}
\englishschool{Shanghai Jiao Tong University}
\englishlocation{Shanghai, China}
\englishmajor{Physics}
\englishdate{mmm dd, 2018}


\begin{document}

\maketitle

\makeenglishtitle

\makeatletter
\ifsjtu@submit\relax
  \includepdf{pdf/original.pdf}
  \cleardoublepage
  \includepdf{pdf/authorization.pdf}
  \cleardoublepage
\else
\ifsjtu@review\relax
% exclude the original claim and authorization
\else
  \makeDeclareOriginal
  \makeDeclareAuthorization
\fi
\fi
\makeatother

\frontmatter
\pagestyle{main}

% 中文摘要,约 3000 字
\begin{abstract}
TODO
\end{abstract}

%---------------------------------------------------------------------

\begin{englishabstract}
TODO
\end{englishabstract}


\tableofcontents
\listoffigures
\addcontentsline{toc}{chapter}{\listfigurename}
\listoftables
\addcontentsline{toc}{chapter}{\listtablename}
\listofalgorithms
\addcontentsline{toc}{chapter}{算法索引}

%%
%% Copyright (c) 2018 Weitian LI <liweitianux@sjtu.edu.cn>
%% Creative Commons BY 4.0
%%

\DeclareAcronym{h}{
  short = \si{\hubble},
  long = 无量纲的 Hubble 常数,
  class = symbol,
}

\DeclareAcronym{H0}{
  short = \ensuremath{H_0},
  long = 当前的 Hubble 常数,
  sort = H,
  class = symbol,
}

\DeclareAcronym{ns}{
  short = \ensuremath{n_s},
  long = 原初扰动的标量谱指数,
  class = symbol,
}

\DeclareAcronym{Ob0}{
  short = \ensuremath{\Omega_b},
  long = 当前的宇宙重子物质密度,
  sort = Omega-b,
  class = symbol,
}

\DeclareAcronym{Om0}{
  short = \ensuremath{\Omega_m},
  long = 当前的宇宙物质密度(包含重子物质和暗物质),
  sort = Omega-m,
  class = symbol,
}

\DeclareAcronym{Ol0}{
  short = \ensuremath{\Omega_{\Lambda}},
  long = 当前的宇宙常数或暗能量密度,
  sort = Omega-Lambda,
  class = symbol,
}

\DeclareAcronym{sigma8}{
  short = \ensuremath{\sigma_8},
  long = 原初扰动在 \SI{8}{\per\hubble\Mpc} 尺度上的幅度,
  class = symbol,
}

\endinput
 % 主要符号、缩略词对照表

\mainmatter
\pagestyle{main}

\include{tex/intro}
\include{tex/example}
\include{tex/faq}
\begin{summary}

这里是全文总结内容。

\end{summary}


\appendix
\renewcommand\theequation{\Alph{chapter}--\arabic{equation}}
\renewcommand\thefigure{\Alph{chapter}--\arabic{figure}}
\renewcommand\thetable{\Alph{chapter}--\arabic{table}}
\renewcommand\thealgorithm{\Alph{chapter}--\arabic{algorithm}}

%% 附录内容,本科学位论文可以用翻译的文献替代。
\include{tex/app_setup}
\include{tex/app_eq}
\include{tex/app_cjk}
\include{tex/app_log}

\backmatter

\printbibliography[heading=bibintoc]

% 致谢、发表论文、申请专利、参与项目、简历
% 用于盲审的论文需隐去致谢、发表论文、申请专利、参与的项目
%
% 研究生学位论文送盲审印刷格式的统一要求
% http://www.gs.sjtu.edu.cn/inform/3/2015/20151120_123928_738.htm
%
\makeatletter
% 盲审删去删去致谢页
\ifsjtu@review\relax\else
  \include{tex/ack}  % 致谢
\fi

\ifsjtu@bachelor
  % 学士学位论文要求在最后有一个英文大摘要,单独编页码
  \pagestyle{biglast}
  \include{tex/end_english_abstract}
\else
  % 盲审论文中,发表学术论文及参与科研情况等仅以第几作者注明即可,
  % 不要出现作者或他人姓名
  \ifsjtu@review\relax
    \include{tex/pubreview}
    \include{tex/projectsreview}
  \else
    \include{tex/pub}  % 发表论文
    \include{tex/projects}  % 参与的项目
  \fi
\fi
\makeatother

% \include{tex/patents}  % 申请专利
% %%
%% Copyright (c) 2018 Weitian LI <liweitianux@sjtu.edu.cn>
%% Creative Commons BY 4.0
%%

\begin{resume}
  \begin{resumesection}{基本情况}
    李维天,男,1991 年 9 月生于湖南邵阳。
  \end{resumesection}

  \begin{resumelist}{教育背景}
    \item 2013 年 9 月至今,上海交通大学,博士研究生,物理学
    \item 2009 年 9 月至 2013 年 6 月,上海交通大学,本科,应用物理学
  \end{resumelist}

  \begin{resumesection}{研究兴趣}
    低频射电观测,宇宙再电离时期探测,数据分析
  \end{resumesection}

  \begin{resumelist}{联系方式}
    \item E-mail: \email{liweitianux@sjtu.edu.cn}, \hspace{1em} \email{wt@liwt.net}
    \item Github: \url{https://github.com/liweitianux}
  \end{resumelist}
\end{resume}
  % 个人简历

\end{document}

% EOF
