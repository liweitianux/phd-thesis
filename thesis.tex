%%
%% Copyright (c) 2018 Weitian LI <liweitianux@sjtu.edu.cn>
%% Creative Commons BY 4.0
%%

% Class options:
%   bachelor|master|doctor	% 必选项
%   fontset=fandol|adobe|windows
%   oneside|twoside
%   openany|openright
%   zihao=-4|5			% 正文字号: 小四、五号(默认)
%   english			% 启用英文模版
%   review			% 盲审论文,隐去作者姓名、学号、
%				% 导师姓名、致谢、发表论文和参与的项目
%   submit			% 定稿提交的论文,插入签名扫描版的
%				% 原创性声明、授权声明
\documentclass[doctor, openright, twoside]{sjtuthesis}

\usepackage{soul}

\defaultfontfeatures{Mapping=tex-text}
\setmainfont{TeX Gyre Pagella}
\setsansfont{TeX Gyre Heros}
\setmonofont{M+ 1mn}
\setmathfont{TeX Gyre Pagella Math}

\setCJKmainfont[ItalicFont={Noto Sans CJK SC}]{Noto Serif CJK SC}
\setCJKsansfont{AR PL KaitiM GB}
\setCJKmonofont{Noto Sans Mono CJK SC}
\xeCJKsetup{PunctStyle=kaiming}

\newcommand{\doi}[1]{\href{https://doi.org/#1}{doi:#1}}
\newcommand{\email}[1]{\href{mailto:#1}{\texttt{#1}}}

% Journals
\newcommand\aap{Astronomy and Astrophysics}
\newcommand\aapr{Astronomy and Astrophysics Review}
\newcommand\aaps{Astronomy and Astrophysics Supplement Series}
\newcommand\aj{Astronomical Journal}
\newcommand\ao{Applied Optics}
\newcommand\aplett{Astrophysics Letters}
\newcommand\apj{Astrophysical Journal}
\newcommand\apjl{Astrophysical Journal Letters}
\newcommand\apjs{Astrophysical Journal Supplement Series}
\newcommand\apss{Astrophysics and Space Science}
\newcommand\araa{Annual Review of Astronomy and Astrophysics}
\newcommand\arep{Astronomy Reports}
\newcommand\aspc{ASP Conference Series}
\newcommand\baas{Bulletin of the American Astronomical Society}
\newcommand\caa{Chinese Astronomy and Astrophysics}
\newcommand\cjaa{Chinese Journal of Astronomy and Astrophysics}
\newcommand\fcp{Fundamentals of Cosmic Physics}
\newcommand\grl{Geophysics Research Letters}
\newcommand\iaucirc{IAU Cirulars}
\newcommand\icarus{Icarus}
\newcommand\japa{Journal of Astrophysics and Astronomy}
\newcommand\jcap{Journal of Cosmology and Astroparticle Physics}
\newcommand\jcp{Journal of Chemical Physics}
\newcommand\jgr{Journal of Geophysics Research}
\newcommand\jqsrt{Journal of Quantitiative Spectroscopy and Radiative Transfer}
\newcommand\jrasc{Journal of the RAS of Canada}
\newcommand\memras{Memoirs of the RAS}
\newcommand\memsai{Memoire della Societa Astronomica Italiana}
\newcommand\mnassa{Monthly Notes of the Astronomical Society of Southern Africa}
\newcommand\mnras{Monthly Notices of the Royal Astronomical Society}
\newcommand\na{New Astronomy}
\newcommand\nar{New Astronomy Review}
\newcommand\nat{Nature}
\newcommand\nphysa{Nuclear Physics A}
\newcommand\pra{Physical Review A: General Physics}
\newcommand\prb{Physical Review B: Solid State}
\newcommand\prc{Physical Review C}
\newcommand\prd{Physical Review D}
\newcommand\pre{Physical Review E}
\newcommand\prl{Physical Review Letters}
\newcommand\pasa{Publications of the Astronomical Society of Australia}
\newcommand\pasp{Publications of the Astronomical Society of the Pacific}
\newcommand\pasj{Publications of the Astronomical Society of Japan}
\newcommand\physrep{Physics Reports}
\newcommand\planss{Planetary Space Science}
\newcommand\procspie{Proceedings of the Society of Photo-Optical Instrumentation Engineers}
\newcommand\qjras{Quarterly Journal of the RAS}
\newcommand\sci{Science}
\newcommand\skytel{Sky and Telescope}
\newcommand\solphys{Solar Physics}
\newcommand\ssr{Space Science Reviews}


%=====================================================================

% 不得超过 36 字
\title{%
  射电晕对宇宙再电离探测的影响和 \\
  基于深度学习的再电离信号分离新算法%
}
\keywords{% 4-6 个
  低频射电天文,
  宇宙再电离时期,
  射电晕,
  弱信号分离,
  深度学习,
  卷积去噪自编码器
}
\author{李维天}
\studentnumber{0130729026}
\advisor{徐海光~教授}
\school{上海交通大学}
\institute{物理与天文学院}
\major{物理学}
\defenddate{2018 年 mm 月 dd 日}

\englishtitle{\textsc{%
  Impacts of Radio Halos on \\
  EoR Detection and \\
  A Novel Deep-learning-based Method to \\
  Separate the EoR Signal
}}
\englishkeywords{%
  low-frequency radio astronomy,
  epoch of reionization,
  radio halos,
  weak signal separation,
  deep learning,
  convolutional denoising autoencoder
}
\englishauthor{\textsc{Weitian Li}}
\englishadvisor{Prof. \textsc{Haiguang Xu}}
\englishinstitute{School of Physics and Astronomy}
\englishschool{Shanghai Jiao Tong University}
\englishlocation{Shanghai, China}
\englishmajor{Physics}
\englishdate{mmm dd, 2018}

\addbibresource{references.bib}

%=====================================================================
\begin{document}

\maketitle

\makeatletter
\ifsjtu@submit\relax
  \includepdf{pdf/originality.pdf}
  \pdfbookmark[0]{\sjtu@label@originality}{originality}
  \cleardoublepage
  \includepdf{pdf/authorization.pdf}
  \pdfbookmark[0]{\sjtu@label@authorization}{authorization}
  \cleardoublepage
\else
\ifsjtu@review\relax
% exclude the originality and authorization declarations
\else
  \makeDeclareOriginality
  \makeDeclareAuthorization
\fi
\fi
\makeatother

%---------------------------------------------------------------------
\frontmatter

% 中文摘要,约 3000 字
\begin{abstract}
TODO
\end{abstract}

%---------------------------------------------------------------------

\begin{englishabstract}
TODO
\end{englishabstract}


\tableofcontents
\listoffigures
\addcontentsline{toc}{chapter}{\listfigurename}
\listoftables
\addcontentsline{toc}{chapter}{\listtablename}
% \listofalgorithms
% \addcontentsline{toc}{chapter}{\listalgorithmname}

\include{tex/symbols}

%---------------------------------------------------------------------
\mainmatter
\pagestyle{main}

%%
%% Copyright (c) 2018-2019 Weitian LI <liweitianux@sjtu.edu.cn>
%% Creative Commons BY 4.0
%%

\chapter{绪论}
\label{chap:introduction}

%=====================================================================
\section{研究背景}
\label{sec:background}

理解宇宙的结构、起源和演化,是人类孜孜不倦地追求的目标,在哲学和科学中占据重要地位.
经过无数人的努力,宇宙学的\ac{bbt}终于得以建立.
该理论已被大量观测证据所支持,比如星系的红移--距离关系(即 Hubble 定律)、
\ac{cmb}辐射、星系的大尺度分布、早期元素丰度、等等,
是目前宇宙学的标准模型.

根据大爆炸宇宙学模型,宇宙起源于约 138 亿年前的一次大爆炸,然后随着宇宙的膨胀,
温度以及能量密度都逐渐降低,宇宙主要经历了\ac{inflation}、\ac{bbn}、
\ac{recomb}、\ac{da}、形成第一代天体、\ac{reion}、形成星系及大尺度结构
等阶段,如\autoref{fig:univ-history} 所示.

\begin{figure}[!htp]
  \centering
  \includegraphics[width=\textwidth]{universe-history}
  \bicaption[宇宙的演化历史]{%
    宇宙从大爆炸到今天的演化历史.
  }{%
    The evolution of the Universe from the Big Bang
    to the present.
    \\\textcopyright{}
    \acuse{bicep,cern,nasa}
    \acs{bicep}2/\acs{cern}/\acs{nasa}; CC0 1.0.
  }
  \label{fig:univ-history}
\end{figure}

大爆炸之后约 40 万年,宇宙已冷却至大约 \SI{3000}{\kelvin},
于是自由电子被结合到中性原子之中,与重子物质脱耦的光子开始在宇宙中自由传播,
形成弥漫于整个宇宙的背景辐射,即今天所探测到的 \ac{cmb} 辐射.
但是,此时尚未形成发光的天体,因此宇宙进入了\acl{da}.
随着物质的密度扰动在引力作用下增长,第一代天体开始形成并产生辐射,使得重子物质
再次被逐步电离,宇宙从此结束\acl{da}并走入\ac{eor}.
随着各尺度上的天体结构的逐步形成与演化,重子物质被充分电离,宇宙也演化形成今天的格局.

\begin{figure}[!htp]
  \centering
  \includegraphics[width=\textwidth]{cosmic-stages-dare}
  \bicaption[宇宙的黑暗时期与再电离时期示意图]{%
    宇宙的\acl{da}与\acl{eor}示意图,其中显示了\acl{aoi} (A)、\acl{da} (B)、
    \acl{cd} (C) 以及\acl{eor} (D, E).
    上方的粗曲线显示了理论预测的 \hisignal/的强度.
  }{%
    A schematic showing the \acs{da} and the \acs{eor}
    of the Universe, mainly including the \acs{aoi} (A),
    the \acs{da} (B), the \acs{cd} (C), and the \acs{eor} (D, E).
    The thick curve in the top panel shows the predicted intensity
    of the 21\,cm signal.
    \\\textcopyright{}
    \acuse{dare}\ac{dare},
    \url{http://lunar.colorado.edu/dare/science.html}, (2018-09-23).
  }
  \label{fig:cosmic-stages}
\end{figure}

我们已借助多波段观测掌握了大量有关宇宙近期演化
($\acs{z} \lesssim 6$;宇宙已充分电离之后)的信息;
通过研究 \ac{cmb},我们对宇宙的早期历史
($z \gtrsim 1100$;自由电子\acl{recomb}之前)有了深刻理解.
然而,我们对中间的那段时期($z \sim \numrange{6}{1100}$)却知之甚少.
这段时期可细分为以下四个阶段\cite{koopmans2015}:
\acl{aoi} (\acs{aoi}; $z \sim \numrange{200}{1100}$)、
\acl{da} ($z \sim \numrange{30}{200}$)、
\acl{cd} (\acs{cd}; $z \sim \numrange{15}{30}$)
以及\acl{eor} ($z \sim \numrange{6}{15}$),
如\autoref{fig:cosmic-stages} 所示.
对于其中距离我们相对较近的\acl{eor},
我们目前仅获得非常有限的间接观测信息,比如:
该时期的\ac{hi}对高红移类星体的 Ly$\alpha$ 吸收 \cite{becker2001}、
该时期的自由电子对 \ac{cmb} 光子的 Thomson 散射 \cite{kaplinghat2003}.
但是,我们仍然缺乏来自\acl{eor}的直接观测证据,
对该时期的基本性质和关键物理过程仍不清楚,比如:
第一代天体是何时以及如何形成的?
主要的电离源有哪些以及它们是如何影响再电离过程的?
电离氢区的尺度以及演化过程如何?
研究\acl{eor}的对于理解宇宙早期结构形成以及星系的形成与演化有重要意义,
是建立完整的宇宙演化图景的关键环节之一.
具体请参见 \citeay{fan2006}, \citeay{morales2010},
\citeay{pritchard2012}, \citeay{zaroubi2013},
\citeay{koopmans2015} 等综述文.

在\acl{eor}及其之前的\acl{da},尽管缺乏发光天体可供观测,
但是宇宙中丰富的\acl{hi}所辐射的 21\,cm 谱线
(以下简称 \emph{\hisignal/};
详见 \autoref{sec:21cm-signal})为探测该时期提供了有效途径.
对 \hisignal/的探测是目前对\acl{eor}及其之前的\acl{da}开展系统性
研究的最直接而有效的观测手段 \cite{koopmans2015,furlanetto2016}.

\acl{hi}的 21\,cm 谱线的本征频率约为 \SI{1420}{\MHz}.
源自\acl{eor}的 \hisignal/(以下简称 \emph{EoR 信号})经历显著红移后
应出现在约 \SIrange{90}{200}{\MHz},对应低频射电波段.
EoR 信号到达地球时已非常微弱,仅约几 \si{\mK} 至十几 \si{\mK},
因此需要具有极高灵敏度的低频观测设备才能捕获该信号.
目前的主流技术是采用大规模低频干涉阵列,已建成或正在建设的干涉阵列主要有:
\ac{21cma} \cite{zheng2016}、
\ac{gmrt} \cite{paciga2011}、
\ac{mwa} \cite{bowman2013,tingay2013}、
\ac{lofar} \cite{vanHaarlem2013}、
\ac{lwa} \cite{ellingson2009}、
\ac{paper} \cite{parsons2010}、
\ac{hera} \cite{deboer2017}、
\ac{ska} \cite{mellema2013,koopmans2015}.
然而,利用干涉阵列探测 EoR 信号仍面临诸多困难与挑战\cite{wijnholds2010},
其中主要包括:
识别并扣除强烈的前景干扰、扣除人工源的\ac{rfi}、修正电离层的扰动、
苛刻的仪器校准要求、海量数据处理和高动态范围成像.

在低频射电波段,强烈的前景干扰(主要源自银河系以及河外点源;
详见 \autoref{sec:fg-intro})比待探测的 EoR 信号高出约 5 个数量级;
即便按干涉阵列所测量的天空亮度涨落来衡量,前景干扰的涨落也是待测信号的数千倍
\cite{zaroubi2013}.
如何准确把握前景干扰并将其有效扣除,是成功探测 EoR 信号的关键.
由于低频射电观测和巡天数据的严重不足,我们对该波段的前景的了解非常有限,
无法达到探测 EoR 信号所要求的精度.
因此,我们需要挖掘已有海量的中高频射电观测以及其他多波段观测数据,
并结合逐渐增长的低频观测数据,深入理解低频射电前景辐射,构建并完善前景模型,
为识别并扣除前景干扰提供有力支撑.

虽然在本质上,前景辐射的频谱是光滑的,而 EoR 信号的频谱呈锯齿状,
两者具有很好的可区分性 \cite{wang2006,jelic2008,harker2009,wang2013}.
然而在实际情况中,受到干涉阵列的复杂仪器效应、观测干扰、数据处理技术的限制
等因素的影响,前景频谱的光滑性遭到破坏,导致 EoR 信号的提取变得尤其困难
\cite{liu2009ps,labropoulos2009,gehlot2018,mertens2018}.
如何研发出行之有效的前景处理和 EoR 信号提取算法,亦是当前的重要研究课题.


%=====================================================================
\section{研究内容}
\label{sec:content}

本文的研究内容分为以下两部分:
\begin{itemize}
\item
\emph{改进低频射电天空的模拟:}
深刻理解各前景成分的性质(如强度、空间分布、频谱结构)并充分把握它们对 EoR 探测
的干扰方式,是研发具有针对性的前景去除和 EoR 信号分离算法的前提与关键 (ref???).
由于复杂的仪器效应和严重的观测干扰,低频干涉阵列的系统校准非常困难
\cite{noordam2004,intema2009,wijnholds2010,barry2016,gehlot2018},
严重制约仪器达到探测 EoR 信号所要求的极高灵敏度.
在现阶段缺乏足够可用的高质量低频射电观测数据的情况下,挖掘已有多波段观测数据并
准确模拟低频射电天空,是开展前景干扰研究以及 EoR 信号分离算法研发的可行办法.

\hspace{2\ccwd}%
在诸多前景成分之中,银河系的弥散辐射 [包括\ac{synrad}和\ac{brad}]
以及河外\ac{pntsrc}辐射是最主要的成分,目前已被广泛地研究和较好地理解
\cite{shaver1999,diMatteo2004,gleser2008,liu2012,murray2017,spinelli2018}.
除此之外,剩下的前景辐射主要来自河外\ac{extsrc},其中包括:
\ac{icm} \cite{feretti2012} 产生的\ac{rh}、\ac{rr}和\ac{rmh}、
\ac{gc}之外的\ac{igm} \cite{keshet2004}、
以及\ac{lsf} \cite{vazza2015}.
对于这些河外\acl{extsrc},已获得的观测证据不多,在低频射电波段更是不足.
关于它们将具体如何影响 EoR 探测,目前的理解非常有限,亟待深入且系统的研究.

\hspace{2\ccwd}%
与其他几类河外\acl{extsrc}相比,\acl{rh}拥有更多的观测证据和理论研究,支撑我们
构建一个更佳的模型用来模拟\acl{rh}的低频射电辐射,改进低频射电天空的模拟,
进而在考虑干涉阵列的实际仪器效应的情况下,有效地评估\acl{rh}对 EoR 探测的影响.

\item
\emph{研发 EoR 信号分离新算法:}
为了提取淹没于前景干扰中的 EoR 信号,一系列方法已被提出来用于处理前景
(详见 \autoref{sec:fg-methods}).
这些前景处理方法可大致分为\ac{fgrm}和\ac{fgavd}两大类,
但都依赖于一个重要前提:前景辐射的频谱必须非常光滑.
据此,这些方法通过构建一个模型来拟合光滑的前景成分并扣除,或者在功率谱空间尽量
避开前景污染区域,从而提取出微弱的 EoR 信号 \cite{chapman2016}.

\hspace{2\ccwd}%
然而在实际情况中,干涉阵列的\ac{beam}存在频率依赖效应(以下简称\emph{波束效应}),
即\acl{beam}的形状随观测频率而变化,因此 CLEAN 后残留的前景源会产生沿频率方向
快速变化的涨落,严重破坏前景频谱的光滑性 \cite{liu2009ps},
导致现有方法无法有效分辨前景干扰与 EoR 信号并分离出 EoR 信号
(详见 \autoref{sec:fdeffect}).

\hspace{2\ccwd}%
考虑到干涉列阵的\acl{beam}的形状非常复杂,为现有方法打造一个实际可用的模型
用以克服上述复杂的波束效应将很困难 \cite{lochner2015},
因此研发基于\ac{dl}的 EoR 信号分离新算法是一条更加可行且具有吸引力的途径
\cite{herbel2018,vafaeiSadr2019},
通过从数据中学习知识并自适应地优化模型,达到克服波束效应并分离 EoR 信号的目标.

\end{itemize}

本文的研究目标是:
(1)改进\acl{rh}的模拟,考虑干涉阵列的实际仪器效应,
获得更精细、更符合实际的低频射电天空的模拟图像,
进而有效地评估\acl{rh}对 EoR 探测的影响.
(2)研发基于\acl{dl}的能够有效克服干涉阵列的波束效应的 EoR 信号分离新算法,
并运用到上述模拟数据进行测试和优化.


%=====================================================================
\section{研究方案}
\label{sec:plan}

本文遵循以下主要步骤开展开展工作,完成研究内容,达到研究目标:
\begin{enumerate}
\item
调研\acl{rh}的相关理论研究和观测证据,理解其形成机制和演化规律,
构建模型并编程实现模拟\acl{rh}的低频射电辐射.
搜集\acl{rh}的现有观测数据,调节模型的参数,获得可靠的模拟结果.

\item
采用典型的干涉阵列(如 SKA1-Low)的布局方案,对上一步所得的\ac{skymap}
开展模拟观测,得到\ac{vis}数据,再利用 CLEAN 算法成像获得相应的“观测”图像.
通过这种\ac{e2e}模拟,干涉阵列的复杂仪器效应(如本文关注的波束效应)得以
有效地整合到研究流程之中.

\item
基于上述模拟所得的“观测”图像,利用一维和二维\ac{ps},对比\acl{rh}和
EoR 信号的异同,量化\acl{rh}在运用\acl{fgrm}或\acl{fgavd}的情况下
对 EoR 探测的影响,有效评估\acl{rh}作为前景干扰成分的重要程度.

\item
对比分析目前的主流\acl{dl}方法,筛选出合适的算法并加以必要的改进,
适用到此处的 EoR 信号分离场景.
利用已有模拟数据对算法进行训练和调优,挑选出满足要求的最佳算法.

\end{enumerate}


%=====================================================================
\section{本文框架}
\label{sec:structure}

本文余下章节安排如下:
\autoref{chap:interferometry}将介绍射电天文学和射电干涉技术的基础知识,
包括基本辐射理论、天线原理、干涉阵列及综合孔径成像等.
在\autoref{chap:detection},我们将介绍利用\acl{hi}
\hisignal/探测宇宙再电离时期的方法和困难、以及前景处理方法.
在\autoref{chap:simulation},我们首先模拟各前景成分和 EoR 信号的\acl{skymap},
然后进行干涉阵列的模拟观测,得到整合了实际仪器效应的观测图像.
据此,我们在\autoref{chap:halo}借助\acl{ps}量化评估\acl{rh}对
EoR 探测的具体影响.
\autoref{chap:cdae}将阐述我们提出的基于\acl{dl}的 EoR 分离新算法并演示其效果.
最后,我们对全文进行总结并简要展望.

全文采用一个由 \lcdm/ 模型描述的平直宇宙,具体参数为:
$\acs{H0} = 100\,\acs{h} = \SI{71}{\km\per\second\per\Mpc}$、
$\acs{Om0} = 0.27$、
$\acs{Ol0} = 1 - \acs{Om0} = 0.73$、
$\acs{Ob0} = 0.046$、
$\acs{ns} = 0.96$ 以及 $\acs{sigma8} = 0.81$.
如无额外说明,本文给出的误差对应 \SI{68}{\percent} 的置信水平;
使用的幂律谱形式为 $\acs{S-nu} \propto \acs{freq}^{-\acs{sidx}}$,
其中 \acs{S-nu} 为\acl{S-nu}、\acs{sidx} 为\acl{sidx}.
本文使用的中文术语遵循《英汉天文学名词数据库》
\footnote{英汉天文学名词数据库:\url{http://astrodict.china-vo.org/}}.


%% EOF

\begin{summary}

这里是全文总结内容。

\end{summary}


\appendix

%%
%% Copyright (c) 2018-2019 Weitian LI <liweitianux@sjtu.edu.cn>
%% Creative Commons BY 4.0
%%

\chapter{Fokker--Planck 方程数值算法}
\label{chap:fpsolver}

在磁流体中,带电粒子可与其中的湍流发生随机散射而通过\emph{二阶 Fermi 加速}机制
获得能量\cite{fermi1949,fermi1954,davis1956},
该过程可由 Fokker--Planck 方程描述
\cite{schlickeiser1989,eilek1991,schlickeiser2002}.
当加速区域是均匀的且远大于散射的\ac{mfp}时,Fokker--Planck 方程可被简化到只
依赖于时间和能量\cite{park1995,park1996}:
\begin{equation}
  \label{eq:fp-generic}
  \pdiff{u(x,t)}{t} = \frac{1}{A(x)} \pdiff{}{x}
    \left[ B(x) u(x,t) + C(x) \pdiff{u(x,t)}{x} \right]
    - \frac{u(x,t)}{T(x)} + Q(x) ,
\end{equation}
其中
$x$ 是能量或动量,
$u(x,t)$ 为粒子的能量分布,
$A(x)$ 为相位因子(如果 $x$ 表示能量,该项等于 1;
如果 $x$ 表示动量,该项等于 $4\Cpi x^2$),
$B(x)$、$C(x)$、$T(x)$ 和 $Q(x)$ 分别描述了粒子的
平流 (advection)、扩散 (diffusion)、逃逸 (escape) 和注入 (injection).
这几个系数需满足 $A(x) > 0, C(x) > 0, T(x) \ge 0, Q(x) \ge 0$.

然而,简化后的 Fokker--Planck 方程仍然只能在有限的几种特殊情况下获得解析解,
而对于一般情况则必须求助于数值算法.
由 \citeay{chang1970} 提出的有限差分法 (finite difference scheme)
是一种有效的算法,下文将具体介绍该算法.


%=====================================================================
\section{数值算法}

采用一个包含 $M+1$ 个点的网格对 $x$ 离散化:$x_m (m = 0, 1, \cdots, M)$.
在网格单元中点处,$x$ 的值定义为:
\begin{equation}
  \label{eq:x-mid}
  x_{m+1/2} = (x_m + x_{m+1}) / 2 ,
\end{equation}
同时 $\Delta x$ 的值定义为:
\begin{equation}
  \label{eq:dx-mid}
  \Delta x_{m+1/2} = x_{m+1} - x_m ,
\end{equation}
于是可得:
\begin{equation}
  \label{eq:dx}
  \Delta x_m = (x_{m+1} - x_{m-1}) / 2 .
\end{equation}
对时间 $t$ 离散化,并采用记法:
\begin{equation}
  \label{eq:u-t}
  u_m^n = u(x_m, t_n) .
\end{equation}

接着,定义 $x$-空间的粒子流量 $F(x,t)$ 为:
\begin{equation}
  \label{eq:fp-f}
  F(x,t) = B(x) u(x,t) + C(x) \pdiff{u(x,t)}{x} .
\end{equation}
于是\emph{无流量 (no-flux) 边界条件}可写为\cite{park1995}:
\begin{equation}
  \label{eq:no-flux}
  F(x_0, t) = F(x_M, t) = 0 .
\end{equation}

对\autoref{eq:fp-generic} 离散化可得:
\begin{equation}
  \label{eq:fp-disc}
  \frac{u_m^{n+1} - u_m^n}{\Delta t}
    = \frac{1}{A_m} \frac{F_{m+1/2}^{n+1} - F_{m-1/2}^{n+1}}{\Delta x_m}
      - \frac{u_m^{n+1}}{T_m} + Q_m ,
\end{equation}
其中 $\Delta t = t_{n+1} - t_n$ 为时间步长.
同时\autoref{eq:no-flux} 的无流量边界条件成为:
\begin{equation}
  \label{eq:no-flux-disc}
  F_{-1/2}^{n+1} = F_{M+1/2}^{n+1} = 0 .
\end{equation}

\citeay{chang1970} 给出如下 $F_{m+1/2}^{n+1}$ 的表达式:
\begin{align}
  \label{eq:fp-f-chang70}
  F_{m+1/2}^{n+1} & = (1 - \delta_{m+1/2}) B_{m+1/2} u_{m+1}^{n+1}
      + \delta_{m+1/2} B_{m+1/2} u_m^{n+1}
      + C_{m+1/2} \frac{u_{m+1}^{n+1} - u_m^{n+1}}{\Delta x_{m+1/2}} \\
    & = \frac{C_{m+1/2}}{\Delta x_{m+1/2}} \left[
      W_{m+1/2}^{+} u_{m+1}^{n+1} - W_{m+1/2}^{-} u_m^{n+1} \right] ,
\end{align}
其中
\begin{align}
  \delta_m & = \frac{1}{w_m} - \frac{1}{\exp(w_m) - 1} ,
    \label{eq:fp-delta-m} \\
  W_m^{\pm} & = W_m \exp(\pm w_m / 2) ,
    \label{eq:fp-Wm-pm} \\
  W_m & = w_m / [2 \sinh(w_m / 2)] ,
    \label{eq:fp-Wm} \\
  w_m & = \frac{B_m}{C_m} \Delta x_m .
    \label{eq:fp-wm}
\end{align}
考虑到 $|w_m|$ 可能会非常大或者非常小,为了使数值计算更稳定,可采用\cite{park1996}:
\begin{equation}
  \label{eq:fp-Wm-calc}
  W_m = \left\{
    \begin{alignedat}{2}
      & \left[ 1 + \frac{w_m^2}{24} + \frac{w_m^4}{1920} \right]^{-1} ,
        & \quad\text{when~} |w_m| < 0.1 , \\
      & \frac{|w_m| \exp(-|w_m|/2)}{1 - \exp(-|w_m|)} ,
        & \quad\text{when~} |w_m| \ge 0.1 .
    \end{alignedat}
  \right.
\end{equation}

将\autoref{eq:fp-f-chang70} 代入\autoref{eq:fp-disc},
可整理成如下三对角 (tridigonal) 线性方程组:
\begin{equation}
  \label{eq:fp-tridigonal}
  \left\{
    \begin{aligned}
      -a_m u_{m-1}^{n+1} + b_m u_m^{n+1} - c_m u_{m+1}^{n+1} & = r_m, \\
      a_0 = c_M & = 0 ,
    \end{aligned}
  \right.
\end{equation}
其中各项系数如下:
\begin{equation}
  \label{eq:fp-coefs}
  \left\{
    \begin{aligned}
      a_m & = \frac{\Delta t}{A_m \Delta x_m}
        \frac{C_{m-1/2}}{\Delta x_{m-1/2}} W_{m-1/2}^{-} , \\
      c_m & = \frac{\Delta t}{A_m \Delta x_m}
        \frac{C_{m+1/2}}{\Delta x_{m+1/2}} W_{m+1/2}^{+} , \\
      b_m & = 1 + \frac{\Delta t}{A_m \Delta x_m}
        \left[ \frac{C_{m-1/2}}{\Delta x_{m-1/2}} W_{m-1/2}^{+}
        + \frac{C_{m+1/2}}{\Delta x_{m+1/2}} W_{m+1/2}^{-} \right]
        + \frac{\Delta t}{T_m} , \\
      r_m & = u_m^n + \Delta t Q_m .
    \end{aligned}
  \right.
\end{equation}
注意,上式无法给出 $b_0$ 和 $b_M$,这需要利用边界条件[\autoref{eq:no-flux-disc}]
重新推导系数,可得:
\begin{equation}
  \label{eq:fp-coefs-b}
  \left\{
    \begin{aligned}
      b_0 & = 1 + \frac{\Delta t}{A_0 \Delta x_0}
        \frac{C_{1/2}}{\Delta x_{1/2}} W_{1/2}^{-}
        + \frac{\Delta t}{T_0} , \\
      b_M & = 1 + \frac{\Delta t}{A_M \Delta x_M}
        \frac{C_{M-1/2}}{\Delta x_{M-1/2}} W_{M-1/2}^{+}
        + \frac{\Delta t}{T_M} .
    \end{aligned}
  \right.
\end{equation}
\autoref{eq:fp-tridigonal} 的线性方程组可由快速的三对角矩阵算法
(亦称 Thomas 算法)求解\cite{press1992}.


%=====================================================================
\section{算法测试}

简单/解析情形对比...

%% EOF


%---------------------------------------------------------------------
\backmatter

\printbibliography[heading=bibintoc]

% 致谢、发表论文、申请专利、参与项目、简历
% 用于盲审的论文需隐去致谢、发表论文、申请专利、参与的项目

\makeatletter
% 盲审删去删去致谢页
\ifsjtu@review\relax\else
  %%
%% Copyright (c) 2018 Weitian LI <liweitianux@sjtu.edu.cn>
%% Creative Commons BY 4.0
%%

\begin{thanks}

首先感谢父母和亲人,是他们的无私关爱和支持让我得以取得今天的成绩。

感谢导师徐海光教授的悉心培养,让我获得做人与做学问的全面成长。

感谢师兄师姐的指导,特别是王婧颖师姐。
感谢课题组里一起奋斗的小伙伴:
朱正浩、胡丹、马志贤、单晨曦、郑东超、朱永凯、连晓丽、刘宇星。
感谢好友朱睿敏以及 Jeffrey Hsu 对论文的帮助。
还需要感谢国家天文台和上海天文台的老师、学长和学姐的关心和指导。

感谢国家自然科学基金委
(项目编号:11433002、11621303、11835009、61371147、11125313)
和科学技术部(项目编号:2018YFA0404601、2017YFF0210903)
为本工作提供的资助。

上海交通大学的互联网是国内首屈一指的,如果没有这个优良条件,这里的一切都将是浮云。
我也将无法忘记在这里结识的一群好朋友。

本工作的完成离不开以下项目/工具/网站的支持:
\href{https://github.com/adobe-fonts}{Adobe 开源字体} (%
\href{https://github.com/adobe-fonts/source-han-serif}{思源\textbf{宋体}},
\href{https://github.com/adobe-fonts/source-serif-pro}{Source \textbf{Serif} \textit{Pro}},
\href{https://github.com/adobe-fonts/source-sans-pro}{\textsf{Source \textbf{Sans} \textit{Pro}}},
\href{https://github.com/adobe-fonts/source-code-pro}{\texttt{Source \textbf{Code} \textit{Pro}}}),
\href{https://arxiv.org/}{arXiv},
\href{http://ads.harvard.edu/}{Astrophysics Data System},
\href{https://www.bing.com/dict}{Bing 词典},
\href{https://www.debian.org/}{Debian GNU/Linux},
\href{https://www.gnu.org/s/emacs/}{Emacs},
\href{https://fcitx-im.org/}{Fcitx},
\href{https://git-scm.com/}{Git},
\href{https://github.com/}{Github},
\href{https://www.google.com/}{Google 搜索},
\href{https://github.com/sjtug/SJTUThesis}{交大学位论文模板},
\href{https://keras.io/}{Keras},
\href{https://www.latex-project.org/}{\LaTeX},
\href{https://www.libreoffice.org/}{LibreOffice},
\href{https://www.mozilla.org/en-US/firefox/}{Mozilla Firefox},
\href{https://ned.ipac.caltech.edu/}{NASA/IPAC Extragalactic Database},
\href{https://okular.kde.org/}{Okular},
\href{https://www.openssh.com/}{OpenSSH},
\href{https://github.com/OxfordSKA/OSKAR}{OSKAR},
\href{https://www.python.org/}{Python} (%
\href{https://www.astropy.org/}{Astropy},
\href{https://jupyter.org/}{Jupyter},
\href{https://matplotlib.org/}{matplotlib},
\href{https://www.numpy.org/}{NumPy},
\href{https://pandas.pydata.org/}{pandas},
\href{https://scipy.org/}{SciPy}),
\href{http://ds9.si.edu/}{SAOImage DS9},
\href{https://shadowsocks.org/}{ShadowSocks},
\href{https://stackoverflow.com/}{Stack Overflow},
\href{https://syncthing.net/}{Syncthing},
\href{https://www.tensorflow.org/}{TensorFlow},
\href{https://github.com/tmux/tmux}{Tmux},
\href{https://www.vim.org/}{Vim},
\href{https://www.wechat.com/}{微信},
\href{https://www.wikipedia.org/}{Wikipedia},
\href{http://wps-community.org/}{WPS Office},
\href{https://sourceforge.net/projects/wsclean/}{WSClean},
\href{https://www.xfce.org/}{XFCE},
\href{http://www.zsh.org/}{Zsh}。
此外,感谢 \href{https://www.dragonflybsd.org/}{DragonFly BSD}
项目及其 IRC 上那群很棒的人。

最后,感谢女友尹璐璐,感谢她十年来坚定不移的支持和鼓励,她是我今生的至爱。

\end{thanks}

\fi
\ifsjtu@bachelor
  % 本科学位论文要求在最后有一个英文大摘要,单独编页码
  \pagestyle{biglast}
  \include{tex/app-abstract}
\else
  % 盲审论文中,发表学术论文及参与科研情况等仅以第几作者注明即可,
  % 不要出现作者或他人姓名
  \ifsjtu@review\relax
    \include{tex/publications-review}
    % \include{tex/projects-review}
  \else
    %%
%% Copyright (c) 2018 Weitian LI <liweitianux@sjtu.edu.cn>
%% Creative Commons BY 4.0
%%

\begin{publications}{99}
  \linespread{1.1}

  \item
    \textsc{\emph{Li, Weitian}; Xu, Haiguang; Ma, Zhixian; Zhu, Ruimin;
    Hu, Dan; Zhu, Zhenghao; Shan, Chenxi; Zhu, Jie; Wu, Xiang-Ping}.
    \enquote{\it Separating the EoR Signal with a Convolutional Denoising
      Autoencoder: a Deep-learning-based Method,}
    2018, Monthly Notices of the Royal Astronomical Society Letters, ???
  \item
    \textsc{\emph{Li, Weitian}; Xu, Haiguang; Ma, Zhixian; Hu, Dan;
    Zhu, Zhenghao; Shan, Chenxi; Wang, Jingying; Gu, Junhua;
    Lian, Xiaoli; Zheng, Qian; Zhu, Jie; Wu, Xiang-Ping}.
    \enquote{\it Contribution of Radio Halos to the Foreground for
      SKA EoR Experiments,}
    2018, The Astrophysical Journal, ???
  \item
    \textsc{Ma, Zhixian; Xu, Haiguang; Zhu, Jie; Hu, Dan;
    \emph{Li, Weitian}; Shan, Chenxi; Zhu, Zhenghao; Gu, Liyi;
    Liu, Chengze; Wu, Xiang-Ping}.
    \enquote{\it A Machine Learning Based Morphological Classification
      of 14,251 Radio AGNs Selected from the Best--Heckman Sample,}
    2018, The Astrophysical Journal Supplement Series, ???
  \item
    \textsc{Hu, Dan; Xu, Haiguang; Kang, Xi; \emph{Li, Weitian};
    Zhu, Zhenghao; Ma, Zhixian; Shan, Chenxi; Zhang, Zhongli;
    Gu, Liyi; Liu, Chengze; Wu, Xiang-Ping}.
    \enquote{\it A Study of the Merger History of the Galaxy Group
      HCG 62 Based on X-ray Observations and SPH Simulations,}
    2018, The Astrophysical Journal, ???,
    \arxiv{1811.05782}
  \item
    \textsc{Zheng, Qian; Johnston-Hollitt, Melanie;
    Duchesne, Stefan\,W; \emph{Li, Weitian}}.
    \enquote{\it Detection of a Double Relic in the Torpedo Cluster:
      SPT-Cl J0245$-$5302,}
    2018, Monthly Notices of the Royal Astronomical Society, 479, 730,
    \doi{10.1093/mnras/sty1467},
    \arxiv{1803.06634}
  \item
    \textsc{Ma, Zhixian; Zhu, Jie; \emph{Li, Weitian}; Xu, Haiguang}.
    \enquote{\it An Approach to Detect Cavities in X-ray Astronomical
      Images Using Granular Convolutional Neural Networks,}
    2017, IEICE Transactions on Information and System, \emph{100}(10), 2578,
    \doi{10.1587/transinf.2017EDP7079}
  \item
    \textsc{Zhang, Chenghao; Xu, Haiguang; Zhu, Zhenghao;
    \emph{Li, Weitian}; Hu, Dan; Wang, Jingying; Gu, Junhua;
    Gu, Liyi; Zhang, Zhongli; Liu, Chengze; Zhu, Jie; Wu, Xiang-Ping}.
    \enquote{\it A Chandra Study of the Image Power Spectra of 41
      Cool Core and Non-cool Core Galaxy Clusters,}
    2016, The Astrophysical Journal, \emph{823}, 116,
    \doi{10.3847/0004-637X/823/2/116},
    \arxiv{1604.04127}
  \item
    \textsc{Zhu, Zhenghao; Xu, Haiguang; Wang, Jingying; Gu, Junhua;
    \emph{Li, Weitian}; Hu, Dan; Zhang, Chenghao; Gu, Liyi; An, Tao;
    Liu, Chengze; Zhang, Zhongli; Zhu, Jie; Wu, Xiang-Ping}.
    \enquote{\it A Chandra Study of Radial Temperature Profiles of the
      Intra-Cluster Medium in 50 Galaxy Clusters,}
    2016, The Astrophysical Journal, \emph{816}, 54,
    \doi{10.3847/0004-637X/816/2/54},
    \arxiv{1511.04699}
  \item
    \textsc{Wang, Jingying; Xu, Haiguang; An, Tao; Gu, Junhua;
    Guo, Xueying; \emph{Li, Weitian}; Wang, Yu; Liu, Chengze;
    Martineau-Huynh, Olivier; Wu, Xiang-Ping}.
    \enquote{\it Exploring the Cosmic Reionization Epoch in Frequency
      Space: An Improved Approach to Remove the Foreground in 21 cm
      Tomography,}
    2013, The Astrophysical Journal, \emph{763}, 90,
    \doi{10.1088/0004-637X/763/2/90},
    \arxiv{1211.6450}

  \vspace{1ex} % conference papers
  \item
    \textsc{Ma, Zhixian; Zhu, Jie; \emph{Li, Weitian}; Xu, Haiguang}.
    \enquote{\it Radio Galaxy Morphology Generation Using Residual
      Convolutional Autoencoder and Gaussian Mixture Models,}
    2018, IEEE 25th International Conference on Image Processing (ICIP),
    Athens, Greece, October 7--10, 2018, 3044--3048,
    \doi{10.1109/ICIP.2018.8451231}
  \item
    \textsc{Ma, Zhixian; Zhu, Jie; \emph{Li, Weitian}; Xu, Haiguang}.
    \enquote{\it Radio Galaxy Morphology Generation Using DNN Autoencoder
      and Gaussian Mixture Models,}
    2018, IEEE 14th International Conference on Signal Processing (ICSP),
    Beijing, China, August 12--14, 2018, 522--526,
    \arxiv{1806.00398}
  \item
    \textsc{Ma, Zhixian; Zhu, Jie; \emph{Li, Weitian}; Xu, Haiguang}.
    \enquote{\it Detection of Point Sources in X-ray Astronomical Images
      Using Elliptical Gaussian Filters,}
    2017, IEEE 2nd International Conference on Image, Vision and Computing (ICIVC),
    Chengdu, China, June 2--4, 2018, 36--40,
    \doi{10.1109/ICIVC.2017.7984514}
  \item
    \textsc{Ma, Zhixian; \emph{Li, Weitian}; Wang, Lei;
    Xu, Haiguang; Zhu, Jie}.
    \enquote{\it X-ray Astronomical Point Sources Recognition Using
      Granular Binary-tree SVM,}
    2016, IEEE 13th International Conference on Signal Processing (ICSP),
    Chengdu, China, November 6--10, 2018, 1021--1026,
    \doi{10.1109/ICSP.2016.7877984}
\end{publications}

    % \include{tex/projects}
    % \include{tex/patents}
    %%
%% Copyright (c) 2018 Weitian LI <liweitianux@sjtu.edu.cn>
%% Creative Commons BY 4.0
%%

\begin{resume}
  \begin{resumesection}{基本情况}
    李维天,男,1991 年 9 月生于湖南邵阳。
  \end{resumesection}

  \begin{resumelist}{教育背景}
    \item 2013 年 9 月至今,上海交通大学,博士研究生,物理学
    \item 2009 年 9 月至 2013 年 6 月,上海交通大学,本科,应用物理学
  \end{resumelist}

  \begin{resumesection}{研究兴趣}
    低频射电观测,宇宙再电离时期探测,数据分析
  \end{resumesection}

  \begin{resumelist}{联系方式}
    \item E-mail: \email{liweitianux@sjtu.edu.cn}, \hspace{1em} \email{wt@liwt.net}
    \item Github: \url{https://github.com/liweitianux}
  \end{resumelist}
\end{resume}

  \fi
\fi
\makeatother

\end{document}
