%%
%% Copyright (c) 2018-2019 Weitian LI <liweitianux@sjtu.edu.cn>
%% Creative Commons BY 4.0
%%

% Class options:
%   bachelor|master|doctor	% 必选项
%   fontset=fandol|adobe|windows
%   oneside|twoside
%   openany|openright
%   zihao=-4|5			% 正文字号: 小四、五号(默认)
%   english			% 启用英文模版
%   review			% 盲审论文,隐去作者姓名、学号、
%				% 导师姓名、致谢、发表论文和参与的项目
%   submit			% 定稿提交的论文,插入签名扫描版的
%				% 原创性声明、授权声明
\documentclass[
  doctor,
  openright,
  twoside,
  %review,
]{sjtuthesis}

\usepackage{microtype}
\usepackage{fontawesome5}

\defaultfontfeatures{Mapping=tex-text}
\setmainfont{Source Serif Pro}
\setsansfont{Source Sans Pro}
\setmonofont{Source Code Pro}
\setmathfont{Asana Math}

\setCJKmainfont{Source Han Serif SC}
\setCJKsansfont{FandolKai}
\setCJKmonofont{FandolFang}
\xeCJKsetup{PunctStyle=kaiming}

\graphicspath{{./}{figures/}{figures/self/}}

% biblatex
% Credit: http://www.khirevich.com/latex/biblatex/
\ExecuteBibliographyOptions{
  backref=true,
  mincitenames=1,
  maxcitenames=2,
}
% period after the authors
\renewcommand{\labelnamepunct}{\addperiod\space}
% discard the period at the very end of a record
\renewcommand{\finentrypunct}{}
% discard the period after the doi link
\renewcommand{\newunitpunct}{\addspace\midsentence}
% use lowercase English for backref
\DefineBibliographyStrings{english}{%
  backrefpage  = {\lowercase{s}ee p.}, % for single page number
  backrefpages = {\lowercase{s}ee pp.} % for multiple page numbers
}
\DeclareDelimFormat{finalnamedelim}{\addspace\&\space}
\DeclareFieldFormat[book]{title}{\textbf{#1}\addperiod\space}
\DeclareFieldFormat[article,inproceedings]{title}{\textit{#1}\addperiod\space}
\DeclareFieldFormat[inproceedings]{booktitle}{\textit{#1}\addperiod\space}
\DeclareFieldFormat[article,inproceedings]{volume}{\textbf{#1}\addcolon\space}
\DeclareFieldFormat{pages}{#1\addperiod\space}
%
\newcommand{\citeay}[1]{\citeauthor{#1} \citeyear{#1} \parencite{#1}}
\newcommand{\citet}[1]{\citeauthor{#1} (\citeyear{#1})\cite{#1}}

% Change 'emph' style to bold face
\let\emph\relax  % there's no \RedeclareTextFontCommand
\DeclareTextFontCommand{\emph}{\boldmath\bfseries}

% Continuous footnote numbering
\counterwithout{footnote}{chapter}

% hyperref's \autoref command
% Credit: https://tex.stackexchange.com/a/66150
\def\equationautorefname~#1\null{式~(#1)\null}
\def\chapterautorefname~#1\null{第{#1}章\null}
\def\sectionautorefname{\textsection}
\def\subsectionautorefname{\textsection}
\def\figureautorefname{图}
\def\tableautorefname{表}

% custom commands
\newcommand{\doi}[1]{\href{https://doi.org/#1}{\textsc{doi}:#1}}
\newcommand{\arxiv}[1]{\href{https://arixv.org/abs/#1}{\textsc{arXiv}:#1}}
\newcommand{\email}[1]{\href{mailto:#1}{\texttt{#1}}}
\newcommand{\circled}[1]{{\large\textcircled{\small #1}}}
\newcommand{\Cpi}{\symup{\pi}}  % constant pi
\newcommand{\Ce}{\symup{e}}  % constant e
\newcommand{\Ci}{\symup{i}}  % complex i
\newcommand{\R}[1]{\mathrm{#1}}  % text math alphabets
\newcommand{\B}[1]{\symbf{#1}}  % single-letter bold math
\newcommand{\V}[1]{\symbfit{#1}}  % bold italic
\newcommand{\M}[1]{\symbfsf{#1}}  % bold sans-serif (matrix)
\newcommand{\D}[1]{\,\mathrm{d}#1}
\newcommand{\diff}[2]{\frac{\mathrm{d}#1}{\mathrm{d}#2}}
\newcommand{\klos}{\text{$k_{\parallel}$}}
\newcommand{\kperp}{\text{$k_{\bot}$}}
%
\def\lcdm/{$\Lambda$CDM}
\def\hisignal/{21\,cm~信号}

% Journals
\newcommand\aap{Astronomy and Astrophysics}
\newcommand\aapr{Astronomy and Astrophysics Review}
\newcommand\aaps{Astronomy and Astrophysics Supplement Series}
\newcommand\aj{Astronomical Journal}
\newcommand\ao{Applied Optics}
\newcommand\aplett{Astrophysics Letters}
\newcommand\apj{Astrophysical Journal}
\newcommand\apjl{Astrophysical Journal Letters}
\newcommand\apjs{Astrophysical Journal Supplement Series}
\newcommand\apss{Astrophysics and Space Science}
\newcommand\araa{Annual Review of Astronomy and Astrophysics}
\newcommand\arep{Astronomy Reports}
\newcommand\aspc{ASP Conference Series}
\newcommand\baas{Bulletin of the American Astronomical Society}
\newcommand\caa{Chinese Astronomy and Astrophysics}
\newcommand\cjaa{Chinese Journal of Astronomy and Astrophysics}
\newcommand\fcp{Fundamentals of Cosmic Physics}
\newcommand\grl{Geophysics Research Letters}
\newcommand\iaucirc{IAU Cirulars}
\newcommand\icarus{Icarus}
\newcommand\japa{Journal of Astrophysics and Astronomy}
\newcommand\jcap{Journal of Cosmology and Astroparticle Physics}
\newcommand\jcp{Journal of Chemical Physics}
\newcommand\jgr{Journal of Geophysics Research}
\newcommand\jqsrt{Journal of Quantitiative Spectroscopy and Radiative Transfer}
\newcommand\jrasc{Journal of the RAS of Canada}
\newcommand\memras{Memoirs of the RAS}
\newcommand\memsai{Memoire della Societa Astronomica Italiana}
\newcommand\mnassa{Monthly Notes of the Astronomical Society of Southern Africa}
\newcommand\mnras{Monthly Notices of the Royal Astronomical Society}
\newcommand\na{New Astronomy}
\newcommand\nar{New Astronomy Review}
\newcommand\nat{Nature}
\newcommand\nphysa{Nuclear Physics A}
\newcommand\pra{Physical Review A: General Physics}
\newcommand\prb{Physical Review B: Solid State}
\newcommand\prc{Physical Review C}
\newcommand\prd{Physical Review D}
\newcommand\pre{Physical Review E}
\newcommand\prl{Physical Review Letters}
\newcommand\pasa{Publications of the Astronomical Society of Australia}
\newcommand\pasp{Publications of the Astronomical Society of the Pacific}
\newcommand\pasj{Publications of the Astronomical Society of Japan}
\newcommand\physrep{Physics Reports}
\newcommand\planss{Planetary Space Science}
\newcommand\procspie{Proceedings of the Society of Photo-Optical Instrumentation Engineers}
\newcommand\qjras{Quarterly Journal of the RAS}
\newcommand\sci{Science}
\newcommand\skytel{Sky and Telescope}
\newcommand\solphys{Solar Physics}
\newcommand\ssr{Space Science Reviews}

% New math operators
\DeclareMathOperator{\erf}{erf}
\DeclareMathOperator{\erfc}{erfc}
\DeclareMathOperator{\sinc}{sinc}

% siunitx settings and new units
\sisetup{
  range-phrase=\text{--},
  range-units=single,
  product-units=repeat,
  list-separator={, },
  list-final-separator={, and },
  separate-uncertainty=true,
}
%
\DeclareSIUnit\arcsec{arcsec}
\DeclareSIUnit\arcmin{arcmin}
\DeclareSIUnit\cMpc{cMpc}  % comoving Mpc
\DeclareSIUnit\cGpc{cGpc}  % comoving Gpc
\DeclareSIUnit\deg{deg}
\DeclareSIUnit\erg{erg}
\DeclareSIUnit\gauss{G}
\DeclareSIUnit\hubble{\ensuremath{\mathit{h}}}
\DeclareSIUnit\jansky{Jy}
\DeclareSIUnit\lightyear{ly}
\DeclareSIUnit\parsec{pc}
\DeclareSIUnit\rayleigh{Rayleigh}
\DeclareSIUnit\solarmass{\ensuremath{\mathrm{M}_{\odot}}}
\DeclareSIUnit\year{yr}
%
\DeclareSIUnit\keV{\kilo\electronvolt}
\DeclareSIUnit\kHz{\kilo\hertz}
\DeclareSIUnit\kpc{\kilo\parsec}
\DeclareSIUnit\mJy{\milli\jansky}
\DeclareSIUnit\mm{\milli\meter}
\DeclareSIUnit\mK{\milli\kelvin}
\DeclareSIUnit\Gpc{\giga\parsec}
\DeclareSIUnit\Gyr{\giga\year}
\DeclareSIUnit\MHz{\mega\hertz}
\DeclareSIUnit\Mpc{\mega\parsec}
\DeclareSIUnit\Myr{\mega\year}
\DeclareSIUnit\uG{\micro\gauss}

\usepackage{acro}  % acronyms, symbols, glossaries
\acsetup{
  list-heading=chapter,
}
%%
%% Copyright (c) 2018 Weitian LI <liweitianux@sjtu.edu.cn>
%% Creative Commons BY 4.0
%%

\DeclareAcronym{h}{
  short = \si{\hubble},
  long = 无量纲的 Hubble 常数,
  class = symbol,
}

\DeclareAcronym{H0}{
  short = \ensuremath{H_0},
  long = 当前的 Hubble 常数,
  sort = H,
  class = symbol,
}

\DeclareAcronym{ns}{
  short = \ensuremath{n_s},
  long = 原初扰动的标量谱指数,
  class = symbol,
}

\DeclareAcronym{Ob0}{
  short = \ensuremath{\Omega_b},
  long = 当前的宇宙重子物质密度,
  sort = Omega-b,
  class = symbol,
}

\DeclareAcronym{Om0}{
  short = \ensuremath{\Omega_m},
  long = 当前的宇宙物质密度(包含重子物质和暗物质),
  sort = Omega-m,
  class = symbol,
}

\DeclareAcronym{Ol0}{
  short = \ensuremath{\Omega_{\Lambda}},
  long = 当前的宇宙常数或暗能量密度,
  sort = Omega-Lambda,
  class = symbol,
}

\DeclareAcronym{sigma8}{
  short = \ensuremath{\sigma_8},
  long = 原初扰动在 \SI{8}{\per\hubble\Mpc} 尺度上的幅度,
  class = symbol,
}

\endinput

%%
%% Copyright (c) 2018-2019 Weitian LI <liweitianux@sjtu.edu.cn>
%% Creative Commons BY 4.0
%%

\DeclareAcronym{21cma}{
  short = 21CMA,
  long = 21 CentiMeter Array,
  class = glossary,
}

\DeclareAcronym{ab}{
  short = antenna beam,
  long = 天线波束,
  class = glossary,
}

\DeclareAcronym{am}{
  short = AM,
  long = 调幅,
  foreign = amplitude modulation,
  class = glossary,
}

\DeclareAcronym{aoi}{
  short = Age of Ignorance,
  long = 无知时期,
  class = glossary,
}

\DeclareAcronym{as}{
  short = aperture synthesis,
  long = 综合孔径,
  pdfstring = 综合孔径,
  class = glossary,
}

\DeclareAcronym{astron}{
  short = ASTRON,
  long = 荷兰射电天文研究所,
  foreign = Netherlands Institute for Radio Astronomy,
  class = glossary,
}

\DeclareAcronym{bbn}{
  short = Big Bang Nucleosynthesis,
  long = 太初核合成,
  class = glossary,
}

\DeclareAcronym{bbt}{
  short = Big Bang Theory,
  long = 大爆炸理论,
  class = glossary,
}

\DeclareAcronym{beam}{
  short = beam,
  long = 波束,
  class = glossary,
}

\DeclareAcronym{bf}{
  short = beamforming,
  long = 波束成形,
  class = glossary,
}

\DeclareAcronym{bicep}{
  short = BICEP,
  long = Background Imaging of Cosmic Extragalactic Polarization,
  class = glossary,
}

\DeclareAcronym{bighorns}{
  short = BIGHORNS,
  long = Broadband Instrument for Global Hydrogen Reionisation Signal,
  class = glossary,
}

\DeclareAcronym{brad}{
  short = Bremsstrahlung,
  long = 轫致辐射,
  pdfstring = 轫致辐射,
  class = glossary,
}

\DeclareAcronym{bw-smear}{
  short = bandwidth smearing,
  long = 带宽涂污,
  class = glossary,
}

\DeclareAcronym{cctor}{
  short = complex correlator,
  long = 复相关器,
  class = glossary,
}

\DeclareAcronym{cd}{
  short = Cosmic Dawn,
  long = 宇宙黎明,
  class = glossary,
}

\DeclareAcronym{cdae}{
  short = CDAE,
  long = 卷积去噪自编码器,
  foreign = convolutional denoising autoencoder,
  class = glossary,
}

\DeclareAcronym{cdm}{
  short = CDM,
  long = 冷暗物质,
  foreign = Cold Dark Matter,
  class = glossary,
}

\DeclareAcronym{cern}{
  short = CERN,
  long = European Organization for Nuclear Research,
  foreign = Conseil européen pour la recherche nucléaire,
  class = glossary,
}

\DeclareAcronym{cgs}{
  short = CGS,
  long = centimeter--gram--second,
  class = glossary,
}

\DeclareAcronym{cmb}{
  short = CMB,
  long = 宇宙微波背景,
  foreign = Cosmic Microwave Background,
  class = glossary,
}

\DeclareAcronym{cnn}{
  short = CNN,
  long = 卷积神经网络,
  foreign = convolutional neural network,
  class = glossary,
}

\DeclareAcronym{conv-theorem}{
  short = convolution theorem,
  long = 卷积定理,
  class = glossary,
}

\DeclareAcronym{ctor}{
  short = correlator,
  long = 相关器,
  class = glossary,
}

\DeclareAcronym{da}{
  short = Dark Ages,
  long = 黑暗时期,
  class = glossary,
}

\DeclareAcronym{dae}{
  short = DAE,
  long = 去噪自编码器,
  foreign = denoising autoencoder,
  class = glossary,
}

\DeclareAcronym{dare}{
  short = DARE,
  long = Dark Ages Radio Explorer,
  class = glossary,
}

\DeclareAcronym{delay-center}{
  short = delay center,
  long = 延迟中心,
  class = glossary,
}

\DeclareAcronym{dc}{
  short = direction cosine,
  long = 方向余弦,
  class = glossary,
}

\DeclareAcronym{deconv}{
  short = deconvolution,
  long = 解卷积,
  class = glossary,
}

\DeclareAcronym{dirty-map}{
  short = dirty map,
  long = 脏图,
  class = glossary,
}

\DeclareAcronym{dl}{
  short = deep learning,
  long = 深度学习,
  class = glossary,
}

\DeclareAcronym{ds}{
  short = delay spectrum,
  long = 延迟谱,
  class = glossary,
}

\DeclareAcronym{e2e}{
  short = end-to-end,
  long = 端对端,
  class = glossary,
}

\DeclareAcronym{edges}{
  short = EDGES,
  long = Experiment to Detect the Global EoR Signature,
  class = glossary,
}

\DeclareAcronym{eor}{
  short = EoR,
  long = 再电离时期,
  foreign = Epoch of Reionization,
  class = glossary,
}

\DeclareAcronym{eor-window}{
  short = EoR window,
  long = 再电离窗口,
  class = glossary,
}

\DeclareAcronym{exosphere}{
  short = exosphere,
  long = 散逸层,
  class = glossary,
}

\DeclareAcronym{extsrc}{
  short = extended source,
  long = 展源,
  pdfstring = 展源,
  class = glossary,
}

\DeclareAcronym{fast}{
  short = FAST,
  long = Five-hundred-meter Aperture Spherical radio Telescope,
  class = glossary,
}

\DeclareAcronym{fft}{
  short = FFT,
  long = 快速 Fourier 变换,
  foreign = Fast Fourier Transform,
  class = glossary,
}

\DeclareAcronym{fftt}{
  short = FFTT,
  long = 快速 Fourier 变换望远镜,
  foreign = Fast Fourier Transform Telescope,
  class = glossary,
}

\DeclareAcronym{fgavd}{
  short = foreground avoidance,
  long = 前景回避法,
  pdfstring = 前景回避法,
  class = glossary,
}

\DeclareAcronym{fgrm}{
  short = foreground removal,
  long = 前景扣除法,
  pdfstring = 前景扣除法,
  class = glossary,
}

\DeclareAcronym{fm}{
  short = FM,
  long = 调频,
  foreign = frequency modulation,
  class = glossary,
}

\DeclareAcronym{fov}{
  short = Fo\!V,
  long = 视场,
  foreign = field of view,
  class = glossary,
}

\DeclareAcronym{fringe}{
  short = fringe,
  long = 条纹,
  class = glossary,
}

\DeclareAcronym{g-units}{
  short = Gaussian units,
  long = 高斯单位制,
  class = glossary,
}

\DeclareAcronym{gain}{
  short = gain,
  long = 增益,
  class = glossary,
}

\DeclareAcronym{gbt}{
  short = GBT,
  long = Green Bank Telescope,
  class = glossary,
}

\DeclareAcronym{gc}{
  short = galaxy cluster,
  short-plural = galaxy clusters,
  long = 星系团,
  pdfstring = 星系团,
  class = glossary,
}

\DeclareAcronym{gleam}{
  short = GLEAM,
  long = GaLactic and Extragalactic All-sky MWA,
  class = glossary,
}

\DeclareAcronym{gleam-x}{
  short = GLEAM-X,
  long = GaLactic and Extragalactic All-sky MWA eXtended,
  class = glossary,
}

\DeclareAcronym{gmrt}{
  short = GMRT,
  long = Giant Metrewave Radio Telescope,
  class = glossary,
}

\DeclareAcronym{gps}{
  short = GPS,
  long = 全球卫星定位系统,
  foreign = Global Positioning System,
  class = glossary,
}

\DeclareAcronym{gpu}{
  short = GPU,
  long = 图形处理器,
  foreign = graphics processing unit,
  class = glossary,
}

\DeclareAcronym{hera}{
  short = HERA,
  long = Hydrogen Epoch of Reionization Array,
  class = glossary,
}

\DeclareAcronym{hi}{
  short = H\textsc{i},
  long = 中性氢,
  pdfstring = 中性氢,
  foreign = neutral hydrogen,
  class = glossary,
}

\DeclareAcronym{icm}{
  short = ICM,
  long = 星系团内介质,
  foreign = intracluster medium,
  class = glossary,
}

\DeclareAcronym{icrar}{
  short = ICRAR,
  long = International Centre for Radio Astronomy Research,
  class = glossary,
}

\DeclareAcronym{igm}{
  short = IGM,
  long = 星系际介质,
  foreign = intergalactic medium,
  class = glossary,
}

\DeclareAcronym{inflation}{
  short = inflation,
  long = 暴胀,
  class = glossary,
}

\DeclareAcronym{ionosphere}{
  short = ionosphere,
  long = 电离层,
  class = glossary,
}

\DeclareAcronym{leda}{
  short = LEDA,
  long = Large-aperture Experiment to Detect the Dark Ages,
  class = glossary,
}

\DeclareAcronym{lofar}{
  short = LOFAR,
  long = LOw Frequency ARray,
  class = glossary,
}

\DeclareAcronym{lotss}{
  short = LoTSS,
  long = LOFAR Two-metre Sky Survey,
  class = glossary,
}

\DeclareAcronym{lsf}{
  short = large-scale filaments,
  long = 大尺度纤维状结构,
  pdfstring = 大尺度纤维状结构,
  class = glossary,
}

\DeclareAcronym{lwa}{
  short = LWA,
  long = Long Wavelength Array,
  class = glossary,
}

\DeclareAcronym{magnetosphere}{
  short = magnetosphere,
  long = 磁层,
  class = glossary,
}

\DeclareAcronym{mcf}{
  short = mutual coherence function,
  long = 互相干函数,
  class = glossary,
}

\DeclareAcronym{mem}{
  short = MEM,
  long = 最大熵方法,
  foreign = maximum entropy method,
  class = glossary,
}

\DeclareAcronym{mesosphere}{
  short = mesosphere,
  long = 中间层,
  class = glossary,
}

\DeclareAcronym{mfp}{
  short = mean free path,
  long = 平均自由程,
  class = glossary,
}

\DeclareAcronym{miteor}{
  short = MITEoR,
  long = MIT Epoch of Reionization,
  class = glossary,
}

\DeclareAcronym{mwa}{
  short = MWA,
  long = Murchison Widefield Array,
  class = glossary,
}

\DeclareAcronym{nao}{
  short = NAO,
  long = 国家天文台,
  foreign = National Astronomical Observatories,
  class = glossary,
}

\DeclareAcronym{nasa}{
  short = NASA,
  long = National Aeronautics and Space Administration,
  class = glossary,
}

\DeclareAcronym{nfw}{
  short = NFW,
  long = Navarro--Frenk--White,
  class = glossary,
}

\DeclareAcronym{od}{
  short = overdensity,
  long = 过密度,
  class = glossary,
}

\DeclareAcronym{paper}{
  short = PAPER,
  long = Precision Array for Probing the Epoch of Reionization,
  class = glossary,
}

\DeclareAcronym{pa}{
  short = phased array,
  long = 相控阵,
  class = glossary,
}

\DeclareAcronym{passband}{
  short = passband,
  long = 通带,
  class = glossary,
}

\DeclareAcronym{pb}{
  short = primary beam,
  long = 初级波束,
  class = glossary,
}

\DeclareAcronym{phase-refpos}{
  short = phase reference position,
  long = 相位参考位置,
  class = glossary,
}

\DeclareAcronym{pl}{
  short = polarization leakage,
  long = 偏振泄漏,
  class = glossary,
}

\DeclareAcronym{pp}{
  short = power pattern,
  long = 功率方向图,
  class = glossary,
}

\DeclareAcronym{propagator}{
  short = propagator,
  long = 传播子,
  class = glossary,
}

\DeclareAcronym{ps}{
  short = power spectrum,
  short-plural = power spectra,
  long = 功率谱,
  class = glossary,
}

\DeclareAcronym{psf}{
  short = PSF,
  long = 点扩散函数,
  foreign = point spread function,
  class = glossary,
}

\DeclareAcronym{pntsrc}{
  short = point source,
  long = 点源,
  pdfstring = 点源,
  class = glossary,
}

\DeclareAcronym{recomb}{
  short = recombination,
  long = 复合,
  class = glossary,
}

\DeclareAcronym{reion}{
  short = reionization,
  long = 再电离,
  class = glossary,
}

\DeclareAcronym{rfi}{
  short = radio frequency interference,
  long = 射频干扰,
  class = glossary,
}

\DeclareAcronym{rh}{
  short = radio halo,
  long = 射电晕,
  class = glossary,
}

\DeclareAcronym{rmh}{
  short = radio mini-halo,
  long = 迷你射电晕,
  class = glossary,
}

\DeclareAcronym{rms}{
  short = {r.m.s\@.},
  long = 方均根,
  foreign = root mean square,
  class = glossary,
}

\DeclareAcronym{rr}{
  short = radio relic,
  long = 射电遗迹,
  class = glossary,
}

\DeclareAcronym{saras}{
  short = SARAS,
  long = Shaped Antenna measurement of the background RAdio Spectrum,
  class = glossary,
}

\DeclareAcronym{sb}{
  short = synthesized beam,
  long = 综合波束,
  class = glossary,
}

\DeclareAcronym{sc}{
  short = supercluster,
  long = 超星系团,
  pdfstring = 超星系团,
  class = glossary,
}

\DeclareAcronym{sci-hi}{
  short = SCI-HI,
  long = Sonda Cosmológica de las Islas para la Detección de Hidrógeno Neutro,
  class = glossary,
}

\DeclareAcronym{sf}{
  short = sampling function,
  long = 采样函数,
  class = glossary,
}

\DeclareAcronym{ska}{
  short = SKA,
  long = Square Kilometre Array,
  class = glossary,
}

\DeclareAcronym{ska1low}{
  short = SKA1-Low,
  long = SKA 一期低频阵列,
  class = glossary,
}

\DeclareAcronym{skymap}{
  short = sky map,
  long = 天图,
  class = glossary,
}

\DeclareAcronym{station}{
  short = station,
  long = 站点,
  class = glossary,
}

\DeclareAcronym{stb}{
  short = station beam,
  long = 站点波束,
  class = glossary,
}

\DeclareAcronym{synrad}{
  short = synchrotron radiation,
  long = 同步辐射,
  pdfstring = 同步辐射,
  class = glossary,
}

\DeclareAcronym{t-smear}{
  short = time smearing,
  long = 时间涂污,
  class = glossary,
}

\DeclareAcronym{tb}{
  short = brightness temperature,
  long = 亮温度,
  class = glossary,
}

\DeclareAcronym{thermosphere}{
  short = thermosphere,
  long = 热层,
  class = glossary,
}

\DeclareAcronym{vis}{
  short = visibility,
  long = 可见度,
  pdfstring = 可见度,
  class = glossary,
}

\DeclareAcronym{vla}{
  short = VLA,
  long = 甚大阵射电望远镜,
  foreign = Very Large Array,
  class = glossary,
}

\DeclareAcronym{w-proj}{
  short = $w$-projection,
  long = $w$ 投影,
  class = glossary,
}

\DeclareAcronym{w-stack}{
  short = $w$-stacking,
  long = $w$ 堆叠,
  class = glossary,
}

\DeclareAcronym{wsrt}{
  short = WSRT,
  long = Westerbork Synthesis Radio Telescope,
  class = glossary,
}


\endinput


% Help track changes
% Credit: https://tex.stackexchange.com/a/49913
\newcommand{\editone}[1]{{\leavevmode\color{cyan}#1}}
\newcommand{\edittwo}[1]{{\leavevmode\color{magenta}#1}}
%\renewcommand{\editone}[1]{{#1}}
%\renewcommand{\edittwo}[1]{{#1}}

\AtBeginBibliography{
  \linespread{1.1}
  \small
}
\addbibresource{references.bib}


%=====================================================================

% 不得超过 36 字
\title{%
  射电晕对宇宙再电离探测的影响和\texorpdfstring{\\}{}%
  基于深度学习的再电离信号分离新算法%
}
\keywords{% 4-6 个
  低频射电天文,
  宇宙再电离时期,
  射电晕,
  弱信号分离,
  深度学习,
  卷积去噪自编码器
}
\author{李维天}
\studentnumber{0130729026}
\advisor{徐海光~教授}
\school{上海交通大学}
\institute{物理与天文学院}
\major{物理学}
\defenddate{2018 年 mm 月 dd 日}

\englishtitle{%
  Impacts of Radio Halos on EoR Detection and \\
  A Novel Deep-learning-based Method to \\
  Separate the EoR Signal
}
\englishkeywords{%
  low-frequency radio astronomy,
  epoch of reionization,
  radio halos,
  weak signal separation,
  deep learning,
  convolutional denoising autoencoder
}
\englishauthor{\textbf{Wéitiān Lǐ}}
\englishadvisor{Prof. \textbf{Hǎiguāng Xú}}
\englishinstitute{School of Physics and Astronomy}
\englishschool{Shanghai Jiao Tong University}
\englishlocation{Shanghai, China}
\englishmajor{Physics}
\englishdate{mmm dd, 2018}

%---------------------------------------------------------------------
% Copyright @ last page
%
% Date format: yyyy.mm.dd
\newcommand*{\twodigits}[1]{\ifnum#1<10 0\fi\the#1}
\renewcommand*{\today}{%
  \leavevmode\hbox{\the\year.\twodigits\month.\twodigits\day}
}
\fancypagestyle{lastpage}{
  \fancyhf{}
  \renewcommand{\headrulewidth}{0pt}
  \fancyfoot[L]{\color{gray}%
    \faCopyright{} 2018 Wéitiān Lǐ,
    \href{http://creativecommons.org/licenses/BY/4.0/}{%
      \color{gray}\faCreativeCommonsBy{} BY 4.0},
    \href{https://github.com/liweitianux/phd-thesis}{%
      \color{gray}\faGithub{} liweitianux/phd-thesis},
    \faPencilAlt{} \today
  }
}
\AtEndDocument{%
  \label{doc:lastpage}
  %\ifodd\pageref{doc:lastpage}
    \newpage\mbox{}
    \thispagestyle{lastpage}
  %\fi
}


%=====================================================================
\begin{document}

\maketitle

\makeatletter
\ifsjtu@submit\relax
  \includepdf{scans/originality.pdf}
  \pdfbookmark[0]{\sjtu@label@originality}{originality}
  \cleardoublepage
  \includepdf{scans/authorization.pdf}
  \pdfbookmark[0]{\sjtu@label@authorization}{authorization}
  \cleardoublepage
\else
\ifsjtu@review\relax
% exclude the originality and authorization declarations
\else
  \makeDeclareOriginality
  \makeDeclareAuthorization
\fi
\fi
\makeatother

%---------------------------------------------------------------------
\frontmatter

% 中文摘要,约 3000 字
\begin{abstract}
TODO
\end{abstract}

%---------------------------------------------------------------------

\begin{englishabstract}
TODO
\end{englishabstract}


\tableofcontents
\listoffigures
\addcontentsline{toc}{chapter}{\listfigurename}
\listoftables
\addcontentsline{toc}{chapter}{\listtablename}
% \listofalgorithms
% \addcontentsline{toc}{chapter}{\listalgorithmname}

\printacronyms[
  include-classes=symbol,
  name={主要符号对照表},
]

%---------------------------------------------------------------------
\mainmatter
\pagestyle{main}

%%
%% Copyright (c) 2018-2019 Weitian LI <liweitianux@sjtu.edu.cn>
%% Creative Commons BY 4.0
%%

\chapter{绪论}
\label{chap:introduction}

%=====================================================================
\section{研究背景}
\label{sec:background}

理解宇宙的结构、起源和演化,是人类孜孜不倦地追求的目标,在哲学和科学中占据重要地位.
经过无数人的努力,宇宙学的\ac{bbt}终于得以建立.
该理论已被大量观测证据所支持,比如星系的红移--距离关系(即 Hubble 定律)、
\ac{cmb}辐射、星系的大尺度分布、早期元素丰度、等等,
是目前宇宙学的标准模型.

根据大爆炸宇宙学模型,宇宙起源于约 138 亿年前的一次大爆炸,然后随着宇宙的膨胀,
温度以及能量密度都逐渐降低,宇宙主要经历了\ac{inflation}、\ac{bbn}、
\ac{recomb}、\ac{da}、形成第一代天体、\ac{reion}、形成星系及大尺度结构
等阶段,如\autoref{fig:univ-history} 所示.

\begin{figure}[!htp]
  \centering
  \includegraphics[width=\textwidth]{universe-history}
  \bicaption[宇宙的演化历史]{%
    宇宙从大爆炸到今天的演化历史.
  }{%
    The evolution of the Universe from the Big Bang
    to the present.
    \\\textcopyright{}
    \acuse{bicep,cern,nasa}
    \acs{bicep}2/\acs{cern}/\acs{nasa}; CC0 1.0.
  }
  \label{fig:univ-history}
\end{figure}

大爆炸之后约 40 万年,宇宙已冷却至大约 \SI{3000}{\kelvin},
于是自由电子被结合到中性原子之中,与重子物质脱耦的光子开始在宇宙中自由传播,
形成弥漫于整个宇宙的背景辐射,即今天所探测到的 \ac{cmb} 辐射.
但是,此时尚未形成发光的天体,因此宇宙进入了\acl{da}.
随着物质的密度扰动在引力作用下增长,第一代天体开始形成并产生辐射,使得重子物质
再次被逐步电离,宇宙从此结束\acl{da}并走入\ac{eor}.
随着各尺度上的天体结构的逐步形成与演化,重子物质被充分电离,宇宙也演化形成今天的格局.

\begin{figure}[!htp]
  \centering
  \includegraphics[width=\textwidth]{cosmic-stages-dare}
  \bicaption[宇宙的黑暗时期与再电离时期示意图]{%
    宇宙的\acl{da}与\acl{eor}示意图,其中显示了\acl{aoi} (A)、\acl{da} (B)、
    \acl{cd} (C) 以及\acl{eor} (D, E).
    上方的粗曲线显示了理论预测的 \hisignal/的强度.
  }{%
    A schematic showing the \acs{da} and the \acs{eor}
    of the Universe, mainly including the \acs{aoi} (A),
    the \acs{da} (B), the \acs{cd} (C), and the \acs{eor} (D, E).
    The thick curve in the top panel shows the predicted intensity
    of the 21\,cm signal.
    \\\textcopyright{}
    \acuse{dare}\ac{dare},
    \url{http://lunar.colorado.edu/dare/science.html}, (2018-09-23).
  }
  \label{fig:cosmic-stages}
\end{figure}

我们已借助多波段观测掌握了大量有关宇宙近期演化
($\acs{z} \lesssim 6$;宇宙已充分电离之后)的信息;
通过研究 \ac{cmb},我们对宇宙的早期历史
($z \gtrsim 1100$;自由电子\acl{recomb}之前)有了深刻理解.
然而,我们对中间的那段时期($z \sim \numrange{6}{1100}$)却知之甚少.
这段时期可细分为以下四个阶段\cite{koopmans2015}:
\acl{aoi} (\acs{aoi}; $z \sim \numrange{200}{1100}$)、
\acl{da} ($z \sim \numrange{30}{200}$)、
\acl{cd} (\acs{cd}; $z \sim \numrange{15}{30}$)
以及\acl{eor} ($z \sim \numrange{6}{15}$),
如\autoref{fig:cosmic-stages} 所示.
对于其中距离我们相对较近的\acl{eor},
我们目前仅获得非常有限的间接观测信息,比如:
该时期的\ac{hi}对高红移类星体的 Ly$\alpha$ 吸收 \cite{becker2001}、
该时期的自由电子对 \ac{cmb} 光子的 Thomson 散射 \cite{kaplinghat2003}.
但是,我们仍然缺乏来自\acl{eor}的直接观测证据,
对该时期的基本性质和关键物理过程仍不清楚,比如:
第一代天体是何时以及如何形成的?
主要的电离源有哪些以及它们是如何影响再电离过程的?
电离氢区的尺度以及演化过程如何?
研究\acl{eor}的对于理解宇宙早期结构形成以及星系的形成与演化有重要意义,
是建立完整的宇宙演化图景的关键环节之一.
具体请参见 \citeay{fan2006}, \citeay{morales2010},
\citeay{pritchard2012}, \citeay{zaroubi2013},
\citeay{koopmans2015} 等综述文.

在\acl{eor}及其之前的\acl{da},尽管缺乏发光天体可供观测,
但是宇宙中丰富的\acl{hi}所辐射的 21\,cm 谱线
(以下简称 \emph{\hisignal/};
详见 \autoref{sec:21cm-signal})为探测该时期提供了有效途径.
对 \hisignal/的探测是目前对\acl{eor}及其之前的\acl{da}开展系统性
研究的最直接而有效的观测手段 \cite{koopmans2015,furlanetto2016}.

\acl{hi}的 21\,cm 谱线的本征频率约为 \SI{1420}{\MHz}.
源自\acl{eor}的 \hisignal/(以下简称 \emph{EoR 信号})经历显著红移后
应出现在约 \SIrange{90}{200}{\MHz},对应低频射电波段.
EoR 信号到达地球时已非常微弱,仅约几 \si{\mK} 至十几 \si{\mK},
因此需要具有极高灵敏度的低频观测设备才能捕获该信号.
目前的主流技术是采用大规模低频干涉阵列,已建成或正在建设的干涉阵列主要有:
\ac{21cma} \cite{zheng2016}、
\ac{gmrt} \cite{paciga2011}、
\ac{mwa} \cite{bowman2013,tingay2013}、
\ac{lofar} \cite{vanHaarlem2013}、
\ac{lwa} \cite{ellingson2009}、
\ac{paper} \cite{parsons2010}、
\ac{hera} \cite{deboer2017}、
\ac{ska} \cite{mellema2013,koopmans2015}.
然而,利用干涉阵列探测 EoR 信号仍面临诸多困难与挑战\cite{wijnholds2010},
其中主要包括:
识别并扣除强烈的前景干扰、扣除人工源的\ac{rfi}、修正电离层的扰动、
苛刻的仪器校准要求、海量数据处理和高动态范围成像.

在低频射电波段,强烈的前景干扰(主要源自银河系以及河外点源;
详见 \autoref{sec:fg-intro})比待探测的 EoR 信号高出约 5 个数量级;
即便按干涉阵列所测量的天空亮度涨落来衡量,前景干扰的涨落也是待测信号的数千倍
\cite{zaroubi2013}.
如何准确把握前景干扰并将其有效扣除,是成功探测 EoR 信号的关键.
由于低频射电观测和巡天数据的严重不足,我们对该波段的前景的了解非常有限,
无法达到探测 EoR 信号所要求的精度.
因此,我们需要挖掘已有海量的中高频射电观测以及其他多波段观测数据,
并结合逐渐增长的低频观测数据,深入理解低频射电前景辐射,构建并完善前景模型,
为识别并扣除前景干扰提供有力支撑.

虽然在本质上,前景辐射的频谱是光滑的,而 EoR 信号的频谱呈锯齿状,
两者具有很好的可区分性 \cite{wang2006,jelic2008,harker2009,wang2013}.
然而在实际情况中,受到干涉阵列的复杂仪器效应、观测干扰、数据处理技术的限制
等因素的影响,前景频谱的光滑性遭到破坏,导致 EoR 信号的提取变得尤其困难
\cite{liu2009ps,labropoulos2009,gehlot2018,mertens2018}.
如何研发出行之有效的前景处理和 EoR 信号提取算法,亦是当前的重要研究课题.


%=====================================================================
\section{研究内容}
\label{sec:content}

本文的研究内容分为以下两部分:
\begin{itemize}
\item
\emph{改进低频射电天空的模拟:}
深刻理解各前景成分的性质(如强度、空间分布、频谱结构)并充分把握它们对 EoR 探测
的干扰方式,是研发具有针对性的前景去除和 EoR 信号分离算法的前提与关键 (ref???).
由于复杂的仪器效应和严重的观测干扰,低频干涉阵列的系统校准非常困难
\cite{noordam2004,intema2009,wijnholds2010,barry2016,gehlot2018},
严重制约仪器达到探测 EoR 信号所要求的极高灵敏度.
在现阶段缺乏足够可用的高质量低频射电观测数据的情况下,挖掘已有多波段观测数据并
准确模拟低频射电天空,是开展前景干扰研究以及 EoR 信号分离算法研发的可行办法.

\hspace{2\ccwd}%
在诸多前景成分之中,银河系的弥散辐射 [包括\ac{synrad}和\ac{brad}]
以及河外\ac{pntsrc}辐射是最主要的成分,目前已被广泛地研究和较好地理解
\cite{shaver1999,diMatteo2004,gleser2008,liu2012,murray2017,spinelli2018}.
除此之外,剩下的前景辐射主要来自河外\ac{extsrc},其中包括:
\ac{icm} \cite{feretti2012} 产生的\ac{rh}、\ac{rr}和\ac{rmh}、
\ac{gc}之外的\ac{igm} \cite{keshet2004}、
以及\ac{lsf} \cite{vazza2015}.
对于这些河外\acl{extsrc},已获得的观测证据不多,在低频射电波段更是不足.
关于它们将具体如何影响 EoR 探测,目前的理解非常有限,亟待深入且系统的研究.

\hspace{2\ccwd}%
与其他几类河外\acl{extsrc}相比,\acl{rh}拥有更多的观测证据和理论研究,支撑我们
构建一个更佳的模型用来模拟\acl{rh}的低频射电辐射,改进低频射电天空的模拟,
进而在考虑干涉阵列的实际仪器效应的情况下,有效地评估\acl{rh}对 EoR 探测的影响.

\item
\emph{研发 EoR 信号分离新算法:}
为了提取淹没于前景干扰中的 EoR 信号,一系列方法已被提出来用于处理前景
(详见 \autoref{sec:fg-methods}).
这些前景处理方法可大致分为\ac{fgrm}和\ac{fgavd}两大类,
但都依赖于一个重要前提:前景辐射的频谱必须非常光滑.
据此,这些方法通过构建一个模型来拟合光滑的前景成分并扣除,或者在功率谱空间尽量
避开前景污染区域,从而提取出微弱的 EoR 信号 \cite{chapman2016}.

\hspace{2\ccwd}%
然而在实际情况中,干涉阵列的\ac{beam}存在频率依赖效应(以下简称\emph{波束效应}),
即\acl{beam}的形状随观测频率而变化,因此 CLEAN 后残留的前景源会产生沿频率方向
快速变化的涨落,严重破坏前景频谱的光滑性 \cite{liu2009ps},
导致现有方法无法有效分辨前景干扰与 EoR 信号并分离出 EoR 信号
(详见 \autoref{sec:fdeffect}).

\hspace{2\ccwd}%
考虑到干涉列阵的\acl{beam}的形状非常复杂,为现有方法打造一个实际可用的模型
用以克服上述复杂的波束效应将很困难 \cite{lochner2015},
因此研发基于\ac{dl}的 EoR 信号分离新算法是一条更加可行且具有吸引力的途径
\cite{herbel2018,vafaeiSadr2019},
通过从数据中学习知识并自适应地优化模型,达到克服波束效应并分离 EoR 信号的目标.

\end{itemize}

本文的研究目标是:
(1)改进\acl{rh}的模拟,考虑干涉阵列的实际仪器效应,
获得更精细、更符合实际的低频射电天空的模拟图像,
进而有效地评估\acl{rh}对 EoR 探测的影响.
(2)研发基于\acl{dl}的能够有效克服干涉阵列的波束效应的 EoR 信号分离新算法,
并运用到上述模拟数据进行测试和优化.


%=====================================================================
\section{研究方案}
\label{sec:plan}

本文遵循以下主要步骤开展开展工作,完成研究内容,达到研究目标:
\begin{enumerate}
\item
调研\acl{rh}的相关理论研究和观测证据,理解其形成机制和演化规律,
构建模型并编程实现模拟\acl{rh}的低频射电辐射.
搜集\acl{rh}的现有观测数据,调节模型的参数,获得可靠的模拟结果.

\item
采用典型的干涉阵列(如 SKA1-Low)的布局方案,对上一步所得的\ac{skymap}
开展模拟观测,得到\ac{vis}数据,再利用 CLEAN 算法成像获得相应的“观测”图像.
通过这种\ac{e2e}模拟,干涉阵列的复杂仪器效应(如本文关注的波束效应)得以
有效地整合到研究流程之中.

\item
基于上述模拟所得的“观测”图像,利用一维和二维\ac{ps},对比\acl{rh}和
EoR 信号的异同,量化\acl{rh}在运用\acl{fgrm}或\acl{fgavd}的情况下
对 EoR 探测的影响,有效评估\acl{rh}作为前景干扰成分的重要程度.

\item
对比分析目前的主流\acl{dl}方法,筛选出合适的算法并加以必要的改进,
适用到此处的 EoR 信号分离场景.
利用已有模拟数据对算法进行训练和调优,挑选出满足要求的最佳算法.

\end{enumerate}


%=====================================================================
\section{本文框架}
\label{sec:structure}

本文余下章节安排如下:
\autoref{chap:interferometry}将介绍射电天文学和射电干涉技术的基础知识,
包括基本辐射理论、天线原理、干涉阵列及综合孔径成像等.
在\autoref{chap:detection},我们将介绍利用\acl{hi}
\hisignal/探测宇宙再电离时期的方法和困难、以及前景处理方法.
在\autoref{chap:simulation},我们首先模拟各前景成分和 EoR 信号的\acl{skymap},
然后进行干涉阵列的模拟观测,得到整合了实际仪器效应的观测图像.
据此,我们在\autoref{chap:halo}借助\acl{ps}量化评估\acl{rh}对
EoR 探测的具体影响.
\autoref{chap:cdae}将阐述我们提出的基于\acl{dl}的 EoR 分离新算法并演示其效果.
最后,我们对全文进行总结并简要展望.

全文采用一个由 \lcdm/ 模型描述的平直宇宙,具体参数为:
$\acs{H0} = 100\,\acs{h} = \SI{71}{\km\per\second\per\Mpc}$、
$\acs{Om0} = 0.27$、
$\acs{Ol0} = 1 - \acs{Om0} = 0.73$、
$\acs{Ob0} = 0.046$、
$\acs{ns} = 0.96$ 以及 $\acs{sigma8} = 0.81$.
如无额外说明,本文给出的误差对应 \SI{68}{\percent} 的置信水平;
使用的幂律谱形式为 $\acs{S-nu} \propto \acs{freq}^{-\acs{sidx}}$,
其中 \acs{S-nu} 为\acl{S-nu}、\acs{sidx} 为\acl{sidx}.
本文使用的中文术语遵循《英汉天文学名词数据库》
\footnote{英汉天文学名词数据库:\url{http://astrodict.china-vo.org/}}.


%% EOF

\chapter{射电干涉技术基础}
\label{chap:interferometry}

%=====================================================================
\section{射电天文学简介}
\label{sec:radio-astronomy}

TODO

%---------------------------------------------------------------------
\subsection{射电天文学是什么?}

TODO

%---------------------------------------------------------------------
\subsection{射电窗口}

TODO

%---------------------------------------------------------------------
\subsection{机遇和挑战}

TODO


%=====================================================================
\section{辐射理论}
\label{sec:radiation}

TODO

%---------------------------------------------------------------------
\subsection{亮度和流量密度}

TODO

%---------------------------------------------------------------------
\subsection{黑体辐射和亮温度}

TODO


%=====================================================================
\section{天线原理}
\label{sec:antenna}

TODO

%---------------------------------------------------------------------
\subsection{辐射方向图}

TODO

%---------------------------------------------------------------------
\subsection{增益和阻抗}

TODO

%---------------------------------------------------------------------
\subsection{主瓣和旁瓣}

TODO

ERA: eq:(3.96,3.118)

%---------------------------------------------------------------------
\subsection{有效面积}

TODO

%---------------------------------------------------------------------
\subsection{互易定理}

(???) TODO

%---------------------------------------------------------------------
\subsection{天线温度}

TODO


%=====================================================================
\section{干涉仪和综合孔径}
\label{sec:interferometer}

TODO

%---------------------------------------------------------------------
\subsection{基本原理}

TODO

二元干涉仪

%---------------------------------------------------------------------
\subsection{综合孔径}

TODO

坐标系统, 优点和缺点

%---------------------------------------------------------------------
\subsection{可视度}

TODO

%---------------------------------------------------------------------
\subsection{$uv$ 覆盖}

TODO

integration time, earth rotation, phase tracking

%---------------------------------------------------------------------
\subsection{三个波束}

antenna beam, (??? primary beam), station beam, sythesized beam (PSF)

%---------------------------------------------------------------------
\subsection{三个中心}

phase center, pointing center, delay center

phase tracking, drift scan

%---------------------------------------------------------------------
\subsection{灵敏度}

point-source sensitivity, brightness sensitivity

ERA: 3.6.3.2:confusion

%---------------------------------------------------------------------
\subsection{数字波束合成}

multi-beam, phased array, 优点和缺点

%---------------------------------------------------------------------
\subsection{脏图}

weighting (natural, uniform, robust/Briggs)

%---------------------------------------------------------------------
\subsection{CLEAN 算法}

TODO

%---------------------------------------------------------------------
\subsection{大视场成像}

$w$-term, $w$-projection, $w$-stacking


%=====================================================================
\section{低频干涉阵列}
\label{sec:instruments}

TODO

%---------------------------------------------------------------------
\subsection{21CMA}

TODO

%---------------------------------------------------------------------
\subsection{LOFAR}

TODO

%---------------------------------------------------------------------
\subsection{MWA}

TODO

%---------------------------------------------------------------------
\subsection{SKA}

TODO

%---------------------------------------------------------------------
\subsection{HERA}

TODO

%---------------------------------------------------------------------
\subsection{MITEoR}

TODO

%---------------------------------------------------------------------
\subsection{LWA}

TODO


%=====================================================================
\section{小结}

TODO


%% EOF

%%
%% Copyright (c) 2018 Weitian LI <liweitianux@sjtu.edu.cn>
%% Creative Commons BY 4.0
%%

\chapter{宇宙再电离时期的探测}
\label{chap:detection}


%=====================================================================
\section{早期宇宙}
\label{sec:early-universe}

TODO


%=====================================================================
\section{中性氢 \SI{21}{\cm} 信号}
\label{sec:21cm-signal}

TODO


%=====================================================================
\section{探测方法}

TODO

%---------------------------------------------------------------------
\subsection{直接成像法}

TODO

%---------------------------------------------------------------------
\subsection{功率谱测量法}

TODO

%---------------------------------------------------------------------
\subsection{主要困难}

cosmological foreground contamination,
instrumental effects, ionopsheric distortion,
radio frequency interference (RFI),
calibration imperfection, etc.


%=====================================================================
\section{主要前景成分}

TODO

%---------------------------------------------------------------------
\subsection{银河系\acl*{synrad}}  % 同步辐射

TODO

%---------------------------------------------------------------------
\subsection{银河系\acl*{ffrad}}  % 自由-自由辐射

TODO

%---------------------------------------------------------------------
\subsection{河外\acl*{pntsrc}}  % 河外点源

TODO

%---------------------------------------------------------------------
\subsection{河外\acl{extsrc}}  % 河外展源

galaxy clusters (halos, relics, mini-halos),
intergalactic medium (virial shocks),
cosmic filaments, etc.

星系团射电晕


%=====================================================================
\section{前景处理方法}

key characteristic: frequency structure difference between
the 21~cm signal and foreground emission.

%---------------------------------------------------------------------
\subsection{\acl*{fgrm}}  % 前景扣除法

TODO

%---------------------------------------------------------------------
\subsection{\acl*{fgav}}  % 前景回避法

2D power spectrum, EoR window


%=====================================================================
\section{\acl*{eor-window}}  % 再电离窗口
\label{sec:eor-window}

EoR window, foreground wedge, explanation ...


%=====================================================================
\section{小结}

TODO


%% EOF

%%
%% Copyright (c) 2018-2019 Weitian LI <liweitianux@sjtu.edu.cn>
%% Creative Commons BY 4.0
%%

\chapter{低频射电天空的模拟}
\label{chap:simulation}

%=====================================================================
\section{星系团射电晕}
\label{sec:radio-halos}

theoretical studies and models:
turbulent re-acceleration model,
hadronic model (secondary electron models)

FG21sim, ...

Turbulence; Alfven, slow, fast modes; particle acceleration:
Lazarian et al. 2012, SSRv;
Petrosian 2012, SSRv.

%---------------------------------------------------------------------
\subsection{质量函数}

TODO

%---------------------------------------------------------------------
\subsection{并合历史}

TODO

%---------------------------------------------------------------------
\subsection{射电晕的形成与演化}

TODO

%.....................................................................
\subsubsection{热成分性质}

TODO

%.....................................................................
\subsubsection{电子注入过程}

TODO

%.....................................................................
\subsubsection{初始电子能谱}

TODO

%.....................................................................
\subsubsection{湍流加速机制}

TODO

%.....................................................................
\subsubsection{湍流加速时间段}

TODO

%.....................................................................
\subsubsection{能量损失机制}

TODO

%.....................................................................
\subsubsection{数值算法}

TODO

%.....................................................................
\subsubsection{图像生成}

TODO

%.....................................................................
\subsubsection{参数调节}

TODO


%=====================================================================
\section{银河系}

TODO

%---------------------------------------------------------------------
\subsection{同步辐射}

TODO

%---------------------------------------------------------------------
\subsection{轫致辐射}

\acl{brad}
TODO


%=====================================================================
\section{河外点源}

TODO


%=====================================================================
\section{再电离信号}

TODO


%=====================================================================
\section{干涉阵列的模拟观测}

TODO

%---------------------------------------------------------------------
\subsection{\acs*{ska1low}~阵列布局}

\acl{ska1low} ...
layout configuration, design goals, descriptions

%---------------------------------------------------------------------
\subsection{模拟观测}

OSKAR simulator

%---------------------------------------------------------------------
\subsection{成像}
\label{ssec:imaging}

WSClean imager


%=====================================================================
\section{小结}

TODO


%% EOF

%%
%% Copyright (c) 2018-2019 Weitian LI <liweitianux@sjtu.edu.cn>
%% Creative Commons BY 4.0
%%

\chapter{射电晕对宇宙再电离探测的影响}
\label{chap:halo}

%=====================================================================
\section{评估方法}

TODO


%=====================================================================
\section{一维功率谱}

TODO


%=====================================================================
\section{二维功率谱}

TODO


%=====================================================================
\section{讨论}

TODO

%---------------------------------------------------------------------
\subsection{伪频谱结构的影响}

instrumental frequency artifacts

%---------------------------------------------------------------------
\subsection{远旁瓣的影响}

halos in far side-lobes


%=====================================================================
\section{小结}

TODO

此工作已发表于 \apj{} (ApJ) \cite{li.halo}。


%% EOF

\chapter{基于深度学习的再电离信号分离新算法}
\label{chap:cdae}

%=====================================================================
\section{波束的频率依赖效应}

frequency-dependent beam effects


%=====================================================================
\section{传统前景扣除方法}

TODO

%---------------------------------------------------------------------
\subsection{参数化方法}

TODO

%---------------------------------------------------------------------
\subsection{非参数化方法}

TODO


%=====================================================================
\section{基于深度学习的新算法}

传统方法的严重不足,机器学习/深度学习方法的必要性,
从数据中学习

%---------------------------------------------------------------------
\subsection{深度学习简介}

TODO

%---------------------------------------------------------------------
\subsection{卷积去噪自编码器}

TODO

%---------------------------------------------------------------------
\subsection{网络结构设计}

TODO

%---------------------------------------------------------------------
\subsection{训练和评估方法}

TODO


%=====================================================================
\section{新算法的演示}

experiment, demonstration

%---------------------------------------------------------------------
\subsection{数据集}

TODO

%---------------------------------------------------------------------
\subsection{数据预处理}

TODO

%---------------------------------------------------------------------
\subsection{训练}

TODO

%---------------------------------------------------------------------
\subsection{结果}

TODO


%=====================================================================
\section{讨论}

TODO

%---------------------------------------------------------------------
\subsection{不使用 Fourier 变换的情形}

TODO

%---------------------------------------------------------------------
\subsection{与多项式拟合方法的对比}

TODO


%=====================================================================
\section{小结}

TODO

此工作已发表于 Monthly Notices of the Royal Astronomical Society
Letters \cite{li2018cdae}。


%% EOF

\begin{summary}

这里是全文总结内容。

\end{summary}


\appendix

%%
%% Copyright (c) 2018 Weitian LI <liweitianux@sjtu.edu.cn>
%% Creative Commons BY 4.0
%%

\chapter{补充公式}
\label{chap:formulas}


在本文所采用的平直 \lcdm/ 宇宙中,
\ac{delta-crit}随红移的变化关系可表示为 \cite{kitayama1996,randall2002}:
\begin{equation}
  \label{eq:delta-crit}
  \acs{delta-crit} = \frac{D(z=0)}{\acs{Dz}}
    \left[ \frac{3 (12\pi)^{2/3}}{20} \right]
    \left[1 + 0.0123 \log_{10} \acs{Ofz} \right] ,
\end{equation}
其中 \acs{Ofz} 是\acl{Ofz}:
\begin{equation}
  \label{eq:omega-fz}
  \acs{Ofz} = \frac{\acs{Om0} (1+z)^3}{\acs{Om0} (1+z)^3 + \acs{Ol0}} ,
\end{equation}
\acs{Dz} 是\acl{Dz},可由下述公式计算:
\begin{equation}
  \label{eq:growth-factor}
  D(x) = \frac{(x^3 + 2)^{1/2}}{x^{3/2}}
    \mathlarger{\int_0^x} y^{3/2} (y^3 + 2)^{-3/2} \,\D{y} ,
\end{equation}
并且 $x_0 \equiv (2 \acs{Ol0}/\acs{Om0})^{1/3}$、$x = x_0 / (1+z)$
[\textcite{peebles1980}, 式~(13.6)]。

在红移 $z$ 时的宇宙年龄具有如下解析计算形式:
\begin{align}
  \label{eq:universe-age}
  t(z; \acs{Om0})
    & = \frac{1}{\acs{H0}} \mathlarger{\int_z^{\infty}}
      \!\frac{\D{z'}}{(1+z')\sqrt{1 + z' (3+3z'+z'^2) \acs{Om0}}}
      \nonumber \\
    & = \frac{2}{3 \acs{H0} \sqrt{1-\acs{Om0}}} \sinh^{-1}
      \!\left( \sqrt{\frac{\acs{Om0}^{-1} - 1}{(1+z)^3}} \right),
\end{align}
[参见 \textcite{thomas2000}, 式~(18)]。

\acl{Hz}为:
\begin{equation}
  \label{eq:hubble-z}
  \acs{Hz} = \acs{H0} \, \acs{Ez}
    = \acs{H0} \sqrt{\acs{Om0} (1+z)^3 + \acs{Ol0}} ,
\end{equation}
其中 \acs{Ez} 是\acl{Ez} \cite{hogg1999}。
此时的宇宙临界密度为:
\begin{equation}
  \label{eq:rho-crit}
  \acs{rho-crit} = \frac{3 H^2(z)}{8 \pi \acs{G}} ,
\end{equation}
其中 \acs{G} 是\acl{G}。

星系团的\ac{r-vir}由下式给出:
\begin{equation}
  \label{eq:radius-virial}
  \acs{r-vir} = \left[
    \frac{3 \acs{M-vir}}{4\pi \acs{Delta-vir} \acs{rho-crit}}
  \right]^{1/3},
\end{equation}
其中 \acs{M-vir} 是星系团的\acl{M-vir}(亦可当作其总质量)、
\acs{Delta-vir} 是\acl{Delta-vir},由下式给出
\cite{kitayama1996,cassano2005}:
\begin{equation}
  \label{eq:delta-vir}
  \acs{Delta-vir} = 18\pi^2 \left[ 1 + 0.4093 \, w(z)^{0.9052} \right],
\end{equation}
并且 $w(z) \equiv \Omega_f^{-1}(z) - 1$。


The \emph{angular diameter distance} $D_A$ is defined as the ratio of
an object's physical transverse size to its (observed) angular size
(in radians).  Note that it does \emph{not} increases indefinitely
as $z \to \infty$, therefore more distant (e.g., $z > 1$)
objects of same physical size may actually appear larger!
The angular diameter distance is used to convert the observed angular
separations between sources into their proper separations, and it is
related to the \emph{transverse comoving distance} $D_M$ by
\cite{weinberg1972,peebles1993,hogg1999}:
\begin{equation}
  \label{eq:da-dm}
  D_A(z) = \frac{D_M(z)}{1 + z}.
\end{equation}

The \emph{luminosity distance} $D_L$ is defined by the relationship
between the measured bolometric (i.e., integrated over all frequencies)
flux $S_{\R{bolo}}$ and the object's intrinsic bolometric luminosity
$L_{\R{bolo}}$:
\begin{equation}
  \label{eq:dl-def}
  D_L \equiv \sqrt{\frac{L_{\R{bolo}}}{4\pi S_{\R{bolo}}}}.
\end{equation}
And it is related to the transverse comoving distance and angular
diameter distance by \cite{weinberg1972,hogg1999,ellis2007}:
\begin{equation}
  \label{eq:dl-dm-da}
  D_L(z) = (1+z) D_M(z) = (1+z)^2 D_A(z).
\end{equation}

In the radio frequency regime, the Rayleigh-Jeans approximation
always holds, therefore the spectral brightness $I_{\nu}$ can be
equivalently expressed by \emph{brightness temperature} $T_b(\nu)$
through the relation \cite{condon2016}:
\begin{equation}
  \label{eq:brightness-temp}
  T_b(\nu) \equiv \frac{I_{\nu} c^2}{2 k_B \nu^2}.
\end{equation}


%% EOF

%%
%% Copyright (c) 2018-2019 Weitian LI <liweitianux@sjtu.edu.cn>
%% Creative Commons BY 4.0
%%

\chapter{Fokker--Planck 方程数值算法}
\label{chap:fpsolver}

在磁流体中,带电粒子可与其中的湍流发生随机散射而通过\emph{二阶 Fermi 加速}机制
获得能量\cite{fermi1949,fermi1954,davis1956},
该过程可由 Fokker--Planck 方程描述
\cite{schlickeiser1989,eilek1991,schlickeiser2002}.
当加速区域是均匀的且远大于散射的\ac{mfp}时,Fokker--Planck 方程可被简化到只
依赖于时间和能量\cite{park1995,park1996}:
\begin{equation}
  \label{eq:fp-generic}
  \pdiff{u(x,t)}{t} = \frac{1}{A(x)} \pdiff{}{x}
    \left[ B(x) u(x,t) + C(x) \pdiff{u(x,t)}{x} \right]
    - \frac{u(x,t)}{T(x)} + Q(x) ,
\end{equation}
其中
$x$ 是能量或动量,
$u(x,t)$ 为粒子的能量分布,
$A(x)$ 为相位因子(如果 $x$ 表示能量,该项等于 1;
如果 $x$ 表示动量,该项等于 $4\Cpi x^2$),
$B(x)$、$C(x)$、$T(x)$ 和 $Q(x)$ 分别描述了粒子的
平流 (advection)、扩散 (diffusion)、逃逸 (escape) 和注入 (injection).
这几个系数需满足 $A(x) > 0, C(x) > 0, T(x) \ge 0, Q(x) \ge 0$.

然而,简化后的 Fokker--Planck 方程仍然只能在有限的几种特殊情况下获得解析解,
而对于一般情况则必须求助于数值算法.
由 \citeay{chang1970} 提出的有限差分法 (finite difference scheme)
是一种有效的算法,下文将具体介绍该算法.


%=====================================================================
\section{数值算法}

采用一个包含 $M+1$ 个点的网格对 $x$ 离散化:$x_m (m = 0, 1, \cdots, M)$.
在网格单元中点处,$x$ 的值定义为:
\begin{equation}
  \label{eq:x-mid}
  x_{m+1/2} = (x_m + x_{m+1}) / 2 ,
\end{equation}
同时 $\Delta x$ 的值定义为:
\begin{equation}
  \label{eq:dx-mid}
  \Delta x_{m+1/2} = x_{m+1} - x_m ,
\end{equation}
于是可得:
\begin{equation}
  \label{eq:dx}
  \Delta x_m = (x_{m+1} - x_{m-1}) / 2 .
\end{equation}
对时间 $t$ 离散化,并采用记法:
\begin{equation}
  \label{eq:u-t}
  u_m^n = u(x_m, t_n) .
\end{equation}

接着,定义 $x$-空间的粒子流量 $F(x,t)$ 为:
\begin{equation}
  \label{eq:fp-f}
  F(x,t) = B(x) u(x,t) + C(x) \pdiff{u(x,t)}{x} .
\end{equation}
于是\emph{无流量 (no-flux) 边界条件}可写为\cite{park1995}:
\begin{equation}
  \label{eq:no-flux}
  F(x_0, t) = F(x_M, t) = 0 .
\end{equation}

对\autoref{eq:fp-generic} 离散化可得:
\begin{equation}
  \label{eq:fp-disc}
  \frac{u_m^{n+1} - u_m^n}{\Delta t}
    = \frac{1}{A_m} \frac{F_{m+1/2}^{n+1} - F_{m-1/2}^{n+1}}{\Delta x_m}
      - \frac{u_m^{n+1}}{T_m} + Q_m ,
\end{equation}
其中 $\Delta t = t_{n+1} - t_n$ 为时间步长.
同时\autoref{eq:no-flux} 的无流量边界条件成为:
\begin{equation}
  \label{eq:no-flux-disc}
  F_{-1/2}^{n+1} = F_{M+1/2}^{n+1} = 0 .
\end{equation}

\citeay{chang1970} 给出如下 $F_{m+1/2}^{n+1}$ 的表达式:
\begin{align}
  \label{eq:fp-f-chang70}
  F_{m+1/2}^{n+1} & = (1 - \delta_{m+1/2}) B_{m+1/2} u_{m+1}^{n+1}
      + \delta_{m+1/2} B_{m+1/2} u_m^{n+1}
      + C_{m+1/2} \frac{u_{m+1}^{n+1} - u_m^{n+1}}{\Delta x_{m+1/2}} \\
    & = \frac{C_{m+1/2}}{\Delta x_{m+1/2}} \left[
      W_{m+1/2}^{+} u_{m+1}^{n+1} - W_{m+1/2}^{-} u_m^{n+1} \right] ,
\end{align}
其中
\begin{align}
  \delta_m & = \frac{1}{w_m} - \frac{1}{\exp(w_m) - 1} ,
    \label{eq:fp-delta-m} \\
  W_m^{\pm} & = W_m \exp(\pm w_m / 2) ,
    \label{eq:fp-Wm-pm} \\
  W_m & = w_m / [2 \sinh(w_m / 2)] ,
    \label{eq:fp-Wm} \\
  w_m & = \frac{B_m}{C_m} \Delta x_m .
    \label{eq:fp-wm}
\end{align}
考虑到 $|w_m|$ 可能会非常大或者非常小,为了使数值计算更稳定,可采用\cite{park1996}:
\begin{equation}
  \label{eq:fp-Wm-calc}
  W_m = \left\{
    \begin{alignedat}{2}
      & \left[ 1 + \frac{w_m^2}{24} + \frac{w_m^4}{1920} \right]^{-1} ,
        & \quad\text{when~} |w_m| < 0.1 , \\
      & \frac{|w_m| \exp(-|w_m|/2)}{1 - \exp(-|w_m|)} ,
        & \quad\text{when~} |w_m| \ge 0.1 .
    \end{alignedat}
  \right.
\end{equation}

将\autoref{eq:fp-f-chang70} 代入\autoref{eq:fp-disc},
可整理成如下三对角 (tridigonal) 线性方程组:
\begin{equation}
  \label{eq:fp-tridigonal}
  \left\{
    \begin{aligned}
      -a_m u_{m-1}^{n+1} + b_m u_m^{n+1} - c_m u_{m+1}^{n+1} & = r_m, \\
      a_0 = c_M & = 0 ,
    \end{aligned}
  \right.
\end{equation}
其中各项系数如下:
\begin{equation}
  \label{eq:fp-coefs}
  \left\{
    \begin{aligned}
      a_m & = \frac{\Delta t}{A_m \Delta x_m}
        \frac{C_{m-1/2}}{\Delta x_{m-1/2}} W_{m-1/2}^{-} , \\
      c_m & = \frac{\Delta t}{A_m \Delta x_m}
        \frac{C_{m+1/2}}{\Delta x_{m+1/2}} W_{m+1/2}^{+} , \\
      b_m & = 1 + \frac{\Delta t}{A_m \Delta x_m}
        \left[ \frac{C_{m-1/2}}{\Delta x_{m-1/2}} W_{m-1/2}^{+}
        + \frac{C_{m+1/2}}{\Delta x_{m+1/2}} W_{m+1/2}^{-} \right]
        + \frac{\Delta t}{T_m} , \\
      r_m & = u_m^n + \Delta t Q_m .
    \end{aligned}
  \right.
\end{equation}
注意,上式无法给出 $b_0$ 和 $b_M$,这需要利用边界条件[\autoref{eq:no-flux-disc}]
重新推导系数,可得:
\begin{equation}
  \label{eq:fp-coefs-b}
  \left\{
    \begin{aligned}
      b_0 & = 1 + \frac{\Delta t}{A_0 \Delta x_0}
        \frac{C_{1/2}}{\Delta x_{1/2}} W_{1/2}^{-}
        + \frac{\Delta t}{T_0} , \\
      b_M & = 1 + \frac{\Delta t}{A_M \Delta x_M}
        \frac{C_{M-1/2}}{\Delta x_{M-1/2}} W_{M-1/2}^{+}
        + \frac{\Delta t}{T_M} .
    \end{aligned}
  \right.
\end{equation}
\autoref{eq:fp-tridigonal} 的线性方程组可由快速的三对角矩阵算法
(亦称 Thomas 算法)求解\cite{press1992}.


%=====================================================================
\section{算法测试}

简单/解析情形对比...

%% EOF

\chapter{已观测到的射电晕目录}
\label{chap:halos-observed}

\begin{table}[!h]
  \centering
  \bicaption[已观测到的射电晕目录]{%
    目前已观测到的 71 个射电晕及 9 个候选者(截至 2018 年 1 月)
  }{%
    Currently observed 71 radio halos and 9 candidates
    (As of 2018 January)
  }
  \label{tab:halos}
  \begin{tabular}{cc}
    \toprule
    hello & world \\
    \midrule
    hello & world \\
    \bottomrule
  \end{tabular}
\end{table}

%% EOF


%---------------------------------------------------------------------
\backmatter

\printbibliography[heading=bibintoc]

\printacronyms[
  include-classes=glossary,
  name={主要术语对照表},
]


% 致谢、发表论文、申请专利、参与项目、简历
% 用于盲审的论文需隐去致谢、发表论文、申请专利、参与的项目

\makeatletter
% 盲审删去删去致谢页
\ifsjtu@review\relax\else
  %%
%% Copyright (c) 2018 Weitian LI <liweitianux@sjtu.edu.cn>
%% Creative Commons BY 4.0
%%

\begin{thanks}

首先感谢父母和亲人,是他们的无私关爱和支持让我得以取得今天的成绩。

感谢导师徐海光教授的悉心培养,让我获得做人与做学问的全面成长。

感谢师兄师姐的指导,特别是王婧颖师姐。
感谢课题组里一起奋斗的小伙伴:
朱正浩、胡丹、马志贤、单晨曦、郑东超、朱永凯、连晓丽、刘宇星。
感谢好友朱睿敏以及 Jeffrey Hsu 对论文的帮助。
还需要感谢国家天文台和上海天文台的老师、学长和学姐的关心和指导。

感谢国家自然科学基金委
(项目编号:11433002、11621303、11835009、61371147、11125313)
和科学技术部(项目编号:2018YFA0404601、2017YFF0210903)
为本工作提供的资助。

上海交通大学的互联网是国内首屈一指的,如果没有这个优良条件,这里的一切都将是浮云。
我也将无法忘记在这里结识的一群好朋友。

本工作的完成离不开以下项目/工具/网站的支持:
\href{https://github.com/adobe-fonts}{Adobe 开源字体} (%
\href{https://github.com/adobe-fonts/source-han-serif}{思源\textbf{宋体}},
\href{https://github.com/adobe-fonts/source-serif-pro}{Source \textbf{Serif} \textit{Pro}},
\href{https://github.com/adobe-fonts/source-sans-pro}{\textsf{Source \textbf{Sans} \textit{Pro}}},
\href{https://github.com/adobe-fonts/source-code-pro}{\texttt{Source \textbf{Code} \textit{Pro}}}),
\href{https://arxiv.org/}{arXiv},
\href{http://ads.harvard.edu/}{Astrophysics Data System},
\href{https://www.bing.com/dict}{Bing 词典},
\href{https://www.debian.org/}{Debian GNU/Linux},
\href{https://www.gnu.org/s/emacs/}{Emacs},
\href{https://fcitx-im.org/}{Fcitx},
\href{https://git-scm.com/}{Git},
\href{https://github.com/}{Github},
\href{https://www.google.com/}{Google 搜索},
\href{https://github.com/sjtug/SJTUThesis}{交大学位论文模板},
\href{https://keras.io/}{Keras},
\href{https://www.latex-project.org/}{\LaTeX},
\href{https://www.libreoffice.org/}{LibreOffice},
\href{https://www.mozilla.org/en-US/firefox/}{Mozilla Firefox},
\href{https://ned.ipac.caltech.edu/}{NASA/IPAC Extragalactic Database},
\href{https://okular.kde.org/}{Okular},
\href{https://www.openssh.com/}{OpenSSH},
\href{https://github.com/OxfordSKA/OSKAR}{OSKAR},
\href{https://www.python.org/}{Python} (%
\href{https://www.astropy.org/}{Astropy},
\href{https://jupyter.org/}{Jupyter},
\href{https://matplotlib.org/}{matplotlib},
\href{https://www.numpy.org/}{NumPy},
\href{https://pandas.pydata.org/}{pandas},
\href{https://scipy.org/}{SciPy}),
\href{http://ds9.si.edu/}{SAOImage DS9},
\href{https://shadowsocks.org/}{ShadowSocks},
\href{https://stackoverflow.com/}{Stack Overflow},
\href{https://syncthing.net/}{Syncthing},
\href{https://www.tensorflow.org/}{TensorFlow},
\href{https://github.com/tmux/tmux}{Tmux},
\href{https://www.vim.org/}{Vim},
\href{https://www.wechat.com/}{微信},
\href{https://www.wikipedia.org/}{Wikipedia},
\href{http://wps-community.org/}{WPS Office},
\href{https://sourceforge.net/projects/wsclean/}{WSClean},
\href{https://www.xfce.org/}{XFCE},
\href{http://www.zsh.org/}{Zsh}。
此外,感谢 \href{https://www.dragonflybsd.org/}{DragonFly BSD}
项目及其 IRC 上那群很棒的人。

最后,感谢女友尹璐璐,感谢她十年来坚定不移的支持和鼓励,她是我今生的至爱。

\end{thanks}

\fi
\ifsjtu@bachelor
  % 本科学位论文要求在最后有一个英文大摘要,单独编页码
  \pagestyle{biglast}
  \include{tex/app-abstract}
\else
  % 盲审论文中,发表学术论文及参与科研情况等仅以第几作者注明即可,
  % 不要出现作者或他人姓名
  %%
%% Copyright (c) 2018 Weitian LI <liweitianux@sjtu.edu.cn>
%% Creative Commons BY 4.0
%%

\begin{publications}{99}
  \linespread{1.1}

  \item
    \textsc{\emph{Li, Weitian}; Xu, Haiguang; Ma, Zhixian; Zhu, Ruimin;
    Hu, Dan; Zhu, Zhenghao; Shan, Chenxi; Zhu, Jie; Wu, Xiang-Ping}.
    \enquote{\it Separating the EoR Signal with a Convolutional Denoising
      Autoencoder: a Deep-learning-based Method,}
    2018, Monthly Notices of the Royal Astronomical Society Letters, ???
  \item
    \textsc{\emph{Li, Weitian}; Xu, Haiguang; Ma, Zhixian; Hu, Dan;
    Zhu, Zhenghao; Shan, Chenxi; Wang, Jingying; Gu, Junhua;
    Lian, Xiaoli; Zheng, Qian; Zhu, Jie; Wu, Xiang-Ping}.
    \enquote{\it Contribution of Radio Halos to the Foreground for
      SKA EoR Experiments,}
    2018, The Astrophysical Journal, ???
  \item
    \textsc{Ma, Zhixian; Xu, Haiguang; Zhu, Jie; Hu, Dan;
    \emph{Li, Weitian}; Shan, Chenxi; Zhu, Zhenghao; Gu, Liyi;
    Liu, Chengze; Wu, Xiang-Ping}.
    \enquote{\it A Machine Learning Based Morphological Classification
      of 14,251 Radio AGNs Selected from the Best--Heckman Sample,}
    2018, The Astrophysical Journal Supplement Series, ???
  \item
    \textsc{Hu, Dan; Xu, Haiguang; Kang, Xi; \emph{Li, Weitian};
    Zhu, Zhenghao; Ma, Zhixian; Shan, Chenxi; Zhang, Zhongli;
    Gu, Liyi; Liu, Chengze; Wu, Xiang-Ping}.
    \enquote{\it A Study of the Merger History of the Galaxy Group
      HCG 62 Based on X-ray Observations and SPH Simulations,}
    2018, The Astrophysical Journal, ???,
    \arxiv{1811.05782}
  \item
    \textsc{Zheng, Qian; Johnston-Hollitt, Melanie;
    Duchesne, Stefan\,W; \emph{Li, Weitian}}.
    \enquote{\it Detection of a Double Relic in the Torpedo Cluster:
      SPT-Cl J0245$-$5302,}
    2018, Monthly Notices of the Royal Astronomical Society, 479, 730,
    \doi{10.1093/mnras/sty1467},
    \arxiv{1803.06634}
  \item
    \textsc{Ma, Zhixian; Zhu, Jie; \emph{Li, Weitian}; Xu, Haiguang}.
    \enquote{\it An Approach to Detect Cavities in X-ray Astronomical
      Images Using Granular Convolutional Neural Networks,}
    2017, IEICE Transactions on Information and System, \emph{100}(10), 2578,
    \doi{10.1587/transinf.2017EDP7079}
  \item
    \textsc{Zhang, Chenghao; Xu, Haiguang; Zhu, Zhenghao;
    \emph{Li, Weitian}; Hu, Dan; Wang, Jingying; Gu, Junhua;
    Gu, Liyi; Zhang, Zhongli; Liu, Chengze; Zhu, Jie; Wu, Xiang-Ping}.
    \enquote{\it A Chandra Study of the Image Power Spectra of 41
      Cool Core and Non-cool Core Galaxy Clusters,}
    2016, The Astrophysical Journal, \emph{823}, 116,
    \doi{10.3847/0004-637X/823/2/116},
    \arxiv{1604.04127}
  \item
    \textsc{Zhu, Zhenghao; Xu, Haiguang; Wang, Jingying; Gu, Junhua;
    \emph{Li, Weitian}; Hu, Dan; Zhang, Chenghao; Gu, Liyi; An, Tao;
    Liu, Chengze; Zhang, Zhongli; Zhu, Jie; Wu, Xiang-Ping}.
    \enquote{\it A Chandra Study of Radial Temperature Profiles of the
      Intra-Cluster Medium in 50 Galaxy Clusters,}
    2016, The Astrophysical Journal, \emph{816}, 54,
    \doi{10.3847/0004-637X/816/2/54},
    \arxiv{1511.04699}
  \item
    \textsc{Wang, Jingying; Xu, Haiguang; An, Tao; Gu, Junhua;
    Guo, Xueying; \emph{Li, Weitian}; Wang, Yu; Liu, Chengze;
    Martineau-Huynh, Olivier; Wu, Xiang-Ping}.
    \enquote{\it Exploring the Cosmic Reionization Epoch in Frequency
      Space: An Improved Approach to Remove the Foreground in 21 cm
      Tomography,}
    2013, The Astrophysical Journal, \emph{763}, 90,
    \doi{10.1088/0004-637X/763/2/90},
    \arxiv{1211.6450}

  \vspace{1ex} % conference papers
  \item
    \textsc{Ma, Zhixian; Zhu, Jie; \emph{Li, Weitian}; Xu, Haiguang}.
    \enquote{\it Radio Galaxy Morphology Generation Using Residual
      Convolutional Autoencoder and Gaussian Mixture Models,}
    2018, IEEE 25th International Conference on Image Processing (ICIP),
    Athens, Greece, October 7--10, 2018, 3044--3048,
    \doi{10.1109/ICIP.2018.8451231}
  \item
    \textsc{Ma, Zhixian; Zhu, Jie; \emph{Li, Weitian}; Xu, Haiguang}.
    \enquote{\it Radio Galaxy Morphology Generation Using DNN Autoencoder
      and Gaussian Mixture Models,}
    2018, IEEE 14th International Conference on Signal Processing (ICSP),
    Beijing, China, August 12--14, 2018, 522--526,
    \arxiv{1806.00398}
  \item
    \textsc{Ma, Zhixian; Zhu, Jie; \emph{Li, Weitian}; Xu, Haiguang}.
    \enquote{\it Detection of Point Sources in X-ray Astronomical Images
      Using Elliptical Gaussian Filters,}
    2017, IEEE 2nd International Conference on Image, Vision and Computing (ICIVC),
    Chengdu, China, June 2--4, 2018, 36--40,
    \doi{10.1109/ICIVC.2017.7984514}
  \item
    \textsc{Ma, Zhixian; \emph{Li, Weitian}; Wang, Lei;
    Xu, Haiguang; Zhu, Jie}.
    \enquote{\it X-ray Astronomical Point Sources Recognition Using
      Granular Binary-tree SVM,}
    2016, IEEE 13th International Conference on Signal Processing (ICSP),
    Chengdu, China, November 6--10, 2018, 1021--1026,
    \doi{10.1109/ICSP.2016.7877984}
\end{publications}

  % \include{tex/projects}
  % \include{tex/patents}
  \ifsjtu@review\relax\else
    %%
%% Copyright (c) 2018 Weitian LI <liweitianux@sjtu.edu.cn>
%% Creative Commons BY 4.0
%%

\begin{resume}
  \begin{resumesection}{基本情况}
    李维天,男,1991 年 9 月生于湖南邵阳。
  \end{resumesection}

  \begin{resumelist}{教育背景}
    \item 2013 年 9 月至今,上海交通大学,博士研究生,物理学
    \item 2009 年 9 月至 2013 年 6 月,上海交通大学,本科,应用物理学
  \end{resumelist}

  \begin{resumesection}{研究兴趣}
    低频射电观测,宇宙再电离时期探测,数据分析
  \end{resumesection}

  \begin{resumelist}{联系方式}
    \item E-mail: \email{liweitianux@sjtu.edu.cn}, \hspace{1em} \email{wt@liwt.net}
    \item Github: \url{https://github.com/liweitianux}
  \end{resumelist}
\end{resume}

  \fi
\fi
\makeatother

\end{document}
