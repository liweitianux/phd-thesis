%%
%% Copyright (c) 2018 Weitian LI <liweitianux@sjtu.edu.cn>
%% Creative Commons BY 4.0
%%

% Class options:
%   bachelor|master|doctor	% 必选项
%   fontset=fandol|adobe|windows
%   oneside|twoside
%   openany|openright
%   zihao=-4|5			% 正文字号: 小四、五号(默认)
%   review			% 盲审论文,隐去作者姓名、学号、
%				% 导师姓名、致谢、发表论文和参与的项目
%   submit			% 定稿提交的论文,插入签名扫描版的
%				% 原创性声明、授权声明
\documentclass[doctor, fontset=fandol, openright, twoside]{sjtuthesis}


%% Fonts
\defaultfontfeatures{Mapping=tex-text}
\setmainfont{TeX Gyre Pagella}
\setsansfont{TeX Gyre Heros}
\setmonofont{M+ 1mn}
\setmathfont{TeX Gyre Pagella Math}

%% Chinese
\setCJKmainfont[ItalicFont={Noto Sans CJK SC}]{Noto Serif CJK SC}
\setCJKsansfont{AR PL KaitiM GB}
\setCJKmonofont{Noto Sans Mono CJK SC}
\xeCJKsetup{PunctStyle=kaiming}

% References database
\addbibresource{bib/thesis.bib}


%%
%% Title pages
%%

\title{射电晕对探测宇宙再电离信号的影响}
\author{李维天}
\studentnumber{0130729026}
\advisor{徐海光 教授}
%\coadvisor{某某 教授}
\school{上海交通大学}
\institute{物理与天文学院}
\major{物理学}
\defenddate{2018 年 mm 月 dd 日}

\englishtitle{\textsc{%
  The Impacts of Radio Halos on Detecting the
  Epoch of Reionization Signal
}}
\englishauthor{\textsc{Weitian Li}}
\englishadvisor{Prof. \textsc{Haiguang Xu}}
%\englishcoadvisor{Prof. \textsc{Uom Uom}}
\englishinstitute{School of Physics and Astronomy}
\englishschool{Shanghai Jiao Tong University}
\englishlocation{Shanghai, China}
\englishmajor{Physics}
\englishdate{mmm dd, 2018}


\begin{document}

\maketitle

\makeatletter
\ifsjtu@submit\relax
  \includepdf{pdf/originality.pdf}
  \pdfbookmark[0]{\sjtu@label@originality}{originality}
  \cleardoublepage
  \includepdf{pdf/authorization.pdf}
  \pdfbookmark[0]{\sjtu@label@authorization}{authorization}
  \cleardoublepage
\else
\ifsjtu@review\relax
% exclude the originality and authorization declarations
\else
  \makeDeclareOriginality
  \makeDeclareAuthorization
\fi
\fi
\makeatother

\frontmatter
\pagestyle{main}

% 中文摘要,约 3000 字
\begin{abstract}
TODO
\end{abstract}

%---------------------------------------------------------------------

\begin{englishabstract}
TODO
\end{englishabstract}


\tableofcontents
\listoffigures
\addcontentsline{toc}{chapter}{\listfigurename}
\listoftables
\addcontentsline{toc}{chapter}{\listtablename}
\listofalgorithms
\addcontentsline{toc}{chapter}{\listalgorithmname}

%%
%% Copyright (c) 2018 Weitian LI <liweitianux@sjtu.edu.cn>
%% Creative Commons BY 4.0
%%

\DeclareAcronym{h}{
  short = \si{\hubble},
  long = 无量纲的 Hubble 常数,
  class = symbol,
}

\DeclareAcronym{H0}{
  short = \ensuremath{H_0},
  long = 当前的 Hubble 常数,
  sort = H,
  class = symbol,
}

\DeclareAcronym{ns}{
  short = \ensuremath{n_s},
  long = 原初扰动的标量谱指数,
  class = symbol,
}

\DeclareAcronym{Ob0}{
  short = \ensuremath{\Omega_b},
  long = 当前的宇宙重子物质密度,
  sort = Omega-b,
  class = symbol,
}

\DeclareAcronym{Om0}{
  short = \ensuremath{\Omega_m},
  long = 当前的宇宙物质密度(包含重子物质和暗物质),
  sort = Omega-m,
  class = symbol,
}

\DeclareAcronym{Ol0}{
  short = \ensuremath{\Omega_{\Lambda}},
  long = 当前的宇宙常数或暗能量密度,
  sort = Omega-Lambda,
  class = symbol,
}

\DeclareAcronym{sigma8}{
  short = \ensuremath{\sigma_8},
  long = 原初扰动在 \SI{8}{\per\hubble\Mpc} 尺度上的幅度,
  class = symbol,
}

\endinput
 % 主要符号、缩略词对照表

\mainmatter
\pagestyle{main}

\include{tex/intro}
\include{tex/example}
\include{tex/faq}
\begin{summary}

这里是全文总结内容。

\end{summary}


\appendix
\renewcommand\theequation{\Alph{chapter}--\arabic{equation}}
\renewcommand\thefigure{\Alph{chapter}--\arabic{figure}}
\renewcommand\thetable{\Alph{chapter}--\arabic{table}}
\renewcommand\thealgorithm{\Alph{chapter}--\arabic{algorithm}}

%% 附录内容,本科学位论文可以用翻译的文献替代。
\include{tex/app_setup}
\include{tex/app_eq}
\include{tex/app_cjk}
\include{tex/app_log}

\backmatter

\printbibliography[heading=bibintoc]

% 致谢、发表论文、申请专利、参与项目、简历
% 用于盲审的论文需隐去致谢、发表论文、申请专利、参与的项目

\makeatletter
% 盲审删去删去致谢页
\ifsjtu@review\relax\else
  %%
%% Copyright (c) 2018 Weitian LI <liweitianux@sjtu.edu.cn>
%% Creative Commons BY 4.0
%%

\begin{thanks}

首先感谢父母和亲人,是他们的无私关爱和支持让我得以取得今天的成绩。

感谢导师徐海光教授的悉心培养,让我获得做人与做学问的全面成长。

感谢师兄师姐的指导,特别是王婧颖师姐。
感谢课题组里一起奋斗的小伙伴:
朱正浩、胡丹、马志贤、单晨曦、郑东超、朱永凯、连晓丽、刘宇星。
感谢好友朱睿敏以及 Jeffrey Hsu 对论文的帮助。
还需要感谢国家天文台和上海天文台的老师、学长和学姐的关心和指导。

感谢国家自然科学基金委
(项目编号:11433002、11621303、11835009、61371147、11125313)
和科学技术部(项目编号:2018YFA0404601、2017YFF0210903)
为本工作提供的资助。

上海交通大学的互联网是国内首屈一指的,如果没有这个优良条件,这里的一切都将是浮云。
我也将无法忘记在这里结识的一群好朋友。

本工作的完成离不开以下项目/工具/网站的支持:
\href{https://github.com/adobe-fonts}{Adobe 开源字体} (%
\href{https://github.com/adobe-fonts/source-han-serif}{思源\textbf{宋体}},
\href{https://github.com/adobe-fonts/source-serif-pro}{Source \textbf{Serif} \textit{Pro}},
\href{https://github.com/adobe-fonts/source-sans-pro}{\textsf{Source \textbf{Sans} \textit{Pro}}},
\href{https://github.com/adobe-fonts/source-code-pro}{\texttt{Source \textbf{Code} \textit{Pro}}}),
\href{https://arxiv.org/}{arXiv},
\href{http://ads.harvard.edu/}{Astrophysics Data System},
\href{https://www.bing.com/dict}{Bing 词典},
\href{https://www.debian.org/}{Debian GNU/Linux},
\href{https://www.gnu.org/s/emacs/}{Emacs},
\href{https://fcitx-im.org/}{Fcitx},
\href{https://git-scm.com/}{Git},
\href{https://github.com/}{Github},
\href{https://www.google.com/}{Google 搜索},
\href{https://github.com/sjtug/SJTUThesis}{交大学位论文模板},
\href{https://keras.io/}{Keras},
\href{https://www.latex-project.org/}{\LaTeX},
\href{https://www.libreoffice.org/}{LibreOffice},
\href{https://www.mozilla.org/en-US/firefox/}{Mozilla Firefox},
\href{https://ned.ipac.caltech.edu/}{NASA/IPAC Extragalactic Database},
\href{https://okular.kde.org/}{Okular},
\href{https://www.openssh.com/}{OpenSSH},
\href{https://github.com/OxfordSKA/OSKAR}{OSKAR},
\href{https://www.python.org/}{Python} (%
\href{https://www.astropy.org/}{Astropy},
\href{https://jupyter.org/}{Jupyter},
\href{https://matplotlib.org/}{matplotlib},
\href{https://www.numpy.org/}{NumPy},
\href{https://pandas.pydata.org/}{pandas},
\href{https://scipy.org/}{SciPy}),
\href{http://ds9.si.edu/}{SAOImage DS9},
\href{https://shadowsocks.org/}{ShadowSocks},
\href{https://stackoverflow.com/}{Stack Overflow},
\href{https://syncthing.net/}{Syncthing},
\href{https://www.tensorflow.org/}{TensorFlow},
\href{https://github.com/tmux/tmux}{Tmux},
\href{https://www.vim.org/}{Vim},
\href{https://www.wechat.com/}{微信},
\href{https://www.wikipedia.org/}{Wikipedia},
\href{http://wps-community.org/}{WPS Office},
\href{https://sourceforge.net/projects/wsclean/}{WSClean},
\href{https://www.xfce.org/}{XFCE},
\href{http://www.zsh.org/}{Zsh}。
此外,感谢 \href{https://www.dragonflybsd.org/}{DragonFly BSD}
项目及其 IRC 上那群很棒的人。

最后,感谢女友尹璐璐,感谢她十年来坚定不移的支持和鼓励,她是我今生的至爱。

\end{thanks}

\fi
\ifsjtu@bachelor
  % 本科学位论文要求在最后有一个英文大摘要,单独编页码
  \pagestyle{biglast}
  \include{tex/app-abstract}
\else
  % 盲审论文中,发表学术论文及参与科研情况等仅以第几作者注明即可,
  % 不要出现作者或他人姓名
  \ifsjtu@review\relax
    \include{tex/publications-review}
    \include{tex/projects-review}
  \else
    %%
%% Copyright (c) 2018 Weitian LI <liweitianux@sjtu.edu.cn>
%% Creative Commons BY 4.0
%%

\begin{publications}{99}
  \linespread{1.1}

  \item
    \textsc{\emph{Li, Weitian}; Xu, Haiguang; Ma, Zhixian; Zhu, Ruimin;
    Hu, Dan; Zhu, Zhenghao; Shan, Chenxi; Zhu, Jie; Wu, Xiang-Ping}.
    \enquote{\it Separating the EoR Signal with a Convolutional Denoising
      Autoencoder: a Deep-learning-based Method,}
    2018, Monthly Notices of the Royal Astronomical Society Letters, ???
  \item
    \textsc{\emph{Li, Weitian}; Xu, Haiguang; Ma, Zhixian; Hu, Dan;
    Zhu, Zhenghao; Shan, Chenxi; Wang, Jingying; Gu, Junhua;
    Lian, Xiaoli; Zheng, Qian; Zhu, Jie; Wu, Xiang-Ping}.
    \enquote{\it Contribution of Radio Halos to the Foreground for
      SKA EoR Experiments,}
    2018, The Astrophysical Journal, ???
  \item
    \textsc{Ma, Zhixian; Xu, Haiguang; Zhu, Jie; Hu, Dan;
    \emph{Li, Weitian}; Shan, Chenxi; Zhu, Zhenghao; Gu, Liyi;
    Liu, Chengze; Wu, Xiang-Ping}.
    \enquote{\it A Machine Learning Based Morphological Classification
      of 14,251 Radio AGNs Selected from the Best--Heckman Sample,}
    2018, The Astrophysical Journal Supplement Series, ???
  \item
    \textsc{Hu, Dan; Xu, Haiguang; Kang, Xi; \emph{Li, Weitian};
    Zhu, Zhenghao; Ma, Zhixian; Shan, Chenxi; Zhang, Zhongli;
    Gu, Liyi; Liu, Chengze; Wu, Xiang-Ping}.
    \enquote{\it A Study of the Merger History of the Galaxy Group
      HCG 62 Based on X-ray Observations and SPH Simulations,}
    2018, The Astrophysical Journal, ???,
    \arxiv{1811.05782}
  \item
    \textsc{Zheng, Qian; Johnston-Hollitt, Melanie;
    Duchesne, Stefan\,W; \emph{Li, Weitian}}.
    \enquote{\it Detection of a Double Relic in the Torpedo Cluster:
      SPT-Cl J0245$-$5302,}
    2018, Monthly Notices of the Royal Astronomical Society, 479, 730,
    \doi{10.1093/mnras/sty1467},
    \arxiv{1803.06634}
  \item
    \textsc{Ma, Zhixian; Zhu, Jie; \emph{Li, Weitian}; Xu, Haiguang}.
    \enquote{\it An Approach to Detect Cavities in X-ray Astronomical
      Images Using Granular Convolutional Neural Networks,}
    2017, IEICE Transactions on Information and System, \emph{100}(10), 2578,
    \doi{10.1587/transinf.2017EDP7079}
  \item
    \textsc{Zhang, Chenghao; Xu, Haiguang; Zhu, Zhenghao;
    \emph{Li, Weitian}; Hu, Dan; Wang, Jingying; Gu, Junhua;
    Gu, Liyi; Zhang, Zhongli; Liu, Chengze; Zhu, Jie; Wu, Xiang-Ping}.
    \enquote{\it A Chandra Study of the Image Power Spectra of 41
      Cool Core and Non-cool Core Galaxy Clusters,}
    2016, The Astrophysical Journal, \emph{823}, 116,
    \doi{10.3847/0004-637X/823/2/116},
    \arxiv{1604.04127}
  \item
    \textsc{Zhu, Zhenghao; Xu, Haiguang; Wang, Jingying; Gu, Junhua;
    \emph{Li, Weitian}; Hu, Dan; Zhang, Chenghao; Gu, Liyi; An, Tao;
    Liu, Chengze; Zhang, Zhongli; Zhu, Jie; Wu, Xiang-Ping}.
    \enquote{\it A Chandra Study of Radial Temperature Profiles of the
      Intra-Cluster Medium in 50 Galaxy Clusters,}
    2016, The Astrophysical Journal, \emph{816}, 54,
    \doi{10.3847/0004-637X/816/2/54},
    \arxiv{1511.04699}
  \item
    \textsc{Wang, Jingying; Xu, Haiguang; An, Tao; Gu, Junhua;
    Guo, Xueying; \emph{Li, Weitian}; Wang, Yu; Liu, Chengze;
    Martineau-Huynh, Olivier; Wu, Xiang-Ping}.
    \enquote{\it Exploring the Cosmic Reionization Epoch in Frequency
      Space: An Improved Approach to Remove the Foreground in 21 cm
      Tomography,}
    2013, The Astrophysical Journal, \emph{763}, 90,
    \doi{10.1088/0004-637X/763/2/90},
    \arxiv{1211.6450}

  \vspace{1ex} % conference papers
  \item
    \textsc{Ma, Zhixian; Zhu, Jie; \emph{Li, Weitian}; Xu, Haiguang}.
    \enquote{\it Radio Galaxy Morphology Generation Using Residual
      Convolutional Autoencoder and Gaussian Mixture Models,}
    2018, IEEE 25th International Conference on Image Processing (ICIP),
    Athens, Greece, October 7--10, 2018, 3044--3048,
    \doi{10.1109/ICIP.2018.8451231}
  \item
    \textsc{Ma, Zhixian; Zhu, Jie; \emph{Li, Weitian}; Xu, Haiguang}.
    \enquote{\it Radio Galaxy Morphology Generation Using DNN Autoencoder
      and Gaussian Mixture Models,}
    2018, IEEE 14th International Conference on Signal Processing (ICSP),
    Beijing, China, August 12--14, 2018, 522--526,
    \arxiv{1806.00398}
  \item
    \textsc{Ma, Zhixian; Zhu, Jie; \emph{Li, Weitian}; Xu, Haiguang}.
    \enquote{\it Detection of Point Sources in X-ray Astronomical Images
      Using Elliptical Gaussian Filters,}
    2017, IEEE 2nd International Conference on Image, Vision and Computing (ICIVC),
    Chengdu, China, June 2--4, 2018, 36--40,
    \doi{10.1109/ICIVC.2017.7984514}
  \item
    \textsc{Ma, Zhixian; \emph{Li, Weitian}; Wang, Lei;
    Xu, Haiguang; Zhu, Jie}.
    \enquote{\it X-ray Astronomical Point Sources Recognition Using
      Granular Binary-tree SVM,}
    2016, IEEE 13th International Conference on Signal Processing (ICSP),
    Chengdu, China, November 6--10, 2018, 1021--1026,
    \doi{10.1109/ICSP.2016.7877984}
\end{publications}

    \include{tex/projects}
  \fi
\fi
\makeatother

\end{document}

% EOF
